\documentclass[a4paper,12pt]{article}

\usepackage[margin=2cm]{geometry}
\usepackage{amsmath,amssymb,amsthm,latexsym}
\usepackage[all]{xy}
\usepackage{showkeys}

\theoremstyle{plain}
\newtheorem{proposition}{Proposition}[section]
\newtheorem{theorem}[proposition]{Theorem}
\newtheorem{corollary}[proposition]{Corollary}
\newtheorem{lemma}[proposition]{Lemma}
\newtheorem{claim}[proposition]{Claim}
\newtheorem{definition}[proposition]{Definition}
\newtheorem{conjecture}[proposition]{Conjecture}
\theoremstyle{definition}
\newtheorem{example}[proposition]{Example}
\newtheorem{examples}[proposition]{Examples}
\newtheorem{remark}[proposition]{Remark}
\newtheorem{remarks}[proposition]{Remarks}

\newcommand{\ip}[2]{{\langle {#1} , {#2} \rangle}}
\newcommand{\mc}{\mathcal}
\newcommand{\unit}[1]{\tilde{#1}}
\newcommand{\alg}{{\operatorname{alg}}}
\newcommand{\lin}{{\operatorname{lin}}}

\begin{document}

\title{Commentary on ``Unitaires multiplicatifs et dualit\'e pour les
produits crois\'es de $C^*$-alg\`ebres''}
\author{Matthew Daws}
\maketitle

\begin{abstract}
We 
\end{abstract}

\section{Notation}

The original paper assumed throughout that Hilbert spaces are separable.
We shall try hard \emph{not} to use this assumption.  Exceptions are:
Proposition~\ref{prop:4}.

We shall follow the convention that inner products are linear on the right.
We write $\otimes$ for various completed tensor products, which should
be clear by context (either Hilbert space, or the minimal C$^*$-algebraic,
tensor products).

Given $H$ a Hilbert space and $\xi\in H$, define
\[ \theta_\xi, \theta'_\xi\in\mc B(H,H\otimes H); \qquad
\theta_\xi(\eta) = \xi\otimes\eta, \quad
\theta'_\xi(\eta) = \eta\otimes\xi \qquad (\eta\in H). \]
Similarly, for $i=1,2,3$, define $\theta_{i,\xi}\in\mc B(H\otimes H,
H\otimes H\otimes H)$ by $\theta_{1,\xi}(\eta\otimes\zeta)
= \xi\otimes\eta\otimes\zeta$, $\theta_{2,\xi}(\eta\otimes\zeta)
= \eta\otimes\xi\otimes\zeta$ and $\theta_{3,\xi}(\eta\otimes\zeta)
= \eta\otimes\zeta\otimes\xi$.

For $T\in\mc B(H\otimes H)$, we define $T_{12}, T_{13}, T_{23} \in
\mc B(H\otimes H\otimes H)$ using the usual leg-numbering notation.
Notice that $T_{12} \theta_{3,\xi} = \theta_{3,\xi} T$,
$T_{13} \theta_{2,\xi} = \theta_{2,\xi} T$ and
$T_{23} \theta_{1,\xi} = \theta_{1,\xi} T$.  Similarly, if
$\Sigma\in\mc B(H\otimes H)$ denotes the ``swap map'', then
$T_{21} = \Sigma T_{12} \Sigma$, and so forth.

We shall also sometimes work with Hilbert C$^*$-modules (see,
for example, \cite{lance}.  For a Hilbert C$^*$-module $E$ over $A$,
we shall write (in a non-standard way) $\mc B(E)$ for the adjointable
maps on $E$.

Given $T\in\mc B(H\otimes H)$ and $\omega\in \mc B(H)_*$, we define
the slice maps $(\omega\otimes\iota)(T)$ and $(\iota\otimes\omega)(T)$
as usual.  Notice that
\[ \big( \xi \big| (\iota\otimes\omega)(T) \eta \big)
= \ip{\theta_\xi^* T \theta_\eta}{\omega}, \quad
\big( \xi \big| (\omega\otimes\iota)(T) \eta \big)
= \ip{{\theta'_\xi}^* T \theta'_\eta}{\omega}. \]

Given a C$^*$-algebra $A$, we denote by $\unit{A}$ the C$^*$-algebra given
by adjoining a unit, and we denote by $M(A)$ the multiplier algebra of $A$
(see \cite[3.12]{r33}).  If $J$ is a closed two-sided ideal in $A$, let
$M(A;J) = \{ m\in M(A) : mA + Am \subseteq J \}$.  Clearly $M(A;J)$ is a
sub-C$^*$-algebra of $M(A)$.  Restricting each element of $M(A)$ to $J$
defines a member of $M(J)$; indeed, for $m\in M(A)$ and $a\in J$, if $(e_i)$
is an approximate identity for $A$, then $ma = \lim_i (e_im)a \in J$, and
similarly $am\in J$.  Thus we get a $*$-homomorphism $M(A) \rightarrow M(J)$,
and so a $*$-homomorphism $M(A;J) \rightarrow M(J)$.  This latter map is
injective, as if $m\in M(A;J)$ with $mJ=\{0\}=Jm$, then for $a\in A$, and
$(f_i)$ an approximate identity for $J$, then $am = \lim a(mf_i) = 0$,
as $am\in J$; similarly $ma=0$ and so $m=0$.  Thus we can also regard
$M(A;J)$ as a sub-C$^*$-algebra of $M(J)$.

Recall that a $*$-homomorphism $\pi:A\rightarrow M(B)$ is \emph{non-degenerate}
if $\pi(e_i)\rightarrow 1$ strictly (meaning that $\lim_i \pi(e_i)b =
\lim_i b\pi(e_i) = b$ for $b\in B$) for a (or equivalently, any) approximate
identity $(e_i)$ for $A$.  This is equivalent to asking that $\pi$ extends to
a strictly-continuous, unital $*$-homomorphism $\pi:M(A)\rightarrow M(B)$.
(This notion is termed ``sp\'ecial'' in \cite{r50}).

\begin{definition}[D\'efinition 0.1]
A \emph{Hopf-C$^*$-algebra} is a pair $(A,\delta)$ where $A$ is a C$^*$-algebra
and $\delta:A\rightarrow M(\unit{A}\otimes A + A\otimes\unit{A};A\otimes A)$
is a non-degenerate $*$-homomorphism (notice that this means that
$\delta$ is a non-degenerate $*$-homomorphism $A\rightarrow M(A\otimes A)$
such that $\delta(a)(1\otimes b), \delta(a)(b\otimes 1) \in A\otimes A$)
with
\[ \xymatrix{ A \ar[r]^-{\delta} \ar[d]^{\delta} & M(A\otimes A)
\ar[d]^{\iota\otimes\delta} \\ M(A\otimes A) \ar[r]^-{\delta\otimes\iota} &
M(A\otimes A\otimes A). } \]
We call $\delta$ the \emph{coproduct} of $A$.  We say that $A$ is
right simplifiable (or left simplifiable) if $\delta(A)(1\otimes A)$ is linearly
dense in $A\otimes A$ (respectively $\delta(A)(A\otimes 1)$).  We say that
$A$ is bisimplifiable if $A$ is left and right simplifiable.
\end{definition}

Be aware that this clashes with \cite[1.1]{r2}.  Given a Hilbert space $H$,
we can form the interior tensor product (see \cite[Chapter~4]{lance})
$(H\otimes A) \otimes_\delta (A\otimes A)$.  Recall that this is the completion
of $(H\otimes A) \otimes_\alg (A\otimes A) / X$ where $X$ is the linear span of
elements of the form $(\xi\otimes ab)\otimes(c\otimes d) - (\xi\otimes a)
\otimes \delta(b)(c\otimes d)$.  A little bit of work shows that we can
identify, as $A\otimes A$-modules, the spaces $(H\otimes A) \otimes_\delta
(A\otimes A)$ and $H\otimes A\otimes A$ by the map
$(\xi\otimes a)\otimes(c\otimes d) \mapsto \xi\otimes \delta(a)(c\otimes d)$.

\begin{definition}[D\'efinition 0.2]
A \emph{coaction} of a Hopf-C$^*$-algebra on a C$^*$-algebra $B$ is a
non-degenerate $*$-homomorphism $\delta_B:B\rightarrow M(\unit{B}\otimes A;
B\otimes A)$ such that the following diagram commutes:
\[ \xymatrix{ B \ar[r]^-{\delta_B} \ar[d]^{\delta_B} & M(B\otimes A)
\ar[d]^{\iota\otimes\delta} \\ M(B\otimes A) \ar[r]^-{\delta_B\otimes\iota}
& M(B\otimes A\otimes A). } \]
(Again, this means that $\delta_B(b)(1\otimes a) \in B\otimes A$).
A C$^*$-algebra $B$ with a coaction $\delta_B$ of a Hopf-C$^*$-algebra
$(A,\delta)$ is an \emph{$A$-algebra} if additionally $\delta_B$ is injective,
and $\delta_B(B)(1\otimes A)$ is linearly dense in $B\otimes A$.
\end{definition}

\begin{definition}[D\'efinition 0.3]
Let $A$ be a Hopf-C$^*$-algebra.  A \emph{unitary corepresentation} of $A$ on a
Hilbert space (or Hilbert C$^*$-module) $H$ is a unitary $u\in\mc B(H\otimes A)$
such that $(\iota\otimes\delta)(u) = u_{12} u_{13}$; alternatively, in
\[ (H\otimes A)\otimes_\delta (A\otimes A) \cong H\otimes A\otimes A
\quad\text{we have}\quad u\otimes_\delta 1 = u_{12} u_{13}. \]
Let $B$ be a C$^*$-algebra with a coaction $\delta_B$ of $(A,\delta)$.
A \emph{covariant representation} of $(B,\delta_B)$ is a pair $(\pi,u)$ where
$\pi:B\rightarrow\mc B(H)$ is a $*$-representation, and $u$ is a unitary
corepresentation of $A$, such that $(\pi\otimes\iota)\delta_B(b)
= u(\pi(b)\otimes 1)u^*$ for each $b\in B$.
\end{definition}

Remember that $\mc B(H\otimes A) \cong M(\mc B_0(H)\otimes A)$, so if $H$ is
a Hilbert space, we can phrase the above without reference to Hilbert C$^*$-modules.

\begin{definition}[D\'efinition 0.4]
Let $B$ be a C$^*$-algebra with a coaction $\delta_B$ of $(A,\delta)$.
A unitary $u\in M(B\otimes A)$ is a \emph{cocycle for $\delta_B$} if
\[ u_{12} (\delta_B\otimes\iota)(u) = (\iota\otimes\delta)(u). \]
If $u$ is a cocycle for $\delta_B$, the map $\delta_{B,u}:B\rightarrow
M(B\otimes A); x\mapsto u \delta_B(x) u^*$ satisfies $(\delta_{B,u}\otimes\iota)
\delta_{B,u} = (\iota\otimes\delta)\delta_{B,u}$, and hence is a coaction.
\end{definition}

Finally, we recall the notion of morphism for the category of Hopf-C$^*$-algebras.

\begin{definition}[D\'efinition 0.5]
Let $(S,\delta)$ and $(S',\delta')$ be Hopf-C$^*$-algebras.  A \emph{morphism}
$(S,\delta) \rightarrow (S',\delta')$ is a non-degenerate $*$-homomorphism
$\phi:S\rightarrow M(S')$ with $(\phi\otimes\phi)\delta = \delta'\phi$.
\end{definition}

\section{Definitions}

Consult \cite{r30} for motivations on studying the Pentagonal equation.

\begin{definition}[D\'efinition 1.1]
A unitary $V\in\mc B(H\otimes H)$ is \emph{multiplicative} if it
satisfies the \emph{pentagonal equation}:
\[ V_{12} V_{13} V_{23} = V_{23} V_{12}. \]
\end{definition}

\begin{examples}[Exemples 1.2]\label{eg:1}
\begin{enumerate}
\item\begin{itemize}
\item The identity $1\in\mc B(H\otimes H)$ is a multiplicative unitary.
\item\label{eg:1.2} If $V$ is a multiplicative unitary and $U\in\mc B(H,H')$
is a unitary, then $W=(U\otimes U)V(U^*\otimes U^*)$ is a multiplicative
unitary on $H'$.  We say that $V$ and $W$ are \emph{equivalent}.
\item If $V$ is a multiplicative unitary and $\Sigma\in\mc B(H\otimes H)$
is the swap map, then $\Sigma V^* \Sigma$ is also a multiplicative unitary.
We say that $V$ and $W$ are \emph{opposite} if $V$ and $\Sigma W^* \Sigma$
are equivalent.
\item If $V$ and $W$ are two multiplicative unitaries on $H$ and $K$,
respectively, then $V_{13} W_{24} \in \mc B(H\otimes K\otimes H\otimes K)$
is a multiplicative unitary on $H\otimes K$.  We call this the
\emph{tensor product} of $V$ and $W$, sometimes denoted (abusively) by
$V\otimes W$.  Notice that $V\otimes W$ and $W\otimes V$ are equivalent.
\end{itemize}
\item If $G$ is a locally compact group with right Haar measure $dg$, then
$V_G(\xi)(s,t) = \xi(st,t)$ is a multiplicative unitary on $L^2(G,dg)$.
\item If $W$ is the fundamental unitary of a Kac algebra (see \cite{r6},
\cite{r13} and \cite{r17}) then $V=W^*$ is a multiplicative unitary.
\item\label{eg:1.4} If $(A,\delta)$ is a Hopf-C$^*$-algebra, and $\phi$
is a right Haar measure on $A$ (so $\phi\in A^*$ is a state with
$(\phi\otimes\mu)\delta(a) = \phi(a) \mu(1)$ for $a\in A,\mu\in A^*$),
then let $(H,\pi,\xi)$ be the cyclic GNS construction for $\phi$.
If we define $V_\phi$ by $V_\phi(\pi(x)\xi\otimes\eta)
= (\pi\otimes\pi)(\delta(x))(\xi\otimes\eta)$ for $\eta\in H$, then
$V_\phi$ is an isometry which satisfies the pentagonal equation.
If $V_\phi$ surjects, then it is a multiplicative unitary; this is
the case of a compact quantum group in the sense of Woronowicz, \cite{r54}.
\item Let $(A,\delta)$ be a Hopf-C$^*$-algebra.  The coproduct $\delta$ is
a coaction of $A$ on itself.  If also $(\pi,u)$ is a covariant representation
of $(A,\delta)$ on a Hilbert space $H$.  So $(\iota\otimes\delta)(u) = u_{12}
u_{13}$ and $(\pi\otimes\iota)\delta(a) = u(\pi(a)\otimes 1)u^*$ for $a\in A$.
Setting $V=(\iota\otimes\pi)(u)$, we see that $V$ is a multiplicative unitary.
\item Another interpretation of the pentagonal equation is the following:

If $A$ is a finite-dimensional Hopf algebra, and let $E$ be the algebra of
linear maps $A\rightarrow A$.  We identify $E$ with $A^*\otimes A$,
and let $v\in A^*\otimes A$ be the identity map.  Define a homomorphism
$L:A\rightarrow E$ by $L(a)(b) = ab$.  Recall that $A^*$ becomes an algebra
for the product
\[ \ip{xy}{a} = \ip{x\otimes y}{\delta(a)} \qquad
(x,y\in A^*, a\in A). \]
For $x\in A^*$ and $a\in A$, we let $\rho(x)(a) = (\iota\otimes x)\delta(a)$,
so $\rho$ is a homomorphism $A^* \rightarrow E$.

\begin{proposition}[Page~431]
\begin{enumerate}
\item\label{prop:1.1} For $a\in A, x\in A^*$, write
  $\delta(a) = \sum_i a_i \otimes b_i$;
  then $\rho(x) L(a) = \sum_i L(a_i) \rho(xb_i)$.
\item\label{prop:1.2} For $a\in A$, we have that
  $(\rho\otimes\iota)(v) (L(a)\otimes 1)
  = (L\otimes\iota)(\delta(a))(\rho\otimes\iota)(v)$ in $E\otimes A$.
\item\label{prop:1.3} We have that $(\iota\otimes\delta)(v)=v_{12} v_{13}$.
\item\label{prop:1.4} In $A^*\otimes E\otimes A$, we have that
\[ ((\iota\otimes L)(v))_{12} v_{13} ((\rho\otimes\iota)(v))_{23}
= ((\rho\otimes\iota)(v))_{23} ((\iota\otimes L)(v))_{12}. \]
\end{enumerate}
\end{proposition}
\begin{proof}
For (\ref{prop:1.1}), given $b\in A$, we have that $\rho(x) L(a) b
= \rho(x)(ab) = (\iota\otimes x)\delta(ab)
= \sum_i a_i (\iota\otimes xb_i)\delta(b)
= \sum_i L(a_i) \rho(xb_i) b$, as claimed.

For (\ref{prop:1.2}), given $x\in A^*$, we have that $(\iota\otimes x)(v) = x$,
and so $(\iota\otimes x)((\rho\otimes\iota)(v)) = \rho(x)$.  Thus, using part
(\ref{prop:1.1}),
\begin{align*}
(\iota\otimes x)((\rho\otimes\iota)(v)(L(a)\otimes 1)) &= \rho(x)L(a)
= \sum_i L(a_i) \rho(xb_i) \\
&= (\iota\otimes x) \sum_i (L(a_i)\otimes b_i)(\rho\otimes\iota)(v) \\
&= (\iota\otimes x)((L\otimes\iota)(\delta(a))(\rho\otimes\iota)(v)).
\end{align*}
As $x$ was arbitrary, this shows (\ref{prop:1.2}).

Now let $x,y\in A^*$ and $a\in A=A^{**}$.  Then $(a\otimes\iota)(v)=a$,
$(\iota\otimes x)(v)=x$ and $(\iota\otimes y)(v)=y$.  Thus
\[ \ip{a\otimes x\otimes y}{(\iota\otimes\delta)(v)} = 
\ip{x\otimes y}{\delta(a)} = \ip{xy}{a}. \]
However, also
\[ \ip{a\otimes x\otimes y}{v_{12} v_{13}}
= \ip{(\iota\otimes x)(v) (\iota\otimes y)(v)}{a}
= \ip{xy}{a}. \]
Thus we have shown (\ref{prop:1.3}).

Finally, by (\ref{prop:1.2}), we see that
\[ ((\rho\otimes\iota)(v))_{23} ((\iota\otimes L)(v))_{12}
= (\iota\otimes L\otimes\iota)(\iota\otimes\delta)(v)
((\rho\otimes\iota)(v))_{23}. \]
By (\ref{prop:1.3}), this is equal to
$((\iota\otimes L)(v))_{12} v_{13} ((\rho\otimes\iota)(v))_{23}$,
as required to show (\ref{prop:1.4}).
\end{proof}

\begin{corollary}[Page~431]
The operator $V=(\rho\otimes L(v)$ satisfies the pentagonal equation.
\end{corollary}

If $A$ is both unital and counital, then $L$ and $\rho$ inject, and we have
the following.

\begin{proposition}[Page~431]
Let $1\in A$ be the unit of $A$, and $\epsilon\in A^*$ be the unit of $A^*$.
\begin{enumerate}
\item\label{prop:2.1} If $V$, and so $v$, are invertible, then the map
$\kappa:A\rightarrow A; a\mapsto (a\otimes\iota)(v^{-1})$ is the antipode of
$A$.  That is, for $a\in A$,
\[ m(\iota\otimes\kappa)\delta(a) = m(\kappa\otimes\iota)\delta(a)
= \epsilon(a) 1. \]
Here $m:A\otimes A\rightarrow A$ is the multiplication map.
\item\label{prop:2.2} Conversely, if $A$ has an antipode, then $v$ is
invertible.
\end{enumerate}
\end{proposition}
\begin{proof}
For (\ref{prop:2.1}), as above, we have that $\delta(a)
= (a\otimes\iota\otimes\iota)(v_{12} v_{13})$ and
$(\iota\otimes\kappa)\delta(a)
= (a\otimes\iota\otimes\iota)(v_{12} v_{13}^{-1})$.  Thus
\[ m(\iota\otimes\kappa)\delta(a)
= (a\otimes m)(v_{12} v_{13}^{-1})
= (a\otimes\iota)(v v^{-1})
= (a\otimes\iota)(\epsilon\otimes 1) = \epsilon(a) 1. \]
Similarly, $m(\kappa\otimes\iota)\delta(a) = (a\otimes m)(v_{12}^{-1}v_{13})
= \epsilon(a)1$.  This shows (\ref{prop:2.1}).

For (\ref{prop:2.2}), compare with \cite{r1}.  Indeed, set
$u = (\iota\otimes\kappa)(v)$,
\begin{align*} vu &= (\iota\otimes m)(v_{12} u_{13})
= (\iota\otimes m(\iota\otimes\kappa))(v_{12} v_{13})
= (\iota\otimes m(\iota\otimes\kappa)\delta)(v) \\
&= (\iota\otimes\epsilon)(v)\otimes 1 = \epsilon\otimes 1. \end{align*}
So $u=v^{-1}$.
\end{proof}
\end{enumerate}
\end{examples}

We continue studying general multiplicative unitaries.
Let $H$ be a Hilbert space and $V\in\mc B(H\otimes H)$ a multiplicative
unitary.

\begin{definition}[D\'efinition~1.3]
Let $\omega\in\mc B(H)_*$, and define $L(\omega),\rho(\omega)\in\mc B(H)$
by $L(\omega) = (\omega\otimes\iota)(V)$ and
$\rho(\omega) = (\iota\otimes\omega)(V)$.  
Let
\[ A(V) = \{ L(\omega) : \omega\in\mc B(H)_* \} \qquad
\hat A(V) = \{ \rho(\omega) : \omega\in\mc B(H)_* \}. \]
Then $A(V)$ and $\hat A(V)$ form a dual pairing:
\[ \ip{L(\omega)}{\rho(\omega')} = (\omega\otimes\omega')(V)
= \ip{\rho(\omega')}{\omega} = \ip{L(\omega)}{\omega'}. \]
\end{definition}

\begin{proposition}[Proposition~1.4]\label{prop:3}
The spaces $A(V)$ and $\hat A(V)$ are subalgebras of $\mc B(H)$,
and the spaces $A(V)H$ and $\hat A(V)H$ are linearly dense in $H$.
\end{proposition}
\begin{proof}
Let $\omega,\omega'\in\mc B(H)_*$, and define $\psi\in\mc B(H)_*$ by
define $\ip{T}{\psi} = \ip{V^*(1\otimes T)V}{\omega\otimes\omega'}$ for
$T\in\mc B(H)$.  Then, using the pentagonal equation,
\begin{align*}
L(\omega) L(\omega') &= (\omega\otimes\iota)(V) (\omega'\otimes\iota)(V)
= (\omega\otimes\otimes'\otimes\iota)(V_{13}V_{23}) \\
&= (\omega\otimes\otimes'\otimes\iota)(V_{12}^* V_{23} V_{12})
= (\psi\otimes\iota)(V) = L(\psi).
\end{align*}
Similarly, $\rho(\omega) \rho(\omega') = \rho(\psi')$ where
$\ip{T}{\psi'} = (\omega\otimes\omega')(V(T\otimes 1)V^*)$.

Given non-zero $\xi,\eta\in H$, we have that $V^*(\xi\otimes\eta)\not=0$,
and so there are $\alpha,\beta\in H$ with
$\ip{\xi\otimes\eta}{V(\alpha\otimes\beta)} \not=0$.  Thus
$L(\omega_{\xi,\alpha})\beta$ is not orthogonal to $\eta$, and
$\rho(\omega_{\eta,\beta})\alpha$ is not orthogonal to $\xi$, showing
linear density of the spaces $A(V)H$ and $\hat A(V)H$.
\end{proof}

\begin{definition}[D\'efinition~1.5]
Let $V$ be a multiplicative unitary.  We write $S$ for the norm closure of
the algebra $A(V)$, and similarly denote by $\hat S$ the norm closure of
$\hat A(V)$.
\end{definition}

We remark that the functionals $\psi$ which appear in the proof above
are dense in $\mc B(H)_*$.  It follows that $\{ xy : x,y\in A(V) \}$ is dense
in $S$, and similarly $\{ xy : x,y\in\hat A(V) \}$ is dense in $\hat S$.

\begin{proposition}[Proposition~1.6]
Let $C^*(S)$ be the C$^*$-algebra (in $\mc B(H)$) generated by $S$,
and similarly for $C^*(\hat S)$.  Then $V$ is in the von Neumann algebra
generated by $C^*(\hat S) \otimes C^*(S)$.
\end{proposition}
\begin{proof}
Let $T\in\mc B(H\otimes H)$.  For $\omega\in\mc B(H)_*$,
\[ (\iota\otimes\omega\otimes\iota)(T_{13}V_{23} - V_{23}T_{13})
= T(1\otimes L(\omega)) - (1\otimes L(\omega))T. \]
So $T$ commutes with $1\otimes S$ if and only if $T_{13}$ commutes
with $V_{23}$.  A similar calculation shows that $\hat S\otimes 1$
commutes with $T$ if and only if $T_{13}$ commutes with $V_{12}$.

So if $T\in (\hat S\otimes S)'$ then $T_{13}$ commutes with both $V_{23}$
$V_{12}$.  As $V_{13} = V_{12}^* V_{23} V_{12} V_{23}^*$, it follows that
$T_{13}$ commutes with $V_{13}$.  So $V\in (\hat S\otimes S)''$ and
hence certainly $V \in (C^*(\hat S) \otimes C^*(S))''$.
\end{proof}

\begin{definition}[D\'efinition~1.7]
Let $V$ be a multiplicative unitary.  We say that $V$ is of
\emph{compact type} if $S$ is unital.  We say that $V$ is of
\emph{discrete type} if $\hat S$ is unital.
\end{definition}

\begin{definition}[D\'efinition~1.8]\label{defn:2}
Let $V$ be a multiplicative unitary.  A vector $e\in H$ is
\emph{fixed} if $V\theta_e = \theta_e$ (that is, $V(e\otimes\xi) =
e\otimes\xi$ for all $\xi\in H$), and is \emph{cofixed}
if $V\theta'_e = \theta'_e$ (that is, $V(\xi\otimes e) =
\xi\otimes e$ for all $\xi\in H$).
\end{definition}

\begin{proposition}[Proposition~1.9]
Let $e$ be a fixed (respectively cofixed) unit vector.  Then
$L(\omega_e)=1$ and $\rho(\omega_e)$ is the projection onto the subspace
of all fixed vectors (respectively, $\rho(\omega_e)=1$ and $L(\omega_e)$
is the projection onto the subspace of all cofixed vectors).
\end{proposition}
\begin{proof}
Clearly $L(\omega_e) = (\omega_e\otimes\iota)(V) = 1$.  Define
$\psi'\in\mc B(H)_*$ by $\psi'(T) = (e\otimes e|
V(T\otimes 1)V^*(e\otimes e)) = \ip{T}{\omega_e}$, as
$V^*(e\otimes e)=e\otimes e$.
By (the proof of) Proposition~\ref{prop:3}, $\rho(\omega_e)$ is
an idempotent, and as $\|\rho(\omega_e)\|\leq 1$, it follows that
$\rho(\omega_e)$ is a projection.  Now, $\rho(\omega_e)\xi=xi$
if and only if $(\xi|\rho(\omega_e)\xi)=\|\xi\|^2$, that is,
$(\xi\otimes e|V(\xi\otimes e) = \|\xi\otimes e\|^2$.  Thus
the image of $\rho(\omega_e)$ is $\{ \xi\in H : V(\xi\otimes e)=
\xi\otimes e \}$.  However, notice that if $V(\xi\otimes e)=\xi\otimes e$,
then for $\eta\in H$, the vector $\xi\otimes e\otimes\eta$ is fixed by
both $V_{12}$ and $V_{23}$, and hence by $V_{13} = V_{12}^* V_{23} V_{12}
V_{23}^*$, showing that $\xi$ is fixed.

The other case follows by working with $\Sigma V^* \Sigma$ instead of $V$.
\end{proof}

\begin{proposition}[Proposition~1.10]\label{prop:4}
Let $V$ be a multiplicative unitary on $H$, where $H$ is now separable.
Then $V$ is of compact type (respectively, discrete type) if and only if
the spaces of fixed vectors (respectively, cofixed vectors) is not zero.
\end{proposition}
\begin{proof}
If there is a fixed vector $e$ then $L(e)=1$ so $S$ is unital.  Conversely,
suppose that $S$ is unital, and recall from Proposition~\ref{prop:3} that
$S$ acts non-degenerately on $H$, so the unit of $S$ is the identity operator
on $H$.  Thus there is $\omega\in\mc B(H)_*$ with $\|L(\omega)-1\|<1/2$.
Fix a \emph{faithful} normal state $\psi$, using that $H$ is separable.
Then $|\ip{\rho(\psi)}{\omega}| = |\ip{L(\omega)}{\psi}| > 1/2$.
Set $\psi^1=\psi$, and defined inductively
$\ip{x}{\psi^{n+1}} = \ip{V(x\otimes 1)V^*}{\psi\otimes\psi^n}$.
Set $\psi_n = \frac1n \sum_{k=1}^n \psi^k$.
Thus, from Proposition~\ref{prop:3}, $\rho(\psi^n)=\rho(\psi)^n$.
Notice that $\|\rho(\psi)\|\leq 1$ and $(1-\rho(\psi))\rho(\psi_n) =
(\rho(\psi) - \rho(\psi)^{n+1})/n$, which converges to $0$ in norm.

If $T=1-\rho(\psi)$ is an injective operator, then $T^*$ has dense range,
and so there is $\omega'\in\mc B(H)_*$ with $\|\omega - \omega'T\|<1/4$.
As $|\ip{\rho(\psi_n)}{\omega}| = |\ip{L(\omega)}{\psi_n}|\geq 1/2$, because
$\psi_n$ is a state, we arrive at a contradiction.
So $T$ is not injective, and we can find a unit vector $e\in H$ with
$\rho(\psi)(e)=e$.  Then $1 = \ip{\rho(\psi)}{\omega_e}
= \ip{L(\omega_e)}{\psi}$.  As $\|L(\omega_e)\|\leq 1$, we have that
$1-L(\omega_e)$ is positive, and $\ip{1-L(\omega_e)}{\psi}=0$.
As $\psi$ is faithful, we must have that $L(\omega_e)=1$, as required to show
that $e$ is a fixed vector.
\end{proof}

We see that $1\in\mc B(H\otimes H)$ is both compact and discrete.
If $V$ is a multiplicative unitary, then $V$ is of compact (respectively,
discrete) type if and only if $\Sigma V^*\Sigma$ is of discrete (respectively,
compact) type.  The tensor product of two multiplicative unitaries of compact
(discrete) type is again of compact (discrete) type.

If $G$ is a compact group, and we form $V_G$ as in
Example~\ref{eg:1}.\ref{eg:1.2}, then the function which is constant $1$ is
fixed by $V_G$.  Similarly, if $G$ is a discrete group, then the function
which is $1$ at the identity, and $0$ elsewhere, is fixed by $V_G$.

In Example~\ref{eg:1}.\ref{eg:1.4}, the cyclic vector $\xi$ is fixed.

\begin{remarks}[Remarques~1.11]
\begin{enumerate}
\item Let $f\in H$ be a unit vector with $V(f\otimes f)=f\otimes f$.
Then $L(\omega_f)^2 = L(\omega_f)$ and $\rho(\omega_f)^2 = \rho(\omega_f)$;
as both $\|L(\omega_f)\| = \|\rho(\omega_f)\| = 1$, both $L(\omega_f)$ and
$\rho(\omega_f)$ are projections.
\end{enumerate}
\end{remarks}





\section{Commutative multiplicative unitaries}

We will now study commutative multiplicative unitaries, and show that
they correspond to locally compact groups.

Let $V$ be a multiplicative unitary on a Hilbert space $H$.

\begin{definition}[D\'efinition~2.1]
We say that $V$ is \emph{commutative} if $V_{13}$ and $V_{23}$ commute.
We say that $V$ is \emph{cocommutative} if $V_{12}$ and $V_{13}$ commute.
\end{definition}

The multiplicative unitary $V_G$ from Example~\ref{eg:1}.\ref{eg:1.2} is
commutative.  We will show that every commutative multiplicative unitary
is of this form.  Notice that $V$ is commutative (respectively, cocommutative)
if and only if $S$ (respectively, $\hat S$) is abelian.  Also, if $V$ is
commutative, then $V_{13}$ and $V_{23}^*$ commute, and so $C^*(S)$ is abelian.

\begin{theorem}[Th\'eor\`em~2.2]
Let $V$ be a commutative multiplicative unitary, and let $G$ be the spectrum
of the abelian C$^*$-algebra $C^*(S)$.  Then $G$ is a locally compact group
and there is a Hilbert space $J$ such that $V$ is equivalent to the
multiplicative unitary $V_G \otimes 1_{K\otimes K}$.
\end{theorem}





\section{Regular multiplicative unitaries}

In this section, we define and study \emph{regular} multiplicative unitaries
and deduce the existence of a densely defined antipode.

\begin{lemma}[Lemme~3.1]\label{lemma:1}
Let $H$ and $K$ be Hilbert spaces, and let $X\subseteq\mc B(H\otimes K)$.
The closures of the linear spans of
\[ \big\{ (1\otimes h)x(1\otimes k) : h,k\in\mc B_0(K), x\in X \big\} \]
and
\[ \big\{ (\iota\otimes\omega)(x)\otimes k : x\in X, k\in\mc B_0(K),
\omega\in\mc B(K)_* \big\}, \]
agree.
\end{lemma}
\begin{proof}
For $h=\theta_{\xi,\xi'}$ and $k=\theta_{\eta,\eta'}$, and $x\in X$, we have
\[ (1\otimes h)x(1\otimes k) = (\iota\otimes\omega_{\xi',\eta})(x)
\otimes \theta_{\xi,\eta'}, \]
from which the claim follows.
\end{proof}

Given a multiplicative unitary $V$, we set $\mc C(V) =
\{ (\iota\otimes\omega)(\Sigma V) : \omega\in\mc B(H)_* \}$.

\begin{proposition}[Proposition~3.2]\label{prop:5}
The space $\mc C(V)$ is a subalgebra of $\mc B(H)$.  The following conditions
are equivalent:
\begin{enumerate}
\item\label{prop:5.1} The closure of $\mc C(V)$ is $\mc B_0(H)$.
\item\label{prop:5.2} The closure of the linear span of
$\{ (x\otimes 1)V(1\otimes y) : x,y\in\mc B_0(H) \}$ is $\mc B_0(H\otimes H)$.
\end{enumerate}
\end{proposition}
\begin{proof}
For $\omega,\omega'\in\mc B(H)_*$, we have that
\[ (\iota\otimes\omega)(\Sigma V) (\iota\otimes\omega')(\Sigma V)
= (\iota\otimes\omega\otimes\omega')(\Sigma_{13} V_{13} \Sigma_{12} V_{12}). \]
Now, $\Sigma_{13} V_{13} \Sigma_{12} V_{12} =
\Sigma_{13} \Sigma_{12} V_{23} V_{12}
= \Sigma_{23} \Sigma_{13} V_{12} V_{13} V_{23}
= \Sigma_{23} V_{32} \Sigma_{13} V_{13} V_{23}
= V_{23} \Sigma_{23} \Sigma_{13} V_{13} V_{23}$.
Setting $\ip{x}{\psi} = (\omega'\otimes\omega)(V\Sigma(1\otimes x)V)$,
we see that $\psi\in\mc B(H)_*$, and that
\[ (\iota\otimes\omega)(\Sigma V) (\iota\otimes\omega')(\Sigma V)
= (\iota\otimes\psi)(\Sigma V). \]
Thus $\mc C(V)$ is a subalgebra.

Condition (\ref{prop:5.2}) is equivalent to the closure of the linear span
of
\[ \{ \Sigma(x\otimes 1)V(1\otimes y) : x,y\in\mc B_0(H) \}
= \{ (1\otimes x)\Sigma V(1\otimes y) : x,y\in\mc B_0(H) \} \]
being equal to $\mc B_0(H\otimes H)$.
The result follows by Lemma~\ref{lemma:1}.
\end{proof}

As $V$ is unitary, it is clear that the functionals $\psi$ constructed in
the proof are norm dense in $\mc B(H)_*$.  Thus $\{ xy: x,y\in\mc C(V) \}$
is dense in $\mc C(V)$.

\begin{definition}[D\'efinition~3.3]
A multiplicative unitary $V$ is \emph{regular} if the closure of
$\mc C(V)$ is $\mc B_0(H)$.
\end{definition}

Notice that $\mc C(\Sigma V^* \Sigma) = \mc C(V)^*$.  It follows that
$V$ is regular if and only if $\Sigma V^* \Sigma$ is regular.  Given two
equivalent multiplicative unitaries, one is regular if and only if the other
is regular.

\begin{examples}[Exemples~3.4]\label{eg:2}
\begin{enumerate}
\item For $\omega=\omega_{\xi,\eta}$, we have that
$(\iota\otimes\omega)(\Sigma) = \theta_{\eta,\xi}$.
Thus $1\in\mc B(H\otimes H)$ is a regular multiplicative unitary.
\item A direct calculation shows that for a locally compact group $G$,
the multiplicative unitary $V_G$ is regular.  Indeed, this is a special
case of the following.
\item Suppose there ia a unitary $J:H\rightarrow \overline{H}$ with
$J^* \overline{L(\omega)} J = L(\omega^*)$ for each $\omega\in\mc B(H)_*$.
Let $T$ be a Hilbert-Schmidt operator on $H$, so we can identify $T$ with
some vector $\tau\in\overline{H} \otimes H$.  Furthermore, suppose that $T$
is trace class, and let $\omega\in\mc B(H)_*$ be the associated functional.
Define $W$, a unitary on $\overline{H} \otimes H$, by $W = (1\otimes J^*)
\overline{V} (1\otimes J)$.  Notice that the composition of operators
$WT$ is Hilbert-Schmidt, and so can be identified as a member of
$\overline H\otimes H$, which is just $W(\tau)$.

For $\xi,\eta,\alpha,\beta\in H$
\begin{align*}
\big( \overline\beta \otimes \alpha \big| W(\overline\xi\otimes\eta) \big)
&= \big( \overline\beta \otimes J(\alpha) \big|
   \overline{V} (\overline\xi\otimes J(\eta)) \big)
= \big( V(\xi\otimes\overline{J(\eta)}) \big|
   \beta\otimes\overline{J(\alpha)} \big) \\
&= \big( L(\omega_{\beta,\xi})\overline{J(\eta)} \big|
   \overline{J(\alpha)} \big)
= \big( J(\alpha) \big| \overline{L(\omega_{\beta,\xi})} J(\eta) \big) \\
&= \big( \alpha \big| L(\omega_{\xi,\beta}) \eta \big)
= \big( \alpha\otimes\xi \big| \Sigma V(\beta\otimes\eta \big),
\end{align*}
and so
\[ \big( \overline\beta\otimes\alpha \big| WT \big)
= \big( \overline\beta\otimes\alpha \big| W(\tau) \big)
= \big( \alpha \big| (\iota\otimes\omega)(\Sigma V) \beta \big). \]
It follows that $V$ is regular.

In particular, if $V=W^*$ and $W$ is the fundamental unitary for a
Kac algebra in the sense of \cite{r6}, then \cite[Lemme~2.2.3]{r6},
together with the preceding argument, shows that $V$ regular.
\end{enumerate}
\end{examples}

\begin{proposition}[Proposition~3.4.4]\label{prop:12}
\begin{enumerate}
\item\label{prop:12.1} Let $V$ be a multiplicative unitary.  If $V$ is a 
multiplier of $\mc B_0(H) \otimes \mc B(H)$ (or $\mc B(H) \otimes \mc B_0(H)$)
then $\mc C(V) \subseteq \mc B_0(H)$.
\item\label{prop:12.2} Let $A$ be a Hopf-C$^*$-algebra which is unital, and
right simplifiable, and which has a right Haar state $\phi$ which satisfies
$\phi(x^*x)=0$ if and only if $\phi(xx^*)=0$.  Let $(H,\pi,\xi)$ be
the cyclic GNS construction.  Define $V_\phi\in\mc B(H\otimes H)$
by $V_\phi(\pi(x)\xi\otimes\eta) = (\pi\otimes\pi)\delta(x)(\xi\otimes\eta)$
for $\eta\in H$.  Then $V_\phi$ is a regular multiplicative unitary.
\end{enumerate}
\end{proposition}
\begin{proof}
For $x,y\in\mc B_0(H)$, we have that $(x\otimes 1)V \in
\mc B_0(H) \otimes \mc B(H)$ and so $(x\otimes 1)V(1\otimes y)
\in \mc B_0(H) \otimes \mc B_0(H) = \mc B_0(H\otimes H)$.
The result follows by the methods used in Lemma~\ref{lemma:1} and
Proposition~\ref{prop:5}.  The other option follows by working
with $\Sigma V^* \Sigma$.

As $\phi$ is right invariant, $V_\phi$ is isometric, compare
Example~\ref{eg:1}.\ref{eg:1.4}.
Clearly the image of $V_\phi$ contains the set
\[ \big\{ (\pi\otimes\pi)(\delta(x)(1\otimes y)) : x,y\in A \big\}, \]
and so, as $(A,\delta)$ is right simplifiable, we conclude that $V_\phi$
surjects.  So $V_\phi$ is a unitary, and a calculation shows that $V_\phi$
is multiplicative.

Then, for $a,b\in A$ and $\xi_0,\xi_1,\eta\in H$,
\[ V_\phi (\theta_{\pi(a)\xi,\xi_0}\otimes\pi(b))(\xi_1\otimes\eta)
= V_\phi (\pi(a)\xi\otimes\pi(b)\eta) (\xi_0|\xi_1)
= (\pi\otimes\pi)\delta(a)(\xi\otimes\pi(b)\eta) (\xi_0|\xi_1). \]
We can approximate $\delta(a)(1\otimes b)$ be a sum of tensors of the
form $x,y\in A$, so this is approximately
\[ (\pi(x)\otimes\pi(y))(\xi\otimes\eta) (\xi_0|\xi_1)
= \big(\theta_{\pi(x)\xi, \xi_0} \otimes \pi(y)\big)(\xi_1\otimes\eta). \]
Hence $V_\phi$ is a multiplier of $\mc B_0(H)\otimes\pi(A)$.  As $A$ is
unital, it follows that $V_\phi$ is a multiplier of
$\mc B_0(H)\otimes\mc B(H)$, and so the first part of the proposition shows
that $\mc C(V_\phi) \subseteq\mc B_0(H)$.

Then, for $\eta,\eta_1\in H$ and $a\in A$, we have that
\begin{align*}
\big( \eta_1 \big| (\iota\otimes\omega_{\xi,\eta})(\Sigma V_\phi) \pi(a)\xi \big)
&= \big( \xi\otimes\eta_1 \big| V_\phi(\pi(a)\xi\otimes\eta) \big)
= \big( \xi\otimes\eta_1 \big| (\pi\otimes\pi)\delta(a)(\xi\otimes\eta) \big) \\
&= \big( \eta_1 \big| \pi\big( (\phi\otimes\iota)\delta(a) \big) \eta \big)
= \phi(a) (\eta_1|\eta)
= \big( \eta_1 \big| \theta_{\eta,\xi} \pi(a) \xi \big).
\end{align*}
So $(\iota\otimes\omega_{\xi,\eta})(\Sigma V_\phi) = \theta_{\eta,\xi}$.

To show that $\mc C(V_\phi)$ is dense in $\mc B_0(H)$, it suffices to prove
that for each non-zero $\xi_1\in H$, there is $x\in\mc C(V_\phi)$ with
$(\xi|x(\xi_1))\not=0$.  Indeed, this would show that $\{ x^*(\xi) :
x\in\mc C(V_\phi)\}$ is dense in $H$.  Then, for $x\in\mc C(V_\phi)$ and
$\eta\in H$, we have that $\theta_{\eta,x^*\xi} = \theta_{\eta,\xi} x
\in\mc C(V_\phi)$, and thus $\mc C(V_\phi)$ is dense in $\mc B_0(H)$.

Now, for $b,c\in A$ and $\xi_1,\xi_2\in H$, we have that
\begin{align*}
\big( \xi_1 \big| L(\omega_{\pi(b)\xi,\pi(c)\xi})\xi_2 \big)
&= \big( \pi(b)\xi\otimes\xi_1 \big| V_\phi( \pi(c)\xi\otimes\xi_2 ) \big) \\
&= \big( \xi\otimes\xi_1 \big| (\pi\otimes\pi)
   ((b^*\otimes 1)\delta(c))(\xi\otimes\xi_2) \big)
= \big( \xi_1 \big| \pi(d) \xi_2 \big),
\end{align*}
where $d = (\phi\otimes\iota)((b^*\otimes 1)\delta(c))\in A$, as
$(b^*\otimes 1)\delta(c) \in A\otimes A$.
Hence $L(\omega_{\pi(b)\xi,\pi(c)\xi}) = \pi(d) \in \pi(A)$.
Now, $\pi(A)$ is closed in $\mc B(H)$, and so by continuity,
$L(\omega)\in\pi(A)$ for all $\omega\in\mc B(H)_*$.

For $\eta,\eta_1,\eta_2\in H$, we have that
\[ \big( \xi \big| (\iota\otimes\omega_{\eta_2,\eta_1})(\Sigma V_\phi)
   \eta \big)
= \big( \eta_2\otimes\xi \big| V_\phi(\eta\otimes\eta_1) \big)
= \big( \xi \big| L(\omega_{\eta_2,\eta}) \eta_1 \big). \]
Suppose that $(\xi|x(\eta))=0$ for all $x\in\mc C(V_\phi)$.  Thus
$(\xi|L(\omega_{\eta_2,\eta})\eta_1)=0$ for all $\eta_2,\eta_1\in H$,
that is, $L(\omega_{\eta_2,\eta})^*\xi=0$ for all $\eta_2\in H$.
However, $L(\omega_{\eta_2,\eta})=\pi(a)$ for some $a\in A$, and so
$a^*\xi=0 \implies \phi(aa^*)=0 \implies \phi(a^*a)=0 \implies a\xi=0$.
Thus $L(\omega_{\eta_2,\eta})\xi=0$ for all $\eta_2\in H$, which
shows that $V_{\phi}(\eta\otimes\xi)=0$, so $\eta=0$, as required.
\end{proof}

In particular, this result applies to compact quantum groups in the
sense of Woronowicz, \cite{r54}.  Furthermore, in this case, $S=\pi(A)$.

\begin{proposition}[Proposition~3.5]\label{prop:6}
If $V$ is a regular multiplicative unitary, the algebras $S$ and $\hat S$
are self-adjoint.
\end{proposition}
\begin{proof}
Let $E$ be the linear span of
\[ \big\{ (\omega\otimes\omega'\otimes\iota)(\Sigma_{12} V_{23}^*
V_{12} V_{13})^* : \omega,\omega'\in\mc B(H)_* \big\}. \]
As $\Sigma_{12} V_{23}^* V_{12} V_{13} = \Sigma_{12} V_{12} V_{23}^*$,
we see that $E$ is the linear span of
\[ \big\{ (\omega\otimes\omega'\otimes\iota)(V_{23}^*)^*
: \omega,\omega'\in\mc B(H)_* \big\}
= \big\{ (\omega'\otimes\iota)V : \omega'\in\mc B(H)_* \big\}, \]
and so the closure of $E$ is $S$.  Alternatively,
$\Sigma_{12} V_{23}^* V_{12} V_{13} = V_{13}^* \Sigma_{12} V_{12} V_{13}$,
and so
\[ (\omega\otimes\omega'\otimes\iota)(\Sigma_{12} V_{23}^* V_{12} V_{13})
= (\omega\otimes\iota)(V^*(y\otimes 1)V), \]
where $y=(\iota\otimes\omega')(\Sigma V)$.	From this, it follows that
the norm closure of $E$ is the norm closure of
\[ \big\{ (\omega\otimes\iota)(V^*(y\otimes 1)V) :
\omega\in\mc B(H)_*, y\in\mc B_0(H) \big\}, \]
which is clearly self-adjoint.  So $S$ is self-adjoint.  The $\hat S$
case follows, as $\hat S=S(\Sigma V^*\Sigma)^*$.
\end{proof}

\begin{proposition}[Proposition~3.6]\label{prop:7}
Let $V$ be a regular multiplicative unitary, with associated C$^*$-algebras
$S$ and $\hat S$.  We have that
\begin{enumerate}
\item\label{prop:7.1} $V\in M(\mc B_0(H)\otimes S)$ and
  $V\in M(\hat S\otimes\mc B_0(H))$;
\item\label{prop:7.2} The closed linear span of $\{ (x\otimes 1)V(1\otimes y) :
x\in\mc B_0(H), y\in S \}$ is $\mc B_0(H)\otimes S$, and the closed linear
span of $\{ (x\otimes 1)V(1\otimes y) : x\in\hat S, y\in\mc B_0(H) \}$
is $\hat S\otimes\mc B_0(H)$;
\item\label{prop:7.3} $V\in M(\hat S\otimes S)$;
\item\label{prop:7.4} The closed linear span of
$\{ (x\otimes 1)V(1\otimes y) : x\in\hat S, y\in S\}$ is $\hat S\otimes S$.
\end{enumerate}
\end{proposition}
\begin{proof}
For $x,y\in\mc B_0(H)$ and $\omega\in\mc B(H)_*$, we have that
$V(x\otimes L(y\omega)) = (\iota\otimes \omega\otimes\iota)
((V_{13} V_{23})(x\otimes y\otimes 1)) = (\iota\otimes \omega\otimes\iota)
((V_{12}^* V_{23} V_{12})(x\otimes y\otimes 1))$.  As $V(x\otimes y)\in
\mc B_0(H\otimes H)$, we see that $V(x\otimes L(y\omega))$ is in the closed
linear span of
\[ \{ (\iota\otimes\omega\otimes\iota)((V_{12}^* V_{23})(a\otimes b\otimes 1))
: a,b\in\mc B_0(H) \}. \]
Let $\omega = \omega'c$ for some $\omega'\in\mc B(H)_*$ and $c\in\mc B_0(H)$
(we may do this, by Lemma~\ref{lem:ap1}).  Then
\[ (\iota\otimes\omega\otimes\iota)((V_{12}^* V_{23})(a\otimes b\otimes 1))
= (\iota\otimes b\omega'\otimes 1)((1\otimes c\otimes 1) V_{12}^*
(a\otimes 1\otimes 1)V_{23}) \in \mc B_0(H) \otimes S, \]
using Proposition~\ref{prop:5}(\ref{prop:5.2}).

Also $(x\otimes L(\omega^*y^*)^*)V = (x\otimes (y\omega\otimes\iota)(V^*))V
= (\iota\otimes\omega\otimes\iota)( V^*_{23} (x\otimes y\otimes 1) V_{13} )$,
so using Proposition~\ref{prop:5}(\ref{prop:5.2}) is in the closed
linear span of
\[ \{ (\iota\otimes\omega\otimes\iota)( V^*_{23} (a\otimes 1\otimes 1)
V_{12} (1\otimes b\otimes 1) V_{13} ) : a,b\in\mc B_0(H) \}. \]
Notice that $(\iota\otimes\omega\otimes\iota)( V^*_{23} (a\otimes 1\otimes 1)
V_{12} (1\otimes b\otimes 1) V_{13} ) =
(\iota\otimes\omega\otimes\iota)( (a\otimes 1\otimes 1) V^*_{23}
V_{12} V_{13} (1\otimes b\otimes 1)  ) =
(\iota\otimes b\omega\otimes\iota)( (a\otimes 1\otimes 1) V_{12} V^*_{23} )$.
Writing $b\omega = \omega' c$, with $c\in\mc B_0(H)$, as $(a\otimes c)V\in
\mc B_0(H\otimes H)$, we have that
\[ (\iota\otimes b\omega\otimes\iota)( (a\otimes 1\otimes 1) V_{12} V^*_{23} )
= (\iota\otimes\omega'\otimes\iota)( (a\otimes c\otimes 1) V_{12} V^*_{23} )
\in \mc B_0(H) \otimes S, \]
where here we use Proposition~\ref{prop:6}.  This shows the first part of
(\ref{prop:7.1}); the second part follows by working with $\Sigma V^*\Sigma$.

Let $a,b\in\mc B_0(H), \omega\in\mc B(H)_*$ and set $y=L(\omega a)$.  Then
\[ (b\otimes 1)V(1\otimes y) = (\iota\otimes\omega\otimes\iota)\big(
(b\otimes a\otimes 1)V_{13} V_{23} \big)
= (\iota\otimes\omega\otimes\iota)\big( ((b\otimes a)V^* \otimes 1)
V_{23} V_{12} \big). \]
Again, as $V$ is unitary, the closed linear span of $\{ (b\otimes a)V^*:
a,b\in\mc B_0(H) \}$ is $\mc B_0(H)\otimes\mc B_0(H)$.  To show
(\ref{prop:7.2}) it hence suffices to show that
\begin{align*}
\big\{ & (\iota\otimes\omega\otimes\iota)\big( (a\otimes 1\otimes 1)
V_{23} V_{12} \big) : a\in\mc B_0(H), \omega\in\mc B(H)_* \} \\
&= \big\{ (\iota\otimes b\omega\otimes\iota)\big( (a\otimes 1\otimes 1)
V_{23} V_{12} \big) : a,b\in\mc B_0(H), \omega\in\mc B(H)_* \}
\end{align*}
is linearly dense in $B_0(H)\otimes S$.  However,
\[ (\iota\otimes b\omega\otimes\iota)\big( (a\otimes 1\otimes 1) V_{23} V_{12}
= (\iota\otimes\omega\otimes\iota)\big( V_{23} \big( (a\otimes 1)V(1\otimes b)
\otimes 1 \big) \big), \]
and so the result follows by Proposition~\ref{prop:5}.  Similarly, the
second claim of (\ref{prop:7.2}) follows by working with $\Sigma V^*\Sigma$.

For (\ref{prop:7.3}), notice that by (\ref{prop:7.1}), both $V_{12}$ and
$V_{23}$ are multipliers of $\hat S \otimes \mc B_0(H) \otimes S$, and hence so
is $V_{13} = V_{12}^* V_{23} V_{12} V_{23}^*$.  Thus $V\in M(\hat S\otimes S)$,
as claimed.

For (\ref{prop:7.4}), it suffices to show that the closed linear span of
\[ \big\{ (x\otimes a\otimes 1)V_{13}(1\otimes b\otimes y) :
a,b\in\mc B_0(H), x\in\hat S, y\in S \big\} \]
is $\hat S\otimes\mc B_0(H)\otimes S$.  As $V_{13} = V_{12}^* V_{23} V_{12}
V_{23}^*$, as $V^*\in M(\hat S\otimes\mc B_0(H)$ and $V\in M(\mc B_0(H)
\otimes S)$, and as $V$ is unitary, we equivalently can show that the closed
linear span of
\[ \big\{ (x\otimes a\otimes 1)V_{23}V_{12}(1\otimes b\otimes y) :
a,b\in\mc B_0(H), x\in\hat S, y\in S \big\} \]
is $\hat S\otimes\mc B_0(H)\otimes S$.  Notice that
\[ (x\otimes a\otimes 1)V_{23}V_{12}(1\otimes b\otimes y)
= \big( x \otimes (a\otimes 1)V(1\otimes y) \big)
\big( V(1\otimes b) \otimes 1 \big), \]
and so by (\ref{prop:7.2}), we get the closed linear span of
\begin{align*}
\big\{ & \big( x \otimes c\otimes z \big) V_{12}(1\otimes b\otimes 1)
: b,c\in\mc B_0(H), z\in S, x\in\hat S \big\} \\
&= \big\{ (1\otimes c\otimes z) \big( (x\otimes 1)V(1\otimes b) \otimes 1 \big)
: b,c\in\mc B_0(H), z\in S, x\in\hat S \big\},
\end{align*}
which again by (\ref{prop:7.2}) is the closed linear span of
\[ \big\{ (1\otimes c\otimes z) ( x \otimes b \otimes 1)
: b,c\in\mc B_0(H), z\in S, x\in\hat S \big\}, \]
which is of course $\hat S \otimes \mc B_0(H) \otimes S$, as required.
\end{proof}

\begin{corollary}[Corollaire~3.7]\label{corr:1}
Let $V$ be a regular multiplicative unitary, and let $S,\hat S$ be the
associated C$^*$-algebras.  Then:
\begin{enumerate}
\item\label{corr:1.1} The closed linear spans of
$\{ V(x\otimes 1)V^*(1\otimes y) : x,y\in S \}$
and $\{ V(x\otimes 1)V^*(y\otimes 1) : x,y\in S \}$ are both equal to
$S\otimes S$;
\item\label{corr:1.2} The closed linear spans of
$\{ V^*(1\otimes x)V(1\otimes y) : x,y\in\hat S \}$
and $\{ V^*(1\otimes x)V(y\otimes 1) : x,y\in\hat S \}$ are both equal to
$\hat S\otimes\hat S$;
\end{enumerate}
\end{corollary}
\begin{proof}
For $a\in\mc B_0(H), \omega\in\mc B(H)_*$ and $y\in S$,
\begin{align*}
V(L(a\omega)\otimes 1)V^*(1\otimes y)
&= (\omega\otimes\iota\otimes\iota) \big( V_{23} V_{12}(a\otimes 1\otimes 1)
   V^*_{23}(1\otimes 1\otimes y) \big) \\
&= (\omega\otimes\iota\otimes\iota) \big(V_{12}V_{13}
   (a\otimes 1\otimes y)\big). \end{align*}
By Proposition~\ref{prop:7}(\ref{prop:7.1}) we see that
\begin{align*}
\overline{\lin}\{ V(x\otimes 1)V^*(1\otimes y) : x,y\in S \}
&= \overline{\lin}\{ (\omega\otimes\iota\otimes\iota)
(V_{12}(a\otimes 1\otimes y) : a\in\mc B_0(H), y\in S \} \\
&= \overline{\lin}\{ (\omega\otimes\iota)(V(a\otimes 1)) : a\in\mc B_0(H)
\} \otimes S = S\otimes S. \end{align*}
Now consider
\begin{align*} V(L(\omega a)\otimes 1)V^*(y\otimes 1)
&= (\omega\otimes\iota\otimes\iota)\big( V_{23}(a\otimes 1\otimes 1)
V_{12} V_{23}^* (1\otimes y\otimes 1) \big) \\
&= (\omega\otimes\iota\otimes\iota)\big( (a\otimes 1\otimes 1)
V_{12} V_{13} (1\otimes y\otimes 1) \big). \end{align*}
Thus, now using Proposition~\ref{prop:7}(\ref{prop:7.2}),
\begin{align*}
\overline{\lin}\{ & V(x\otimes 1)V^*(y\otimes 1) : x,y\in S \} \\
&= \overline{\lin}\{ (\omega\otimes\iota\otimes\iota)\big(
( (a\otimes 1)V(1\otimes y) \otimes 1 ) V_{13} \big) : a\in\mc B_0(H),
y\in S \} \\
&= \overline{\lin}\{ (\omega\otimes\iota\otimes\iota)\big(
( a\otimes y\otimes 1 ) V_{13} \big) : a\in\mc B_0(H), y\in S \}
= S\otimes S. \end{align*}
This shows (\ref{corr:1.1}), and then (\ref{corr:1.2}) follows
by working with $\Sigma V^*\Sigma$.
\end{proof}

\begin{theorem}[Th\'eor\`eme 3.8]\label{thm:1}
Let $V$ be a regular multiplicative unitary, and let $S,\hat S$ be the
associated C$^*$-algebras.  We may define a coproduct $\delta$ on $S$
by $\delta(x) = V(x\otimes 1)V^*$, and then $(S,\delta)$ becomes a
bisimplifiable Hopf-C$^*$-algebra.  We may define a coproduct $\hat\delta$
on $\hat S$ by $\hat\delta(x) = V^*(1\otimes x)V$, and then
$(\hat S,\hat\delta)$ becomes a bisimplifiable Hopf-C$^*$-algebra.
\end{theorem}
\begin{proof}
By Corollary~\ref{corr:1}(\ref{corr:1.1}) it follows that
$\delta$ is indeed a $*$-homomorphism $S\rightarrow M(S\otimes S)$
such that $\delta(S)(1\otimes S)$ and $\delta(S)(S\otimes 1)$
are (dense) subsets of $S\otimes S$; this also shows that $(S,\delta)$
is bisimplifiable.  That $\delta$ is coassociative follows as
\begin{align*} (\iota\otimes\delta)\delta(x)
= V_{23} V_{12} (x\otimes 1\otimes 1) V_{12}^* V_{23}^*
= V_{12} V_{13} V_{23}(x\otimes 1\otimes 1) V_{23}^* V_{13}^* V_{12}^*
= (\delta\otimes\iota)\delta(x), \end{align*}
as required.  Let $(u_i)$ be a bounded approximate identity for $S$,
and let $x,y\in S$, so with $\tau = \delta(x)(1\otimes y)\in S\otimes S$,
\[ \delta(u_i)\tau = \delta(u_ix)(1\otimes y)
\rightarrow \delta(x)(1\otimes y) = \tau. \]
By Corollary~\ref{corr:1}(\ref{corr:1.1}), such $\tau$ are dense,
and so $\delta$ is non-degenerate.  The results for $\hat S$ follow
from working with $\Sigma V^* \Sigma$.
\end{proof}

\begin{proposition}[Proposition 3.9]\label{prop:8}
The map $\kappa:A(V)\rightarrow S; (\omega\otimes\iota)(V)
\mapsto (\omega\otimes\iota)(V^*)$ is a well-defined algebra antihomomorphism,
called the \emph{antipode}.
\end{proposition}
\begin{proof}
We have that $(\omega\otimes\iota)(V^*) = L(\omega^*)^* \in S$
by Proposition~\ref{prop:6}.  If $L(\omega)=0$ then
\[ 0 = \ip{L(\omega)}{\omega'} = \ip{\rho(\omega')}{\omega}
= \ip{x}{\omega} \qquad (\omega'\in\mc B(H)_*,x\in\hat S), \]
the last equality following by density.  As $\hat S$ is self-adjoint,
also $\ip{x}{\omega^*} = \overline{\ip{x^*}{\omega}} = 0$ for all
$x\in \hat S$, and so $\ip{L(\omega^*)}{\omega'}
= \ip{\rho(\omega')}{\omega^*} = 0$ for all $\omega'\in\mc B(H)_*$.
Thus $L(\omega^*)=0$, and so $\kappa$ is well-defined.

As in the proof of Proposition~\ref{prop:3}, given $\omega,\omega'\in
\mc B(H)_*$, if $\psi\in\mc B(H)_*$ is defined by $\ip{T}{\psi}
= \ip{V^*(1\otimes T)V}{\omega\otimes\omega'}$ then $L(\omega) L(\omega')
= L(\psi)$.  Then $\ip{T}{\psi^*} = \overline{\ip{V^*(1\otimes T^*)V}
{\omega\otimes\omega'}} = \ip{V^*(1\otimes T)V}{\omega^*\otimes(\omega')^*}$
and so $L(\psi^*) = L(\omega^*)L((\omega')^*)$.
Thus $\kappa(L(\omega) L(\omega')) = L(\psi^*)^* = L((\omega')^*)^*
L(\omega^*)^* = \kappa(L(\omega')) \kappa(L(\omega))$ and so $\kappa$
is an antihomomorphism as required.
\end{proof}

\begin{definition}[D\'efinition 3.10]
A multiplicative unitary $V$ is \emph{biregular} if it is regular, and if
$\{ (\omega\otimes\iota)(\Sigma V) : \omega\in\mc B(H)_* \}$ is dense in
$\mc B_0(H)$.
\end{definition}

\begin{remark}[Remarques 3.11(a)]\label{defn:1}
Let $W$ be the fundamental unitary associated to a Kac-von Neumann algebra,
see \cite{r6}.  Set $V=W^*$ and let $\hat\Delta$ be the modular operator
associated with the dual Haar weight $\hat\phi$ on the dual Kac algebra
$\hat M$.  Following \cite[2.1.5(a)]{r6} it follows that $\hat A(V)$ generates
$\hat M$ as a von Neumann algebra; the same is true of $S$.  Then
\cite[corollaire~3.1.10]{r6} shows that the restriction of $\hat\phi$ to
$S^+$, say $\psi$, defines a normal semi-finite weight on $S$.  By
\cite[Lemme~I.1]{r7}, we have that $V^*(1\otimes\hat\Delta)V = \hat\Delta
\otimes\hat\Delta$.  Thus, for $\omega\in M_*$ and all $t\in\mathbb R$,
we have that $L(\hat\Delta^{it}\omega) = \hat\Delta^{it} L(\omega)
\hat\Delta^{-it}$ and so the modular automorphism group $(\sigma_t)$ of
$\hat M$ restricts to $S$ to give a norm-continuous group of automorphisms.
It is now easy to verify that $S$ together with $\kappa$ and $\psi$ gives
a Kac C$^*$-algebra in the sense of \cite{r50}.
\end{remark}

\begin{remark}[Remarques 3.11(b)]
Let $V$ be a regular multiplicative unitary.  For $\omega\in\mc B(H)_*$,
as in the proof above, we see that $L(\omega)=0$ if and only if $\omega$
induces the zero functional on $S$.  As $S$ is a non-degenerate C$^*$-algebra
of $H$ (by Proposition~\ref{prop:3}) we see that if $\omega\geq0$, then
$\omega$ is zero on $S$ only if $\omega=0$.  (This follows, as let
$(e_\alpha)$ be a bounded approximate identity in $S$.  Non-degeneracy
implies that $e_\alpha\rightarrow 1$ strongly, and so $\|\omega\|
= \ip{1}{\omega} = \lim_\alpha \ip{e_\alpha}{\omega}$.)
Similarly, if $\omega\geq0$ and $\rho(\omega)=0$, then $\omega=0$.

For $x\in S$ and $\omega,\omega'\in S^*$, define
\[ x*\omega = (\omega\otimes\iota)\delta(x), \quad
\omega*x = (\iota\otimes\omega)\delta(x), \quad
\omega*\omega' = (\omega\otimes\omega')\circ\delta. \]
By Lemma~\ref{lem:ap1}, we may suppose that $\omega=\omega_0a_0$ for some
$\omega_0\in S^*,a_0\in S$.  Then $x*\omega
= (\omega\otimes\iota)((a_0\otimes 1)\delta(x)) \in S$, as
$(a_0\otimes 1)\delta(x) \in S\otimes S$.  Similarly $\omega*x\in S$.

Suppose now $x\geq0$ and $\omega\geq0$ and that $\omega*x=0$.  If
$\omega\not=0$, write $x=y^*y$ for some $y\in S$, and let $(\pi,H,\xi)$ be the
cyclic GNS construction for $\omega$.  Then
\[ 0 = (\iota\otimes\omega)\delta(x)
= (\iota\otimes\omega_\xi)(\iota\otimes\pi)(V(y^*y\otimes 1)V^*), \]
and so $(y\otimes 1)(\iota\otimes\pi)(V^*)(\cdot\otimes\xi)=0$.
In particular, for $a\in\mc B_0(H), b\in S$, also
\[ 0 = (y\otimes\pi(b))(\iota\otimes\pi)(V^*)(a(\cdot)\otimes\xi)
= (y\otimes 1)(\iota\otimes\pi)\big( (1\otimes b)V^*(a\otimes 1) \big)
(\cdot\otimes\xi). \]
By Proposition~\ref{prop:7}(\ref{prop:7.2}), this shows that
\[ 0 = (y\otimes 1) (c\otimes\pi(d)) (\cdot\otimes\xi)
\qquad (c\in\mc B_0(H), d\in S). \]
It follows that $y=0$, so $x=0$.  In conclusion,
$x\geq0,\omega\geq0,\omega*x=0 \implies x=0$ or $\omega=0$.
\end{remark}

\begin{remark}[Remarques 3.11(c)]
We say that $(A,\delta)$ is \emph{right reduced} (respectively, \emph{left
reduced}) if for non-zero $\omega\in A^*_+, x\in A_+$ also $\omega*x$
(respectively, $x*\omega$) is non-zero.  We have just shown that $(S,\delta)$
arising from a regular multiplicative unitary is right reduced;
similarly $\hat S$ will be left reduced.
\end{remark}

\begin{proposition}[Proposition 3.11.1]\label{prop:9}
Let $(A,\delta)$ be right (respectively left) reduced.  Then:
\begin{enumerate}
\item\label{prop:9.1} For non-zero $\omega,\omega'\in A^*_+$ with
$\omega$ faithful, and for non-zero $x\in A_+$, we have that
$\omega*\omega'$ (respectively $\omega'*\omega$) is faithful, and
$x*\omega$ (respectively $\omega*x$) is strictly positive
(meaning that $\ip{\mu}{x*\omega}>0$ for all states $\mu$,
or that the right ideal generated by $x*\omega$ is all of $A$).
\item\label{prop:9.2} If $A$ is unital and separable, then it admits
a right (respectively, left) faithful Haar state.
\end{enumerate}
\end{proposition}
\begin{proof}
We prove the assertions in the right reduced case; the left reduced case
follows by replacing $\delta$ with $\sigma\delta$ where $\sigma:A\otimes A
\rightarrow A\otimes A$ is the swap map.  For non-zero $y\in A_+$,
\[ \ip{\omega*\omega'}{y} = \ip{\omega}{\omega'*y} \not= 0, \]
as $\omega'*y\not=0$ and $\omega$ is faithful.  Similarly, for a state $\mu$,
\[ \ip{\mu}{x*\omega} = \ip{\omega*\mu}{x} \not=0, \]
by using the previous calculation.  To show (\ref{prop:9.2}), we
use the following lemma.
\end{proof}

\begin{lemma}[Lemme 3.11.2]\label{lem:1}
With $(A,\delta)$ being unital and right reduced, let $\omega$ be a faithful
state.  Then:
\begin{enumerate}
\item\label{lem:1.1} If $x\in A$ with $x*\omega=x$, then $x\in\mathbb C1$;
\item\label{lem:1.2} There is a state $\phi$ with $\omega*\phi=\phi*\omega
=\phi$ (compare \cite{r54}).
\item\label{lem:1.3} Such $\phi$ is also a faithful right Haar state.
\end{enumerate}
\end{lemma}
\begin{proof}
As $(x*\omega)^* = ((\omega\otimes\iota)\delta(x))^* =
(\omega\otimes\iota)\delta(x^*) = x^* * \omega$, for (\ref{lem:1.1})
we may suppose that $x=x^*$.  Notice that $1*\omega =
(\omega\otimes\iota)\delta(1) = 1$.  So for $\lambda\in\mathbb R$,
if $x-\lambda\geq0$ is positive and non-zero, then by
Proposition~\ref{prop:9}(\ref{prop:9.1}) we have that
$(x-\lambda)*\omega = x-\lambda$ is strictly positive.  Taking $\lambda$ to
be the minimum of the spectrum of $x$ shows that $x\in\mathbb R1$ as claimed.

For (\ref{lem:1.2}) let $\phi$ be a weak$^*$-limit of the Cesaro means of
$\omega^n = \omega*\omega*\cdots*\omega$ (n times).  Then $\phi$ is a state,
and clearly $\phi*\omega = \omega*\phi = \phi$.

For (\ref{lem:1.3}), for $x\in A$ we have that $(x*\phi)*\omega = x*(\phi
*\omega) = x*\phi$ and so by (\ref{lem:1.1}) $x*\phi$ is a scalar.  But then
$x*\phi = (x*\phi)*\omega = (\omega\otimes\iota)\delta(x*\phi)
= \ip{\omega}{x*\phi}1 = \ip{\phi}{x}1$ so $\phi$ is a right Haar state.
As $\phi = \omega*\phi$, by Proposition~\ref{prop:9}(\ref{prop:9.1}),
$\phi$ is faithful.
\end{proof}




\section{Multiplicative unitaries of compact type, and Woronowicz
C$^*$-algebras}

In this section, we depart from the original paper, and study the relationship
between Compact Quantum Groups (in the sense of \cite{woro}, a paper not
published at the time) and multiplicative unitaries of compact type.
Compact Quantum Groups have subsumed the theory of Matrix Pseudogroups as
a special case, and an added advantage is that the resulting proofs are
easier in some cases.

Firstly, let $(A,\delta)$ be a compact quantum group.  That is, $A$ is unital
and $(A,\delta)$ is bisimplifiable.  Then \cite{woro} shows that $(A,\delta)$
admits a unique Haar state $\phi$.  By Example~\ref{eg:1}(\ref{eg:1.4}) we
construct a multiplicative unitary $V$ on the GNS space for $\phi$.
By Proposition~\ref{prop:12}(\ref{prop:12.2}) $V$ is regular (we note that
the condition here, that $\phi(x^*x)=0$ if and only if $\phi(xx^*)=0$ is
quite involved to prove-- see \cite[???]{woro}).  The C$^*$-algebra $S$
is simply $\pi(A)$, and the coproduct on $S$ is the natural quotient of
$\delta$.  As $S$ is thus unital, $V$ is of compact type.

[Do we want to give a self-contained (sketch/account) of all of this?  It
might be rather involved...]

We now start with a multiplicative unitary $V$ on $H$ of compact type which
admits a non-zero fixed vector $E\in H$ (see Definition~\ref{defn:2}).
If $H$ is separable, then by Proposition~\ref{prop:4} such a fixed vector 
automatically exists.  Let $\phi = \omega_e\in\mc B(H)_*$.

\begin{definition}[D\'efinition 4.3]\label{defn:3}
For $\xi\in H$ define $\lambda_\xi\in\mc B(H)$ by $\lambda_\xi
= (\theta'_e)^* V^* \theta_\xi$.  That is, for $\eta,\eta'\in H$,
$(\lambda_\xi(\eta)|\eta') = (V(\xi\otimes\eta)|\eta'\otimes e)$.
\end{definition}

\begin{proposition}[Proposition 4.4]\label{prop:13}
For $\xi\in H$, we have that $(\lambda_\xi\otimes 1)V =
V(\lambda_\xi\otimes 1)$.
\end{proposition}
\begin{proof}
We have that $\lambda_\xi^*\otimes 1 = \theta^*_{1,\xi} V_{12} \theta_{2,e}$.
Thus
\begin{align*} V(\lambda_\xi^*\otimes 1) &=
V \theta^*_{1,\xi} V_{12} \theta_{2,e}
= \theta^*_{1,\xi} V_{23} V_{12} \theta_{2,e}
= \theta^*_{1,\xi} V_{12} V_{13} V_{23} \theta_{2,e} \\
&= \theta^*_{1,\xi} V_{12} V_{13} \theta_{2,e}
\quad\text{as $e$ is fixed, so $V\theta_e = \theta_e$} \\
&= \theta^*_{1,\xi} V_{12} \theta_{2,e} V
= (\lambda_\xi^*\otimes 1)V.
\end{align*}
\end{proof}







\section{Constructions with Woronowicz C$^*$-algebras}\label{sec:5}





\section{Irreducible multiplicative unitaries}

\begin{proposition}[Proposition 6.1]\label{prop:10}
Let $V$ be a multiplicative unitary on $H$ and let $U\in\mc B(H)$ be a unitary
with $U^2=1$ such that $\hat V=\Sigma(U\otimes 1)V(U\otimes 1)\Sigma$
and $\tilde V=(U\otimes U)\hat V(U\otimes U)$ are both multiplicative.
Then the following formulae hold:
\begin{enumerate}
\item\label{prop:10.1} $V_{12} (1\otimes U\otimes 1) V_{23}
(1\otimes U\otimes 1)
= (1\otimes U\otimes 1)V_{23}(1\otimes U\otimes 1)V_{13} V_{12}$;
\item\label{prop:10.2} $\hat V_{23} V_{12} V_{13} = V_{13} \hat V_{23}$;
\item\label{prop:10.3} $\tilde V_{12} V_{13} = V_{13} V_{23} \tilde V_{12}$;
\item\label{prop:10.4} the unitaries $\Sigma_{23} \hat V_{23} V_{23}$ and
$V_{12}$ commute;
\item\label{prop:10.5} the unitaries $V_{12} \tilde V_{12} \Sigma_{12}$ and
$V_{23}$ commute.
\end{enumerate}
\end{proposition}
\begin{proof}
We have that
\begin{gather*}
\Sigma_{13} \hat V_{12} \Sigma_{13}
   = (1\otimes U\otimes 1) V_{23} (1\otimes U\otimes 1), \\
\Sigma_{13} \hat V_{23} \Sigma_{13}
   = (U\otimes 1\otimes 1)V_{12}(U\otimes 1\otimes 1),
\end{gather*}
That $\hat V$ is multiplicative means that
\begin{align*} \hat V_{12} \hat V_{13} \hat V_{23}
&= \hat V_{12} \Sigma_{13} (U\otimes 1\otimes 1) V_{13}
(U\otimes 1\otimes 1) \Sigma_{13} \hat V_{23} \\
&= \Sigma_{13}(1\otimes U\otimes 1) V_{23} (U\otimes U\otimes 1) V_{13}
V_{12}(U\otimes 1\otimes 1)\Sigma_{13}
= \hat V_{23} \hat V_{12},
\end{align*}
that is
\[ (U\otimes U\otimes 1) V_{23} (1\otimes U\otimes 1) V_{13}
V_{12}(U\otimes 1\otimes 1)
= (U\otimes 1\otimes 1)V_{12}(1\otimes U\otimes 1)
V_{23} (U\otimes U\otimes 1). \]
Then (\ref{prop:10.1}) follows.

Applying $\Sigma_{23}$ to the left and right of (\ref{prop:10.1}) gives
(\ref{prop:10.2}).  Using $\Sigma_{12}$ instead gives (\ref{prop:10.3}),
once we notice that $\tilde V = \Sigma(1\otimes U)V(1\otimes U)\Sigma$.

As $V$ is multiplicative, (\ref{prop:10.2}) gives that
$\hat V_{23} V_{23} V_{12} = V_{13}\hat V_{23}\hat V_{23}$ and applying
$\Sigma_{23}$ on the left gives (\ref{prop:10.4}).
A similar argument applied to (\ref{prop:10.3}) gives (\ref{prop:10.5}).
\end{proof}

\begin{definition}[D\'efinition 6.2]
A multiplicative unitary $V$ is \emph{irreducible} is there is a unitary
$U\in\mc B(H)$ with:
\begin{enumerate}
\item $U^2=1$ and $(\Sigma(1\otimes U)V)^3 = 1$;
\item the unitaries $\hat V=\Sigma(U\otimes 1)V(U\otimes 1)\Sigma$
and $\tilde V=(U\otimes U)\hat V(U\otimes U)$ are both multiplicative.
\end{enumerate}
\end{definition}

Notice that clearly $\tilde V$ is multiplicative if and only if $\hat V$ is
multiplicative.  That $(\Sigma(1\otimes U)V)^3 = 1$ is equivalent to
$\hat V V \tilde V = (U\otimes 1)\Sigma$.  Finally, observe that $U$ being
unitary with $U^2=1$ is equivalent to $U$ being self-adjoint and unitary.

\begin{proposition}[Proposition 6.3]
Let $V$ be a multiplicative unitary which is regular and irreducible.
Then $\{xy : x\in S, y\in \hat S \}$ is linearly dense in $\mc B_0(H)$.
\end{proposition}
\begin{proof}
Notice that $\Sigma\tilde V^* = (1\otimes U^*)V^*(1\otimes U^*)\Sigma
= (1\otimes U^*)\Sigma(\Sigma V^*\Sigma)(U^*\otimes 1)$ and so
\[ \mc C(\Sigma\tilde V^*) = \{ (\iota\otimes\omega)
((1\otimes U^*)\Sigma(\Sigma V^*\Sigma)(U^*\otimes 1)) : \omega\in\mc B(H)_* \}
= \mc C(\Sigma V^*\Sigma) U^* = \mc C(V)^*U^*, \]
which equals $\mc B_0(H)$ as $V$ is regular.  Hence also
$\{ (\iota\otimes\omega) ((U\otimes 1)\Sigma\tilde V^*)
: \omega\in\mc B(H)_* \}$ is dense in $\mc B_0(H)$.  As $V$ is irreducible,
$(U\otimes 1)\Sigma\tilde V^* = \hat VV$, and so
$\{ (\iota\otimes\omega) (\hat VV) : \omega\in\mc B(H)_* \}$
is dense in $\mc B_0(H)$.  As $\hat S$ acts irreducibly on $H$, also
$\{ (\iota\otimes\omega) (\hat VV)y : \omega\in\mc B(H)_*, y\in\hat S \}$
is linearly dense in $\mc B_0(H)$.

Now, $(\iota\otimes\omega) (\hat VV)y = (\iota\otimes\omega)
(\hat VV(y\otimes 1))$ and as $V$ is a unitary multiplier of
$\hat S\otimes\mc B_0(H)$ (by Proposition~\ref{prop:7}(\ref{prop:7.1}))
it follows that
\[ \{ (\iota\otimes\omega) (\hat V(y\otimes 1)) :
\omega\in\mc B(H)_*, y\in\hat S \} \]
is linearly dense in $\mc B_0(H)$.  As $(\iota\otimes\omega)(\hat V(y\otimes 1))
= (U\omega U\otimes\iota)(V) y = L(U\omega U)y$ the result follows.
\end{proof}

\begin{definition}[D\'efinition 6.4]
A \emph{Kac system} is a triple $(H,V,U)$ where $H$ is a Hilbert space,
$V$ is a biregular multiplicative unitary (see Definition~\ref{defn:1})
and $U$ is a unitary verifing that $V$ is also irreducible.
\end{definition}

\begin{lemma}[D\'efinition 6.5]\label{lem:2}
Let $(H,V,U)$ be a Kac system.  Then:
\begin{enumerate}
\item\label{lem:2.1} $(H,\Sigma V^* \Sigma,U)$ and $(H,\hat V,U)$
   are Kac systems;
\item\label{lem:2.2} The unitaries $V_{12}$ and $\tilde V_{23}$ commute;
\item\label{lem:2.3} The unitaries $V_{23}$ and $\hat V_{12}$ commute.
\end{enumerate}
\end{lemma}
\begin{proof}
By definition, $V$ is biregular if and only if $\mc C(V) =
\{(\iota\otimes\omega)(\Sigma V) : \omega\in\mc B(H)_*\}$ is dense in
$\mc B_0(H)$ and $\{(\omega\otimes\iota(\Sigma V) : \omega\in\mc B(H)_*\}
= \{(\iota\otimes\omega(V\Sigma) : \omega\in\mc B(H)_*\}
= \{(\iota\otimes\omega(\Sigma\hat V) : \omega\in\mc B(H)_*\}
= \mc C(\hat V)$ is dense in $\mc B_0(H)$.  That is, $V$ is biregular
if and only if $V$ and $\hat V$ are regular.

So set $W=\Sigma V^*\Sigma$, so
\[ \hat W = \Sigma(U\otimes 1)\Sigma V^*\Sigma(U\otimes 1)\Sigma
= (1\otimes U) V^* (1\otimes U) = \Sigma \tilde V^* \Sigma. \]
Similarly $\tilde W = \Sigma \hat V^* \Sigma$.  Then (\ref{lem:2.1}) follows.

As $\hat V V \tilde V = (U\otimes 1)\Sigma$ we see that
$\tilde V^*_{23} = \Sigma_{23}(1\otimes U\otimes 1)\hat V_{23} V_{23}
= (1\otimes 1\otimes U)(\Sigma \hat V V)_{23}$ which commutes with $V_{12}$
by Proposition~\ref{prop:10}(\ref{prop:10.4}).  Hence also $\tilde V_{23}$
commutes with $V_{12}$, giving (\ref{lem:2.2}).  Similarly,
Proposition~\ref{prop:10}(\ref{prop:10.5}) shows (\ref{lem:2.3}).
\end{proof}

\begin{definition}[D\'efinition 6.6]
We say that $(H,\hat V,U)$ is the \emph{dual Kac system} to $(H,V,U)$, and that
$(H,\Sigma V^*\Sigma,U)$ is the \emph{opposite Kac system} to $(H,V,U)$.
Two Kac systems $(H,V,U)$ and $(H',V',U')$ are \emph{isomorphic} if there is a
unitary $w\in\mc B(H,H')$ with $(w\otimes w)V=V'(w\otimes w)$ and $wU=U'w$.
We also say that $(H',V',U')$ is \emph{dual to} $(H,V,U)$ if it is isomorphic
to $(H,\hat V,U)$.
\end{definition}

Notice that the Kac systems $(H,\hat V,U)$ and $(H,\tilde V,U)$ are isomorphic
(by $U$).

\begin{definition}[D\'efinition 6.7]
Let $(H,V,U)$ be a Kac system.  For $\omega\in\mc B(H)_*$, we write
\[ \lambda(\omega) = L_{\hat V}(\omega) = (\omega\otimes\iota)(\hat V), \quad
R(\omega) = \rho_{\tilde V}(\omega) = (\iota\otimes\omega)(\tilde V). \]
\end{definition}

[Note: At this point, the original paper overloads notation, and seems to
write $L$ for both the map $\mc B(H)_*\rightarrow S\subseteq\mc B(H)$, and
also for the (trivial) representation of $S$ on $\mc B(H)$.  Then $\lambda$
is now both a map $\mc B(H)_*\rightarrow U\hat SU$, and also the representation
$\hat S\rightarrow \mc B(H)$ given by $y\mapsto UyU$.
We have tried to avoid doing this, and continue to view $S$ and $\hat S$
as concrete subalgebras of $\mc B(H)$.]

\begin{proposition}(Proposition 6.8)\label{lem:3}
We have that:
\begin{enumerate}
\item\label{lem:3.1} $\lambda(\omega) = U \rho(\omega) U$ and
$R(\omega) = UL(\omega)U$;
\item\label{lem:3.2} For all $\omega,\omega'\in\mc B(H)_*$, the operators
$\rho(\omega)$ and $\lambda(\omega')$ commute, and also
$L(\omega)$ and $R(\omega')$ commute;
\item\label{lem:3.3} For $x\in S,y\in\hat S$ we have that
\[ \delta(x) = \hat V^*(1\otimes x)\hat V, \qquad
(U\otimes U)\hat\delta(y)(U\otimes U) = \hat V(UyU\otimes 1)\hat V^*. \]
\end{enumerate}
\end{proposition}
\begin{proof}
For (\ref{lem:3.1}) we simply calculate that
\[ \lambda(\omega) = (\omega\otimes\iota)(\Sigma(U\otimes 1)V(U\otimes 1)\Sigma)
= U (\iota\otimes\omega)(V) U = U\rho(\omega)U, \]
the other case following similarly.

For (\ref{lem:3.2}) we see that
\[ \rho(\omega) \lambda(\omega')
= (\iota\otimes\omega)(V) (\omega'\otimes\iota)(\hat V)
= (\omega'\otimes\iota\otimes\omega)(V_{23} \hat V_{12}), \]
and so the result follows from Lemma\ref{lem:2}(\ref{lem:2.3}).
The other case uses Lemma\ref{lem:2}(\ref{lem:2.2}).

Let $\omega\in\mc B(H)_*$ and set $x=L(\omega)$.  Then
\begin{align*} \delta(x) &= V((\omega\otimes\iota)(V)\otimes 1)V^* = 
(\omega\otimes\iota\otimes\iota)(V_{23}V_{12}V_{23}^*)
= (\omega\otimes\iota\otimes\iota)(V_{12}V_{13}) \\
&= (\omega\otimes\iota\otimes\iota)(\hat V_{23}^* V_{13} \hat V_{23})
= \hat V^*(1\otimes x)\hat V, \end{align*}
where we have used that $V$ is multiplicative, and also
Proposition~\ref{prop:10}(\ref{prop:10.2}).  Then the first part of
(\ref{lem:3.3}) follows as such $x$ are dense in $S$.
Similarly, using Proposition~\ref{prop:10}(\ref{prop:10.3}) shows that
\[ \hat\delta(y) = \tilde V (y\otimes 1) \tilde V^*
\qquad (y\in\hat S). \]
Then the second part of (\ref{lem:3.3}) follows immediately.
\end{proof}

\begin{proposition}[Proposition 6.9]\label{prop:11}
Let $V$ be a mutliplicative unitary on $H$, and let $U\in\mc B(H)$ be
a unitary with $U^2=1$, and such that $V_{12}$ and $\tilde V_{23}$ commute,
and $\hat V_{12}$ and $V_{23}$ commute.   Then:
\begin{enumerate}
\item\label{prop:11.1} If the set $\{ \rho(\omega) L(\omega') : \omega,\omega'
\in\mc B(H)_*\}$ is linearly dense in $\mc B_0(H)$, then $V$ is regular;
\item\label{prop:11.2} If $\hat V$ is multiplicative, and both
$(S\cup \hat S)' = \mathbb C1$ and $(S\cup U\hat SU)'=\mathbb C1$, then
$(1\otimes U)\Sigma \hat V V \tilde V \in\mathbb C 1$.
\end{enumerate}
\end{proposition}
\begin{proof}
We first prove (\ref{prop:11.1}).
Let $\omega,\omega'\in\mc B(H)_*$, set $x = (\iota\otimes\omega)(\Sigma V)
\in\mc C(V)$ and set $s=UL(\omega')U=R(\omega')
=(\iota\otimes\omega')(\tilde V)$.  As $V_{12}$ and $\tilde V_{23}$ commute,
it follows that $(1\otimes s)V = V(1\otimes s)$ and so
\[ sx = (\iota\otimes\omega)((s\otimes 1)\Sigma V)
= (\iota\otimes\omega)(\Sigma V(1\otimes s))
= (\iota\otimes s\omega)(\Sigma V) \in\mc C(V). \]
As $A(V)H$ is linearly dense in $H$ (by Proposition~\ref{prop:3}) it follows
that $\mc C(V)$ has the same closure as the linear span of $UA(V)U\mc C(V)$.

Similarly, setting $t=U\rho(\omega')U=(\omega'\otimes \iota)(\hat V)$ and
using that $\hat V_{12}$ and $V_{23}$ commute will show that
$\mc C(V) U\hat A(V)U$ has closed linear span equal to the closure of
$\mc C(V)$.

We hence see that $\mc C(V)^2$ has closed linear span equal to
$\overline{\lin} \mc C(V) U\hat A(V) \hat A(V) U \mc C(V)$.  As remarked
after Proposition~\ref{prop:5}, $\mc C(V)^2$ is linearly dense in $\mc C(V)$.
By hypothesis, $\hat A(V) \hat A(V)$ is linearly dense in $\mc B_0(H)$.
As $V$ is unitary, it is easy to see that $\mc C(V)H$ and $\mc C(V)^*H$ are
linearly dense in $H$.  It follows that $\mc C(V) U\hat A(V) \hat A(V) U
\mc C(V)$ is linearly dense in $\mc B_0(H)$, and so the same is true of
$\mc C(V)$ showing that $V$ is regular.

For (\ref{prop:11.2}), set $W=(1\otimes U)\Sigma\hat V V \tilde V$.  As
$V_{12}$ commutes with $\tilde V_{23}$, and as we can now apply
Proposition~\ref{prop:10}(\ref{prop:10.4}), we conclude that $V_{12}$ and
$W_{23}$ commute.  Applying Proposition~\ref{prop:10}(\ref{prop:10.4}) to
$\tilde V$, and noting that $\hat{\tilde V} = V$,
we see that $\tilde V_{12}$ and $\Sigma_{23} V_{23} \tilde V_{23}$
commute.  As $\hat V_{12}$ and $V_{23}$ commute, also
$\tilde V_{12}$ and $(1\otimes U\otimes U)V_{23}(1\otimes U\otimes 1)$
commute.  As $W=(U\otimes U)V(U\otimes 1)\Sigma V\tilde V$, we conclude
that $\tilde V_{12}$ and $W_{23}$ commute.  So $W$ will commute with
$(x\otimes 1)$ for all $x$ of the form $(\omega\otimes\iota)(V)$ and of the
form $(\omega\otimes\iota)(\tilde V) = (\omega\otimes\iota)(\Sigma(1\otimes U)
V (1\otimes U)\Sigma) = (\iota\otimes U\omega U)(V)$, that is, for all
$x\in S \cup \hat S$.

If we replace $V$ by $\hat V$ in the argument of the previous paragraph, then
as $\hat{\hat V} = (U\otimes U)V(U\otimes U)$ and $\tilde{\hat V}=V$,
we see that $X=(1\otimes U)\Sigma (U\otimes U)V(U\otimes U)\hat VV$
commutes with $1\otimes x$ for all $x$ of the form $(\omega\otimes\iota)(\hat V)
= U\rho(\omega)U$ and of the form $(\iota\otimes\omega)(\hat V)=
L(U\omega U)$.  That is, for all $x\in S\cup U\hat SU$.  As
$X = \Sigma(U\otimes 1)W(U\otimes 1)\Sigma$, we conclude that $W$
commutes with $1\otimes x$ for all $x\in S\cap U\hat SU$.
Thus $W\in\mathbb C1$ as required.
\end{proof}

\begin{corollary}[Corollaire 6.10]
Let $V$ be a multiplicative unitary and let $U\in\mc B(H)$ be a unitary with
$U^2=1$.  Form $\hat V,\tilde V$ as before, and suppose that $\hat V$ is
multiplicative, that $V_{12}$ commutes with $\tilde V_{23}$, and that
$\hat V_{12}$ commutes with $V_{23}$.  If the closed linear span of
$\{ x UyU : x\in S, y\in \hat S \}$ is $\mc B_0(H)$, then $\tilde V$ and
$\hat V$ are regular.
\end{corollary}
\begin{proof}
Apply the previous proposition to $\hat V$.
\end{proof}

\begin{examples}[Exemples 6.11]
\begin{enumerate}
\item The multiplicative unitary $1\in\mc B(H\otimes H)$ is not
irreducible unless $H=\mathbb C$, as $(\Sigma(1\otimes U))^3
= \Sigma(U\otimes 1)$.
\item Let $G$ be a locally compact group, equipped with the right Haar measure.
Define a unitary $U$ on $L^2(G)$ by $(U\xi)(t)=\Delta^{1/2}(t)\xi(t^{-1})$,
where $\Delta$ is the modular function for the Haar measure.  Then
$(L^2(G), V_G, U)$ is a Haar system (with $V_G\xi(s,t)=\xi(st,t)$ as
in Examples~\ref{eg:1}).  Indeed, we showed in Examples~\ref{eg:2} that $V_G$
is regular.  Then $\Sigma(1\otimes U)V_G\xi(s,t) = V_G\xi(t,s^{-1})\Delta^{1/2}(s)
= \xi(ts^{-1},s^{-1})\Delta^{1/2}(s)$, and it follows that
$(\Sigma(1\otimes U)V_G)^3=1$.  Then $\hat V_G\xi(s,t) = \xi(s,s^{-1}t)
\Delta^{1/2}(s)$ and direct calculation shows this to be multiplicative
and regular.
\item Let $(A,\delta)$ be a compact quantum group and form $(H,V,U)$ as
in Section~\ref{sec:5}.  \textbf{TO FINISH!}
\item Let $W$ be the fundamental unitary of Kac-von Neumann algebra
(see \cite{r6}).  Let $V=W^*$ and set $U=J\hat J = \hat JJ$ (see \cite{r38}).
As $\hat V$ is the fundamental unitary associated with the dual Kac-von Neumann
algebra, it is regular.  It's a result of \cite{r38}, and
Proposition~\ref{prop:11}, that $(1\otimes U)\Sigma \hat V V \tilde V$ is a
scalar, and in fact, it's not hard to show that $(1\otimes U)\Sigma \hat V V
\tilde V = 1$.  Thus $(H,V,U)$ is a Kac system.
\end{enumerate}
\end{examples}

\begin{remark}(Remarque 6.12)
\begin{enumerate}
\item Let $(H,U,V)$ be a Kac system.  As $\hat{\hat V} = \tilde{\tilde V}
= (U\otimes U)V(U\otimes U)$ we have that $(1\otimes U)\Sigma \hat V V \tilde V
= \hat{\hat V} \hat V V (1\otimes U)\Sigma$.  It follows that $\hat V V \tilde V
= \hat{\hat V} \hat V V = (U\otimes 1)\Sigma$ and so $\hat V V \tilde V
= \hat{\hat V} \hat V V = \tilde V \hat{\hat V} \hat V = V \tilde V
\tilde{\tilde V}$.
\item The operator $\mc R = V(U\otimes 1)V(U\otimes 1)$ satisfies the Yang-Baxter
equation: $\mc R_{12} \mc R_{13} \mc R_{23} = \mc R_{23} \mc R_{13} \mc R_{12}$.
\item Some comments about \cite{r11}.
\end{enumerate}
\end{remark}





\section{Multiplicative unitaries and Takesaki-Takai biduality}

Fix a Kac system $(H,V,U)$.

\begin{definition}(D\'efinition 7.1)
Let $\delta_A$ be a coaction of $S$ (or $\hat S$) on a C$^*$-algebra $A$.
Write $\pi_L$ and $\pi_R$ (respectively, $\hat\pi_\lambda$ and $\hat\pi_\rho$)
for the representations of $A$ on the Hilbert C$^*$-module $A\otimes H$
defined by
\[ \pi_L = (\iota\otimes\iota)\circ\delta_A, \qquad
\pi_R = (\iota\otimes U(\cdot)U)\circ\delta_A, \]
respectively,
\[ \hat\pi_\lambda = (\iota\otimes U(\cdot)U)\circ\delta_A, \qquad
\hat\pi_\rho = (\iota\otimes\iota)\delta_A. \]

Denote by $A\times\hat S$ (respectively $A\times S$) the \emph{crossed product}
of $A$ by $S$ (respectively, $\hat S$), which is the C$^*$-algebra generated
by $\{ \pi_L(a)(1\otimes \rho(\omega)) : a\in A, \omega\in\mc B(H)_*\}$
(respectively, $\{ \hat\pi_\lambda(a)(1\otimes L(\omega)) : a\in A,
\omega\in\mc B(H)_*\}$) inside $\mc B(A\otimes H)$.
\end{definition}

Here $U(\cdot)U$ is the $*$-homomorphism $S\rightarrow \mc B(H);
x\mapsto UxU$ (the notation $\pi_R$ being inspired by Proposition~\ref{lem:3}).
[The odd notation is due to the fact that we are concretely viewing
$S$ as a subalgebra of $\mc B(H)$; whereas the original paper has by this
point started using $L$ to denote the inclusion map $S\rightarrow
\mc B(H)$, and so forth; see the comment before Proposition~\ref{lem:3}.]

In fact, it is not really necessary to work with $A\otimes H$.  Instead,
we could work in $M(A\otimes\mc B_0(H))$, noticing that clearly
$M(A\otimes S)$ and $M(A\otimes \hat S)$ are subalgebras of 
$M(A\otimes\mc B_0(H))$.  Then we can form $A\times S$ and $A\times\hat S$
inside $M(A\otimes\mc B_0(H))$.

\begin{lemma}(Lemme 7.2, see \cite{r23})\label{lem:4}
The crossed product $A\times\hat S$ (or $A\times S$) is the closed linear
span of $\{ \pi_L(a)(1\otimes \rho(\omega)) : a\in A, \omega\in\mc B(H)_*\}$
(respectively, $\{ \hat\pi_\lambda(a)(1\otimes L(\omega)) : a\in A,
\omega\in\mc B(H)_*\}$).
\end{lemma}
\begin{proof}
We give a proof for $A\times\hat S$; the proof for $A\times S$ follows by
working with $\hat V$ in place of $V$.  We need to show that, for $a\in A$ and
$\omega\in\mc B(H)_*$, we have that $(1\otimes\rho(\omega))\pi_L(a)$
is in the closed linear span of
$\{ \pi_L(a)(1\otimes \rho(\omega)) : a\in A, \omega\in\mc B(H)_*\}$.
Let $\tilde\pi$ be the representation of $A$ on the Hilbert C$^*$-module
$A\otimes H\otimes H$ defined by
\[ \tilde\pi = (\pi_L\otimes\iota)\circ\delta_A
= (\iota\otimes\delta)\circ\delta_A, \]
which follows as $\delta_A$ is a coaction.  As $\delta_A(\cdot)
= V(\cdot\otimes 1)V^*$, we see that $\tilde\pi(\cdot) =
V_{23}\delta_A(\cdot)_{12}V_{23}^*$, and so
\[ (1\otimes\rho(\omega))\pi_L(a) = (\iota\otimes\iota\otimes\omega)
(V_{23}\pi_L(a)_{12})
= (\iota\otimes\iota\otimes\omega)(\tilde\pi(a)V_{23}). \]
Writing $\omega = \omega' s$ for some $\omega'\in\mc B(H)_*$ and $s\in S$,
we obtain
\[ (1\otimes\rho(\omega))\pi_L(a) =
(\iota\otimes\iota\otimes\omega')\big((\pi_L\otimes\iota)
\big( (1\otimes s)\delta_A(a)\big) V_{23}\big). \]
Now, $(1\otimes s)\delta_A(a) \in A\otimes S$ and so we can approximate
it by a linear span of elements of the form $b\otimes t$.  However, then
observe that
\[ (\iota\otimes\iota\otimes\omega')\big((\pi_L\otimes\iota)
(b\otimes t) V_{23}\big) = \pi_L(b) (1\otimes\rho(\omega' t)). \]
The result follows.
\end{proof}

The previous lemma shows that for each $a\in A$, we have that $\pi_L(a)
\in M(A\times\hat S)$ (by the definition of $A\times\hat S$, we see that
$\pi_L(a)$ is a left multiplier, and the lemma shows that it is also a right
multiplier).  Denote by $\pi$ the resulting $*$-homomorphism $A\rightarrow
M(A\times\hat S)$.  This is non-degenerate, as clearly $\pi(A)(A\times\hat S)$
is dense in $A\times\hat S$.  Similar remarks apply to $A\times S$, leading
to a non-degenerate $*$-homomorphism $\hat\pi:A\rightarrow A\times S$.
Similarly, for $x\in\hat S$, the map $1\otimes x \in M(A\times\hat S)$,
leading to a non-degenerate $*$-homomorphism $\hat\theta:\hat S\rightarrow
M(A\times\hat S)$.  We also obtain $\theta:S\rightarrow M(A\times S)$.

Denote by $\Psi_{L,\rho}$ and $\Psi_{R,\lambda}$ the representations of
$A\times\hat S$ on $A\otimes H$ defined by
\[ \Psi_{L,\rho}\big( \pi(a)\hat\theta(x) \big)
= \pi_L(a)(1\otimes x), \quad
\Psi_{R,\lambda}\big( \pi(a)\hat\theta(x) \big)
= \pi_R(a)(1\otimes UxU)
\qquad (a\in A, x\in\hat S). \]
[Again, chasing the definitions shows that $\Psi_{L,\rho}$ is just the
identity representation.]
Similarly define representations $\hat\Psi_{\lambda,L}$ and
$\hat\Psi_{\rho,R}$ of $A\times S$ on $A\otimes H$ by
\[ \hat\Psi_{\lambda,L}\big( \hat\pi(a) \theta(y) \big)
= \hat\pi_\lambda(a)(1\otimes y), \quad
\hat\Psi_{\rho,R}\big( \hat\pi(a) \theta(y) \big)
= \hat\pi_\rho(a)(1\otimes UyU)
\qquad (a\in A, y\in S). \]

\begin{definition}(D\'efinition 7.3)
Let $\delta_A$ be a coaction of $S$ (respectively, $\hat S$) on $A$.
The \emph{dual coaction} of $\hat S$ (respectively, $S$) on $A\times \hat S$
(respectively $A\times S$) by
\[ \delta_{A\times\hat S}:A\times\hat S \rightarrow
   M(A\times\hat S\otimes\hat S); \quad
   \pi(a)\hat\theta(x) \mapsto (\pi(a)\otimes 1)
   (\hat\theta\otimes\iota)\hat\delta(x)
\qquad (a\in A, x\in\hat S). \]
\[ \delta_{A\times S}:A\times S \rightarrow
   M(A\times S\otimes S); \quad
   \hat\pi(a)\theta(x) \mapsto (\hat\pi(a)\otimes 1)
   (\theta\otimes\iota)\delta(x)
\qquad (a\in A, x\in S). \]
\end{definition}

Notice that for $y=\hat\theta(x) = 1\otimes x$, we have that
\[ \tilde V_{23}(y\otimes 1)\tilde V_{23}^* 
= 1 \otimes \tilde V(x\otimes 1)\tilde V^*
= 1\otimes\hat\delta(x), \]
thanks to (the proof of) Proposition~\ref{lem:3}.  For $y=\pi(a) =\delta(a)
= V^*(a\otimes 1)V$, we have that
\[ \tilde V_{23}(y\otimes 1)\tilde V_{23}^*
= \tilde V_{23}V_{12}^*(a\otimes 1\otimes 1)V_{12}\tilde V_{23}^*
= V_{12}^* \tilde V_{23}(a\otimes 1\otimes 1)\tilde V_{23}^*V_{12}
= \delta(a)\otimes 1, \]
where here we used Lemma~\ref{lem:2}(\ref{lem:2.2}).  As such elements $y$
generate $A\times \hat S$, it follows that $\delta_{A\times\hat S}(\cdot)
= \tilde V_{23}(\cdot\otimes 1)\tilde V_{23}^*$, and so
$\delta_{A\times\hat S}$ is well-defined and a $*$-homomorphism.
Similar remarks apply to $\delta_{A\times S}$.





\appendix
\section{Useful results}

The following is an assortment of results which are used implicitly by
Baaj and Skandalis.  We prove (sketch) proofs to aid the reader.

\begin{lemma}\label{lem:ap1}
Let $A$ be a C$^*$-algebra.  Then $A^* = \{ a\mu : a\in A, \mu\in A^* \}
= \{ \mu a : a\in A, \mu\in A^* \}$.  Let $A$ act faithfully on a Hilbert
space $H$.  Then $\mc B(H)_* = \{ a\omega : a\in A, \omega\in\mc B(H)_* \}
= \{ \omega a : a\in A, \omega\in\mc B(H)_* \}$.
\end{lemma}
\begin{proof}
We firstly claim that $\{ a\mu : a\in A,\mu\in A^* \}$ is linearly dense in
$A^*$-- this follows by a GNS argument, see \cite[Appendix~A]{mnw}.
Then the Cohen Factorisation Theorem shows that actually
$A^* = \{ a\mu : a\in A, \mu\in A^* \} = \{ \mu a : a\in A, \mu\in A^* \}$.
Indeed, given $\lambda\in A^*$ and $\epsilon>0$, we can find $a\in A$ with
$\|a\|\leq 1$ and $\mu\in A^*$ with $a\mu = \lambda$ and
$\|\mu-\lambda\|<\epsilon$.

That $A$ acts non-degenerately on $H$ means, again using the Cohen Factorisation
Theorem, that $H = \{ a(\xi) : a\in A, \xi\in H \}$.  It follows that
$\{ a\omega : a\in A, \omega\in\mc B(H)_* \}$ is linearly dense in $\mc B(H)_*$,
so the result again follows by Cohen Factorisation.
\end{proof}

\begin{thebibliography}{99}

\bibitem{r1}

\bibitem{r2}

\bibitem{r6}

\bibitem{r7}

\bibitem{r11}

\bibitem{r13}

\bibitem{r17}

\bibitem{r23}

\bibitem{r30}

\bibitem{r33}

\bibitem{r38}

\bibitem{r50} Ref 50

\bibitem{r54} Ref 54

\medskip
\hspace{-4ex}The following are extra bibliographic entries not in the original paper.

\bibitem[lan]{lance} Lance's Hilbert C$^*$-module book.

\bibitem[mnw]{mnw} Masuda et al. ``$C^*$-algebraic framework for Quantum Groups''.

\bibitem[wor]{woro} Woronowicz's Compact Quantum Groups paper.

\end{thebibliography}

\end{document}



