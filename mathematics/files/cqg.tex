\documentclass[twoside,a4paper,12pt]{article}

\usepackage[margin=2cm]{geometry}

\usepackage{amsmath,amssymb,latexsym,amsthm}
\usepackage{showkeys}
\usepackage[active]{srcltx}
\usepackage[all]{xy}

\theoremstyle{plain}
\newtheorem{proposition}{Proposition}[section]
\newtheorem{theorem}[proposition]{Theorem}
\newtheorem{corollary}[proposition]{Corollary}
\newtheorem{lemma}[proposition]{Lemma}
\theoremstyle{definition}
\newtheorem{definition}[proposition]{Definition}
\newtheorem{example}[proposition]{Example}
%\newtheorem{remark}[proposition]{Remark}
%\newtheorem{question}[proposition]{Question}
%\newtheorem{problem}[proposition]{Problem}

\newcommand{\ip}[2]{\langle #1,#2 \rangle}
\newcommand{\mc}{\mathcal}
\newcommand{\mf}{\mathfrak}
\newcommand{\G}{{\mathbb G}}
\newcommand{\vnten}{\overline\otimes}
% The next command is shamelessly stolen from Piotr Soltan.
\newcommand{\cotp}{\xymatrix{*+<.7ex>[o][F-]{\scriptstyle\top}}}
\newcommand{\mor}{\operatorname{Mor}}
\newcommand{\tr}{\operatorname{Tr}}
\newcommand{\re}{\operatorname{Re}}
\newcommand{\lin}{\operatorname{lin}}
\newcommand{\op}{{\operatorname{op}}}

\begin{document}

\title{Notes on Compact Quantum Groups}
\author{Matthew Daws}

\section{Introduction}

A compact quantum group is a unital C$^*$-algebra $A$ together with a
coassociative map $\Delta:A\rightarrow A\otimes A$ such that
$\Delta(A)(A\otimes 1)$ and $\Delta(A)(1\otimes A)$ are linearly dense in
$A\otimes A$.  We get the Haar measure $\varphi$ which is the unique state
on $A$ with $(\varphi\otimes\iota)\Delta(a) = (\iota\otimes\varphi)\Delta(a)
= \varphi(a) 1$ for all $a\in A$.

As argued in my PAMS paper, we can find a maximal family of irreducible
unitary corepresentations $\{ v^\alpha = (v^\alpha_{ij})_{i,j=1}^{n_\alpha}
: \alpha\in\mathbb A\}$ such that the associated ``$F$-matrices'' are all
diagonal.

Firstly, if $\mc A$ is the linear span of $\{ v^\alpha_{ij} \}$, then $\mc A$
is a Hopf-$*$-algebra and is dense in $A$.  We have that
\[ \Delta(v^\alpha_{ij}) = \sum_k v^\alpha_{ik} \otimes v^\alpha_{kj},
\quad S(v^\alpha_{ij}) = (v^\alpha_{ji})^*, \quad
\epsilon(v^\alpha_{ij}) = \delta_{ij}, \quad
\varphi(v^\alpha_{ij}) = \delta_{\alpha,\alpha_0}, \]
where $\alpha_0$ is the unique member of $\mathbb A$ with $v^\alpha_0 = 1$.

Then we have positive numbers $(\lambda^\alpha_i)_{i=1}^{n_\alpha}$ such that
$\sum_i \lambda^\alpha_i = \sum_i (\lambda^\alpha_i)^{-1} = \Lambda_\alpha$ say.
We have that
\[ \varphi\big( (v^\alpha_{ij})^* v^\beta_{kl} \big)
= \delta_{\alpha,\beta} \delta_{i,k} \delta_{j,l}
\frac{1}{\Lambda_\alpha \lambda^\alpha_i}, \quad
\varphi\big( v^\alpha_{ij} (v^\beta_{kl})^* \big)
= \delta_{\alpha,\beta} \delta_{i,k} \delta_{j,l}
\frac{\lambda^\alpha_j}{\Lambda_\alpha}. \]

We define characters $f_z$, for $z\in\mathbb C$, on $\mc A$ by
\[ f_z\big( v^\alpha_{ij} \big) = \delta_{i,j} (\lambda^\alpha_i)^z, \]
where of course $t^z = \exp(z\log t)$ for $t>0$.  Then the modular automorphism
group for $\varphi$, restricted to $\mc A$, is given by
\[ \sigma_z: v^\alpha_{ij} \mapsto \sum_{k,l} f_{iz}(v^\alpha_{ik}) v^\alpha_{kl}
f_{iz}(v^\alpha_{lj})
= (\lambda^\alpha_i)^{iz} (\lambda^\alpha_j)^{iz} v^\alpha_{ij}. \]
For example, we can show that $\varphi(ba) = \varphi(a\sigma_{-i}(b))$ for
all $a,b\in\mc A$.  Also, as $J\Lambda(a) = \Lambda(\sigma_{i/2}(a)^*)$ for
$a\in\mc A$, we see that
\[ J\Lambda(v^\alpha_{ij}) = (\lambda^\alpha_i \lambda^\alpha_j)^{-1/2}
\Lambda((v^\alpha_{ij})^*). \]

Similarly, the scaling group on $\mc A$ is given by
\[ \tau_z: v^\alpha_{ij} \mapsto
(\lambda^\alpha_i)^{iz} (\lambda^\alpha_j)^{-iz} v^\alpha_{ij}. \]
Thus in particular,
\[ S(v^\alpha_{ij}) = (v^\alpha_{ji})^*
= R \tau_{-i/2}(v^\alpha_{ij})
= (\lambda^\alpha_i)^{1/2} (\lambda^\alpha_j)^{-1/2} R( v^\alpha_{ij} )
\implies
R( v^\alpha_{ij} ) = \sqrt\frac{\lambda^\alpha_j}{\lambda^\alpha_i}
(v^\alpha_{ji})^*. \]
However, also $R(x) = \hat J x^* \hat J$, and so
\[ \hat J v^\alpha_{ij} \hat J = \sqrt\frac{\lambda^\alpha_j}{\lambda^\alpha_i}
v^\alpha_{ji}. \]




\section{Reduced case and duality}

Now suppose that $\varphi$ is faithful.
Let $(H,\Lambda)$ be the GNS construction for $\varphi$.

For each $\alpha\in\mathbb A$, let $H_\alpha$ be the finite-dimensional subspace
of $H$ spanned by $\{ \Lambda((v^\alpha_{ij})^*) : 1\leq i,j\leq n_\alpha \}$.
Notice that $H_\alpha$ is orthogonal to $H_\beta$ for $\alpha\not=\beta$.
As $\mc A$ is dense in $H$, it follows that $H$ is isomorphic to the Hilbert
space direct sum of $\{ H_\alpha : \alpha\in\mathbb A \}$.  There is a
bijective linear map $U_\alpha: H_\alpha \rightarrow \ell^2_{n_\alpha} \otimes
\ell^2_{n_\alpha}$ given by
\[ U_\alpha : \Lambda((v^\alpha_{ij})^*) \mapsto
\sqrt\frac{\lambda^\alpha_j}{\Lambda_\alpha}
\delta_i \otimes \delta_j. \]
We have that $U_\alpha$ is unitary, because
\[ \big( U_\alpha((v^\alpha_{ij})^*) \big| U_\alpha((v^\alpha_{kl})^*) \big)
= \frac{\lambda^\alpha_j}{\Lambda_\alpha} \big( \delta_i\otimes\delta_j \big|
\delta_k\otimes\delta_l \big) 
= \varphi\big( v^\alpha_{kl} (v^\alpha_{ij})^* \big)
= \big( \Lambda((v^\alpha_{ij})^*) \big| \Lambda((v^\alpha_{kl})^*) \big). \]

From the general LCQG theory, we form the unitary operator $W^*$
on $H\otimes H$ by
\[ W^*\big( \Lambda(a) \otimes \Lambda(b) \big)
= (\Lambda\otimes\Lambda)(\Delta(b)(a\otimes 1))
\qquad (a,b\in A). \]
Notice that it is very easy to show that $W^*$ is unitary in the compact case.
It follows that
\[ W^*\big(\xi \otimes \Lambda((v^\alpha_{ij})^*) \big)
= \sum_k (v^\alpha_{ik})^*(\xi) \otimes \Lambda((v^\alpha_{kj})^*). \]

Now we calculate
\begin{align*} \big( W(\xi\otimes\Lambda((v^\alpha_{ij})^*)) \big|
   \eta\otimes\Lambda((v^\beta_{kl})^*) \big)
&= \sum_p \big( \xi \otimes \Lambda((v^\alpha_{ij})^*) \big|
   (v^\beta_{kp})^*(\eta) \otimes \Lambda((v^\beta_{pl})^*) \big) \\
&= ( v^\alpha_{ki}(\xi) | \eta ) \delta_{\alpha,\beta}
   \delta_{j,l} \frac{\lambda^\alpha_j}{\Lambda_\alpha} \\
&= \sum_p \big (v^\alpha_{pi}(\xi) \otimes \Lambda((v^\alpha_{pj})^*) \big|
   \eta \otimes \Lambda((v^\beta_{kl})^*) \big).
\end{align*}
It follows that for each $\alpha\in\mathbb A$, the unitary $W$ restricts
to $H\otimes H_\alpha$ and is the map
\[ W(\xi\otimes\Lambda((v^\alpha_{ij})^*))
= \sum_p v^\alpha_{pi}(\xi) \otimes \Lambda((v^\alpha_{pj})^*). \]
In particular, $(1\otimes U_\alpha)W(1\otimes U_\alpha^*)$ makes sense
on $H \otimes \ell^2_{n_\alpha} \otimes \ell^2_{n_\alpha}$ and is
\[ w_\alpha = (1\otimes U_\alpha)W(1\otimes U_\alpha^*):
\xi\otimes\delta_i\otimes\delta_j \mapsto
\sum_p v^\alpha_{pi}(\xi) \otimes \delta_p \otimes \delta_j. \]
Thus actually
\[ w_\alpha = \sum_{ij} v^\alpha_{ij} \otimes e_{ij} \otimes 1, \]
where $e_{ij}$ is the usual matrix unit in $\mathbb M_{n_\alpha}
\cong \mc B(\ell^2_{n_\alpha})$.

\subsubsection{Positive cone}

The positive cone of $L^2(\G)^+$ is by definition the closure of
$\{ x JxJ \Lambda(1) : x\in L^\infty(\G) \}$.	If $x\in L^\infty(\G)$
then there is a norm-bounded net $(a_\alpha)$ in $\mc A$ converging to $x$
strongly.  In particular $\Lambda(x) = x\Lambda(1) = \lim_\alpha
a_\alpha \Lambda(1) = \lim_\alpha \Lambda(a_\alpha)$ where the limits are
in the norm of $L^2(\G)$.  Then
\[ x JxJ \Lambda(1) = x J \Lambda(x)
= \lim_\alpha a_\alpha J\Lambda(x)
= \lim_\alpha a_\alpha J\Lambda(a_\alpha), \]
as $\| a_\alpha J\Lambda(x) - a_\alpha J\Lambda(a_\alpha)\|
\leq \|a_\alpha \| \|J\Lambda(x) - J\Lambda(a_\alpha)\| \rightarrow 0$.
Thus the positive cone is the closure of the set
$\{ aJ\Lambda(a) : a\in\mc A \}$.  Recall that $aJ\Lambda(a)
= a\Lambda(\sigma_{i/2}(a)^*) = \Lambda(a \sigma_{i/2}(a)^*)$.  In particular,
\[ P^{it} \Lambda(a \sigma_{i/2}(a)^*)
= \Lambda\big( \tau_t(a \sigma_{i/2}(a)^* \big)
= \Lambda\big( \tau_t(a) \sigma_{i/2}(\tau_t(a))^* \big), \]
which is in the positive cone (as $\tau_t\sigma_s = \sigma_s\tau_t$ for all
$s,t\in\mathbb R$).


\subsection{Further facts about the irreducible corepresentations}

We refer to later results; from our choices (compare
Proposition~\ref{prop:fmatrices}) we have that $F^\alpha$ is diagonal,
with entries $(\Lambda_\alpha \lambda^\alpha_i)^{-1}$.
By (the comment after) Corollary~\ref{cor:fmatrixkappa}, it follows that
\[ \sum_k u^\alpha_{k,i} \frac{\lambda^\alpha_k}{\lambda^\alpha_i}
(u^\alpha_{k,j})^* = \delta_{i,j}, \qquad
\sum_k (u^\alpha_{i,k})^* \frac{\lambda^\alpha_i}{\lambda^\alpha_k}
u^\alpha_{j,k} = \delta_{i,j}. \]
We could also prove these by writing down what it means for $u^\alpha$ to
be unitary, and then applying the map $R$, given the form for this which
we have established above (though is $A$ is not reduced, we then have to
argue a little about the uniqueness of the Hopf algebra.)

Below, we'll see that for each $\alpha$, the contragradient representation
$\overline{u^\alpha}$ is also irreducible (Lemma~\ref{lem:contra_irrep})
and is equivalent to a unitary corepresentation
(Proposition~\ref{prop:conjunitary}).  So there is an invertible scalar
matrix $T$ (which is unique, up to a scalar, by Schur's Lemma,
Proposition~\ref{prop:schur}) and some $\beta$, with
$(1\otimes T^{-1}) \overline{u^\alpha} (1\otimes T) = u^\beta$.
In Lemma~\ref{lem:fmatrixkappa} it's shown that
$TT^*$ is a scalar multiple of $\overline{F^\alpha}$; by considering the traces
of these positive definite matrices, this scalar multiple is a postive number.
It follows that, by rescaling $T$, we may suppose that
$T = (\overline{F^\alpha})^{1/2} U$ for some scalar unitary matrix $U$.

Thus we find that $(1 \otimes U^* (F^\alpha)^{-1/2}) \overline{u^\alpha}
(1\otimes (F^\alpha)^{1/2}U) = u^\beta$, and so
\[ \overline{u^\beta} = (1\otimes U^T (F^\alpha)^{-1/2}) u^\alpha
(1\otimes (F^\alpha)^{1/2} \overline{U})
\implies
(1\otimes (F^\alpha)^{1/2} \overline{U}) \overline{u^\beta}
(1\otimes U^T (F^\alpha)^{-1/2}) = u^\alpha. \]
However, by the same reasoning, there is a scalar unitary $V$ with
$(1 \otimes V^* (F^\beta)^{-1/2}) \overline{u^\beta}
(1\otimes (F^\beta)^{1/2}V) = u^\alpha$.  By Schur, $V^* (F^\beta)^{-1/2}
= \overline{\mu} (F^\alpha)^{1/2} \overline{U}$ for some $\mu\in\mathbb C$.
Thus $\mu (F^\alpha)^{1/2} U (F^\beta)^{1/2}$ is unitary, that is,
\[ |\mu|^2 (F^\beta)^{1/2} U^* F^\alpha U (F^\beta)^{1/2} = I
\quad\Leftrightarrow\quad
|\mu|^2 F^\alpha U = U (F^\beta)^{-1}. \]
As $U (F^\beta)^{-1} U^* = |\mu|^2 F^\alpha$, taking the trace of
both sides shows that $\Lambda_\beta^2 = |\mu|^2$.
Thus $F^\alpha U = U \Lambda_\beta^{-2} (F^\beta)^{-1}$.  Notice that
both the matrices $F^\alpha$ and $\Lambda_\beta^{-2} (F^\beta)^{-1}$
are diagonal, with strictly positive diagonal entries, and with unit trace.

\begin{lemma}
Let $U$ be a unitary matrix, and let $A,B$ be diagonal matrices with
non-zero diagonal entries $(a_i)$ and $(b_i)$.
For each diagonal entry $a$ of $A$,
let $E^A_a$ be the eigenspace of $a$, which is $\lin\{ e_i : a_i=a \}$.
Similarly define $E^B_b$.  Suppose that $AU=UB$.
Then, counting multiplicies, the sequences $\{ a_i \}$ and $\{ b_i\}$ are the
same, and $U$ restricts to a unitary between $E^A_{a_i}$ and $E^B_{a_i}$.
\end{lemma}
\begin{proof}
For each $i$, notice that $A(Ue_i) = UBe_i = b_i Ue_i$, so $b_i$ is an
eigenvalue of $A$, and hence there exists $j$ with $a_j=b_i$.  Similarly,
for each $j$ there is $i$ with $b_i=a_j$, so the sets $\{a_i\}$ and
$\{b_j\}$ agree.

Now observe that $U$ maps $E^B_{b_i}$ into $E^A_{b_i}$, so as $U$ is
invertible, the dimensions of these eigenspaces agree.  Thus, counting
multiplicities, the sequences $\{ a_i\}$ and $\{b_j\}$ agree, and the proof
is complete.
\end{proof}

So in our case $\{ (\Lambda_\alpha \lambda^\alpha_i)^{-1} \}$ and 
$\{ \lambda^\beta_j / \Lambda_\beta \}$ agree counting multiplicity, and
$U$ has the stated simple form.  Then $\Lambda_\alpha^2 = \sum_i \Lambda_\alpha
\lambda^\alpha_i = \sum_i \Lambda_\beta / \lambda^\beta_i = \Lambda_\beta^2$,
so $\Lambda_\alpha=\Lambda_\beta$.  Hence $\{ \lambda^\alpha_i \}$ and
$\{ 1/\lambda^\beta_j \}$ biject according to multiplicity.





\subsection{Duality}

\subsubsection{The involution on $L^1(A)$}

From general LCQG theory we have the homomorphism $\lambda:L^1(A) \rightarrow
\hat A$ given by $\omega\mapsto (\omega\otimes\iota)(W)$.  Recall the
involution $\sharp$ defined on $L^1_\sharp(A)$ which satisfies
\[ \ip{a}{\omega^\sharp} = \overline{ \ip{S(a)^*}{\omega} }
\qquad (a\in\mc A, \omega\in L^1(A)_\sharp). \]
Then $\lambda$ is a $*$-homomorphism when restricted to $L^1_\sharp(A)$.

For $a,b\in\mc A$ we define $\omega(a,b) = \omega_{\Lambda(a),\Lambda(b)}
\in L^1(A)$.  Then for $c\in\mc A$,
\begin{align*} \overline{ \ip{S(c)^*}{\omega(a,b)} }
&= \overline{ \varphi(b^*S(c)^*a) }
= \overline{ \varphi(S(b^*S(c)^*a)) }
= \overline{ \varphi(S(a) c^* S(b^*)) }
= \varphi(S(b^*)^* c S(a)^*) \\
&= \ip{c}{\omega(S(a)^*,S(b^*))}. \end{align*}
That $\varphi$ is $S$-invariant follows immediately from the action of
$\varphi$ and $S$ on the elements $v^\alpha_{ij}$.
Thus $\omega(a,b)\in L^1_\sharp(A)$ with $\omega(a,b)^\sharp =
\omega(S(a)^*,S(b^*))$.


\subsubsection{Identifying the dual}

Define the linear functional on $\mc A$ by
\[ \omega^\alpha_{ij} : v^\beta_{kl} \mapsto \delta_{\alpha,\beta}
\delta_{i,k} \delta_{j,l}. \]
Notice that
\[ \big( v^\beta_{kl} \Lambda((v^\alpha_{ij})^*) \big| \Lambda(1) \big)
\frac{\Lambda_\alpha}{\lambda^\alpha_j}
= \delta_{\alpha,\beta} \delta_{i,k} \delta_{j,l}
= \ip{v^\beta_{kl}}{\omega^\alpha_{ij}}, \]
from which it follows that
\[ \omega^\alpha_{ij} = \frac{\Lambda_\alpha}{\lambda^\alpha_j}
\omega_{\Lambda((v^\alpha_{ij})^*), \Lambda(1)}. \]
From the discussion above, $\omega^\alpha_{ij}\in L^1_\sharp(A)$.

We now compute
\begin{align*}
\lambda(\omega^\alpha_{ij}) \Lambda((v^\beta_{kl})^*)
&= \frac{\Lambda_\alpha}{\lambda^\alpha_j}
(\omega_{\Lambda((v^\alpha_{ij})^*), \Lambda(1)}\otimes\iota)(W)
\Lambda((v^\beta_{kl})^*) \\
&= \frac{\Lambda_\alpha}{\lambda^\alpha_j}
\sum_p \big( v^\beta_{pk}\Lambda((v^\alpha_{ij})^*) \big| \Lambda(1) \big)
   \Lambda((v^\beta_{pl})^*)
= \delta_{\alpha,\beta} \delta_{j,k} \Lambda((v^\beta_{il})^*).
\end{align*}
Thus each $H_\beta$ is an invariant subspace for
$\lambda(\omega^\alpha_{ij})$, and $\lambda(\omega^\alpha_{ij})=0$ on
$H_\beta$ for $\alpha\not=\beta$.  Furthermore,
\[ U_\alpha \lambda(\omega^\alpha_{ij}) U_\alpha^*(\delta_k\otimes\delta_l)
= \delta_{j,k} \delta_i \otimes \delta_l. \]
Hence $U_\alpha \lambda(\omega^\alpha_{ij}) U_\alpha^* = e_{ij}$ the $(i,j)$th
matrix entry of $\mathbb M_{n_\alpha}$, which acts on the 1st component of
$\ell^2_{n_\alpha} \otimes \ell^2_{n_\alpha}$ in the canonical way.

\begin{lemma}
The linear span of $\{ \omega^\alpha_{ij} : \alpha\in\mathbb A, 1\leq i,j
\leq n_\alpha \}$ is dense in $L^1(A)$.
\end{lemma}
\begin{proof}
As $\mc A$ is dense in $A$, it follows that $\{ \omega_{\Lambda(a),\Lambda(b)}
: a,b\in\mc A\}$ is linearly dense in $L^1(A)$.  For $a,b,c\in\mc A$,
\[ \ip{c}{\omega_{\Lambda(a),\Lambda(b)}}
= \varphi(b^*ca) = \varphi(\sigma_i(a)b^*c)
= \ip{c}{\omega_{\Lambda(\sigma_i(a)b^*),\Lambda(1)}}. \]
By continuity, this also holds when $c\in A$, and so we see that
$\{ \omega_{\Lambda(a),\Lambda(1)} : a\in\mc A\}$ is linearly dense in
$L^1(A)$, from which the result follows.
\end{proof}

We hence conclude that
\[ \hat A = \bigoplus_\alpha \mathbb M_{n_\alpha}. \]
Here, for each $\alpha\in\mathbb A$, the copy of $\mathbb M_{n_\alpha}$ acts
on the first factor of $\ell^2_{n_\alpha} \otimes \ell^2_{n_\alpha}
\cong H_\alpha$ and acts as $0$ on $H_\beta$ for $\beta\not=\alpha$, all
this happening on $H \cong \bigoplus_\alpha H_\alpha$.

We know that $W \in M \vnten \hat M$ and thus we can identify
$W$ as a member of $M\vnten\prod_\alpha \mathbb M_{n_\alpha}
= \prod_\alpha M\vnten \mathbb M_{n_\alpha}$.  The calculation in the previous
section immediately shows that $W = (v^\alpha_{ij}) \in
\mathbb M_{n_\alpha}(M) \cong M\vnten M_{n_\alpha}$.

Henceforth, write $e^\alpha_{ij} \in \mathbb M_{n_\alpha}$ for the
standard matrix units, acting on the $\alpha$ part of $H\cong
\bigoplus H_\alpha$.


\subsubsection{Scaling group}

We know that $\lambda(\omega\circ\tau_{-t}) = \hat\tau_t\lambda(\omega)$.
Firstly, we calculate that
\[ \ip{v^\beta_{kl}}{\omega^\alpha_{ij}\circ\tau_{-t}}
= (\lambda^\beta_k)^{-it} (\lambda^\beta_l)^{it}
\ip{v^\beta_{kl}}{\omega^\alpha_{ij}}
= (\lambda^\alpha_i)^{-it} (\lambda^\alpha_j)^{it}
\ip{v^\beta_{kl}}{\omega^\alpha_{ij}}. \]
Thus
\[ \hat\tau_t\big( e^\alpha_{ij} \big)
= \lambda\big( \omega^\alpha_{ij} \circ \tau_{-t} \big)
= (\lambda^\alpha_i)^{-it} (\lambda^\alpha_j)^{it}
\lambda\big( \omega^\alpha_{ij} \big)
= (\lambda^\alpha_i)^{-it} (\lambda^\alpha_j)^{it} e^\alpha_{ij}. \]


\subsubsection{The weight on $\hat M$}

From LCQG theory, we have a GNS construction for $\hat M$ given by
\[ \big( \hat\Lambda(\lambda(\omega)) \big| \Lambda(a) \big)
= \ip{a^*}{\omega} \qquad (a\in A), \]
for a suitable, dense collection of $\omega\in L^1(A)$.  Thus
\[ \big( \hat\Lambda(e^\alpha_{ij}) \big| \Lambda((v^\beta_{kl})^*) \big)
= \ip{v^\beta_{kl}}{\omega^\alpha_{ij}}
= \delta_{\alpha,\beta} \delta_{i,k} \delta_{j,l}
= \frac{\Lambda_\alpha}{\lambda^\alpha_j} \varphi\big( v^\beta_{kl}
(v^\alpha_{ij})^* \big)
= \frac{\Lambda_\alpha}{\lambda^\alpha_j}
\big( \Lambda( (v^\alpha_{ij})^* ) \big| \Lambda((v^\beta_{kl})^*) \big). \]
Thus
\[ \hat\Lambda(e^\alpha_{ij}) = \frac{\Lambda_\alpha}{\lambda^\alpha_j}
\Lambda( (v^\alpha_{ij})^*) \in H_\alpha
\implies
U_\alpha \hat\Lambda(e^\alpha_{ij}) =
\sqrt\frac{\Lambda_\alpha}{\lambda^\alpha_j} \delta_i \otimes \delta_j. \]
We now see that
\[ \hat\varphi\big( (e^\beta_{kl})^* e^\alpha_{ij} \big)
= \big( \hat\Lambda( e^\alpha_{ij} ) \big|
\hat\Lambda( e^\beta_{kl} ) \big)
= \frac{\Lambda_\alpha^2}{\lambda^\alpha_j\lambda^\alpha_l}
\big( \Lambda( (v^\alpha_{ij})^* ) \big|
\Lambda( (v^\beta_{kl})^* ) \big)
= \delta_{\alpha,\beta} \delta_{i,k} \delta_{j,l}
\frac{\Lambda_\alpha}{\lambda^\alpha_l}. \]
In particular,
\[ \hat\varphi\big( e^\alpha_{ij} \big) =
\delta_{i,j} \frac{\Lambda_\alpha}{\lambda^\alpha_i}. \]

Let $\hat T$ be the Tomita map, $\hat T\hat\Lambda(a) = \hat\Lambda(a^*)$
for $a\in\hat M$; notice that this will respect the decomposition $\hat M
= \prod_\alpha \mathbb M_{n_\alpha}$.  Then, on $\mathbb M_{n_\alpha}$,
\begin{align*}
\big( \hat\nabla \hat\Lambda(e^\alpha_{ij}) \big|
   \hat\Lambda(e^\alpha_{kl}) \big)
&= \big( \hat T\hat\Lambda(e^\alpha_{kl}) \big|
   \hat T  \hat\Lambda(e^\alpha_{ij}) \big)
= \big( \hat\Lambda(e^\alpha_{lk}) \big|
   \hat\Lambda(e^\alpha_{ji}) \big)
= \hat\varphi\big( e^\alpha_{ij} e^\alpha_{lk} \big)
= \delta_{j,l} \hat\varphi\big( e^\alpha_{ik} \big) \\
&= \delta_{j,l} \delta_{i,k} \Lambda_\alpha \lambda^\alpha_i
= \frac{\lambda^\alpha_i}{\lambda^\alpha_j}
\hat\varphi(e^\alpha_{lk} e^\alpha_{ij})
= \frac{\lambda^\alpha_i}{\lambda^\alpha_j}
\big( \hat\Lambda(e^\alpha_{ij}) \big|
   \hat\Lambda(e^\alpha_{kl}) \big),
\end{align*}
and so
\[ \hat\nabla \hat\Lambda(e^\alpha_{ij})
= \frac{\lambda^\alpha_i}{\lambda^\alpha_j} \hat\Lambda(e^\alpha_{ij})
\implies U_\alpha \hat\nabla U_\alpha^*(\delta_i\otimes\delta_j)
= \frac{\lambda^\alpha_i}{\lambda^\alpha_j}
\delta_i\otimes\delta_j. \]
By uniqueness of positive square-roots, it follows that
\[\hat J \hat\Lambda(e^\alpha_{ji})
= \hat J\hat T \hat\Lambda(e^\alpha_{ij})
= \hat\nabla^{1/2} \hat\Lambda(e^\alpha_{ij})
= \sqrt\frac{\lambda^\alpha_i}{\lambda^\alpha_j} \hat\Lambda(e^\alpha_{ij}). \]
This also shows that
\[ \hat J \Lambda((v^\alpha_{ij})^*)
= \sqrt\frac{\lambda^\alpha_j}{\lambda^\alpha_i}
\Lambda((v^\alpha_{ji})^*)
= \lambda^\alpha_j J\Lambda(v^\alpha_{ji})
\implies J \hat J \Lambda((v^\alpha_{ij})^*)
= \lambda^\alpha_j \Lambda(v^\alpha_{ji}). \]
Finally, we also see that
\[ U_\alpha \hat J U_\alpha^*(\delta_i\otimes\delta_j)
= \sqrt\frac{\Lambda_\alpha}{\lambda^\alpha_j} U_\alpha \hat J
   \Lambda((v^\alpha_{ij})^*)
= \sqrt\frac{\Lambda_\alpha}{\lambda^\alpha_j}
  \sqrt\frac{\lambda^\alpha_j}{\lambda^\alpha_i}
  U_\alpha \Lambda((v^\alpha_{ji})^*)
= \sqrt\frac{\Lambda_\alpha}{\lambda^\alpha_i}
  U_\alpha \Lambda((v^\alpha_{ji})^*)
= \delta_j \otimes \delta_i. \]



\subsubsection{The antipode}

We calculate that
\begin{align*}
\hat R(e^\alpha_{ij}) \hat\Lambda(e^\beta_{kl})
&= \frac{\Lambda_\beta}{\lambda^\beta_l} J e^\alpha_{ji} J
   \Lambda((u^\beta_{kl})^*)
= \frac{\Lambda_\beta\sqrt{\lambda^\beta_k\lambda^\beta_l}}{\lambda^\beta_l}
   J e^\alpha_{ji} \Lambda(u^\beta_{kl}).
\end{align*}
From above, there is some $\gamma$ and a scalar unitary matrix $U$ with
$(1\otimes U^* (F^\beta)^{-1/2}) \overline{u^\beta} (1\otimes
(F^\beta)^{1/2} U) = u^\gamma$ and $\Lambda^2_\gamma F^\beta U
= U (F^\gamma)^{-1}$.  So $(1\otimes (F^\beta)^{1/2}U)u^\gamma
(1\otimes U^* (F^\beta)^{-1/2}) = \overline{u^\beta}$ and thus
$(1\otimes (F^\beta)^{1/2}\overline{U})\overline{u^\gamma}
(1\otimes U^T (F^\beta)^{-1/2}) = u^\beta$.  It follows that
\begin{align*} \hat R(e^\alpha_{ij}) \hat\Lambda(e^\beta_{kl})
&= \frac{\Lambda_\beta\sqrt{\lambda^\beta_k\lambda^\beta_l}}{\lambda^\beta_l}
   \sum_{p,q} J e^\alpha_{ji} ((F^\beta)^{1/2}\overline{U})_{k,p}
   (U^T (F^\beta)^{-1/2})_{q,l} \Lambda((u^\gamma_{pq})^*) \\
&= \frac{\Lambda_\beta\sqrt{\lambda^\beta_k\lambda^\beta_l}}{\lambda^\beta_l}
   \sum_{p,q} \frac{\sqrt{\lambda^\beta_l}}{\sqrt{\lambda^\beta_k}} U_{k,p}
   \overline{U_{l,q}} J e^\alpha_{ji} \Lambda((u^\gamma_{pq})^*) \\
&= \Lambda_\beta \sum_{p,q}  U_{k,p} \overline{U_{l,q}}
   \frac{\lambda^\gamma_q}{\Lambda_\gamma}
   J e^\alpha_{ji} \Lambda(e^\gamma_{pq})
= \delta_{\alpha,\gamma} \Lambda_\beta \sum_{q}  U_{k,i} \overline{U_{l,q}}
   \frac{\lambda^\gamma_q}{\Lambda_\gamma} J \hat\Lambda(e^\gamma_{jq}) \\
&= \delta_{\alpha,\gamma} \Lambda_\beta \sum_{q}  U_{k,i} \overline{U_{l,q}}
   J \Lambda((u^\gamma_{jq})^*)
= \delta_{\alpha,\gamma} \Lambda_\beta \sum_{q}  U_{k,i} \overline{U_{l,q}}
   \sqrt{\lambda^\gamma_j\lambda^\gamma_q} \Lambda(u^\gamma_{jq}) \\
&= \delta_{\alpha,\gamma} \Lambda_\beta \sum_{q}  U_{k,i} \overline{U_{l,q}}
   \sqrt{\lambda^\gamma_j\lambda^\gamma_q}
   \sum_{s,t} (U^*(F^\beta)^{-1/2})_{j,s} ((F^\beta)^{1/2} U)_{t,q}
   \Lambda((u^\beta_{st})^*) \\
&= \delta_{\alpha,\gamma} \Lambda_\beta \sum_{q}  U_{k,i} \overline{U_{l,q}}
   \sqrt{\lambda^\gamma_j\lambda^\gamma_q}
   \sum_{s,t} \overline{U_{s,j}} U_{t,q}
   \frac{\sqrt{\lambda^\beta_s}}{\sqrt{\lambda^\beta_t}}
   \Lambda((u^\beta_{st})^*)
\end{align*}

Now, we know that $\Lambda_\gamma U_{i,j} =
U_{i,j} \Lambda_\beta \lambda^\gamma_j \lambda^\beta_i$, for each $i,j$.
Similarly, as $\Lambda^2_\gamma U^* F^\beta U = (F^\gamma)^{-1}$, by the
uniqueness of positive square-roots, also $\Lambda_\gamma U^* (F^\beta)^{1/2} U
= (F^\gamma)^{-1/2}$, so $\sqrt{\Lambda_\gamma} U_{i,j} =
\sqrt{\Lambda_\beta \lambda^\gamma_j \lambda^\beta_i} U_{i,j}$.
So we get
\begin{align*} \hat R(e^\alpha_{ij}) \hat\Lambda(e^\beta_{kl})
&= \delta_{\alpha,\gamma} \sqrt{\Lambda_\beta} \sum_{q,s,t}
   U_{k,i} \overline{U_{l,q}}
   \sqrt{\lambda^\gamma_j} \overline{U_{s,j}}
   %\sqrt{\Lambda_\beta\lambda^\beta_t\lambda^\gamma_q} U_{t,q}
   \sqrt{\Lambda_\gamma} U_{t,q}
   \frac{\sqrt{\lambda^\beta_s}}{\lambda^\beta_t}
   \Lambda((u^\beta_{st})^*) \\
&= \delta_{\alpha,\gamma} \sqrt{\Lambda_\beta\Lambda_\gamma} \sum_s
   U_{k,i} \sqrt{\lambda^\gamma_j} \overline{U_{s,j}}
   \frac{\sqrt{\lambda^\beta_s}}{\lambda^\beta_l}
   \Lambda((u^\beta_{sl})^*) \\
&= \delta_{\alpha,\gamma} \sqrt{\Lambda_\beta\Lambda_\gamma} \sum_s
   U_{k,i} \sqrt{\lambda^\gamma_j} \overline{U_{s,j}}
   \frac{\sqrt{\lambda^\beta_s}}{\lambda^\beta_l}
   \frac{\lambda^\beta_l}{\Lambda_\beta}
   \hat\Lambda(e^\beta_{sl}) \\
&= \delta_{\alpha,\gamma} \sqrt{\frac{\Lambda_\gamma}{\Lambda_\beta}}
   U_{k,i} \sqrt{\lambda^\gamma_j} 
   \sum_s \overline{U_{s,j}} \sqrt{\lambda^\beta_s}
   \hat\Lambda(e^\beta_{sl})
= \delta_{\alpha,\gamma} \frac{\Lambda_\gamma}{\Lambda_\beta}
   U_{k,i} \sum_s
   %\sqrt{\Lambda\beta\lambda^\beta_s\lambda^\gamma_j} \overline{U_{s,j}} 
   \overline{U_{s,j}}
   \hat\Lambda(e^\beta_{sl}).
\end{align*}
It follows that, with $\beta$ being the unique index such that
$\overline{u^\alpha}$ is equivalent to $u^\beta$, and recalling that
$\Lambda_\alpha = \Lambda_\beta$, we have that
\[ \hat R(e^\alpha_{ij}) = \sum_{p,k} \frac{\Lambda_\alpha}{\Lambda_\beta}
U_{k,i} \overline{U_{p,j}} e^\beta_{p,k}
= (U^* e^\beta U)_{j,i}. \]
Hence indeed $\hat R$ is an isometry etc.

Next we calculate
\begin{align*} \hat\tau_{-i/2}(e^\alpha_{ij})\hat\Lambda(e^\beta_{kl})
&= \nabla^{1/2} e^\alpha_{ij} \nabla^{-1/2}\hat\Lambda(e^\beta_{kl})
= \frac{\Lambda_\beta}{\lambda^\beta_l}
\nabla^{1/2} e^\alpha_{ij} \nabla^{-1/2} \Lambda((u^\beta_{kl})^*) \\
&= \frac{\Lambda_\beta}{\lambda^\beta_l}
\nabla^{1/2} e^\alpha_{ij} \Lambda( \sigma_{i/2}((u^\beta_{kl})^*) )
= \frac{\Lambda_\beta}{\lambda^\beta_l}
\nabla^{1/2} e^\alpha_{ij} \sqrt{\lambda^\beta_k\lambda^\beta_l}
\Lambda( (u^\beta_{kl})^* ) \\
&= \nabla^{1/2} e^\alpha_{ij} \sqrt{\lambda^\beta_k\lambda^\beta_l}
\hat\Lambda( e^\beta_{kl} )
= \delta_{j,k} \delta_{\alpha,\beta}
\nabla^{1/2} \sqrt{\lambda^\beta_k\lambda^\beta_l}
\hat\Lambda( e^\beta_{il} ) \\
&= \delta_{j,k} \delta_{\alpha,\beta}
\sqrt{\lambda^\beta_k\lambda^\beta_l}
\frac{\Lambda_\beta}{\lambda^\beta_l}
\Lambda( \sigma_{-i/2}((u^\beta_{il})^*) )
= \delta_{j,k} \delta_{\alpha,\beta}
\sqrt{\lambda^\beta_k\lambda^\beta_l}
\frac{\Lambda_\beta}{\lambda^\beta_l}
(\lambda^\beta_i \lambda^\beta_l)^{-1/2}
\Lambda( (u^\beta_{il})^* ) \\
&= \delta_{j,k} \delta_{\alpha,\beta}
\sqrt{\frac{\lambda^\beta_j}{\lambda^\beta_i}}
\hat\Lambda( e^\beta_{il} )
= \sqrt{\frac{\lambda^\beta_j}{\lambda^\beta_i}}
e^\alpha_{ij} \hat\Lambda(e^\beta_{kl})
\end{align*}

So in conclusion, with $\alpha,\beta$ linked as before,
\[ \hat S(e^\alpha_{ij}) = \sqrt{\frac{\lambda^\beta_j}{\lambda^\beta_i}}
(U^* e^\beta U)_{j,i}. \]


\subsubsection{The coproduct}

For $\omega\in L^1(\G)$, we find that
\begin{align*} \hat\Delta(\lambda(\omega_{\xi,\eta}))
&= \hat\Delta\big( (\omega_{\xi,\eta}\otimes\iota)(W) \big)
= (\omega_{\xi,\eta}\otimes\iota\otimes\iota)(W_{13} W_{12}) \\
&= \sum_i (\omega_{\xi,e_i}\otimes\iota)(W) \otimes
   (\omega_{e_i,\eta}\otimes\iota)(W)
= \sum_i \lambda(\omega_{\xi,e_i}) \otimes
   \lambda(\omega_{e_i,\eta}), \end{align*}
where $(e_i)$ is an orthonormal basis for $H$.

We'll use the orthonormal basis $\{ U_\alpha^*(\delta_i\otimes\delta_j)
: \alpha\in\mathbb A, 1\leq i,j\leq n_\alpha \}$.  Now,
\[ ( \omega_{ \Lambda((v^\alpha_{ij})^*),
U_\beta^*(\delta_k\otimes\delta_l) } \otimes\iota)(W)
= \sum_{\gamma,s,t} e^\gamma_{st} \ip{v^\gamma_{st}}
{\omega_{ \Lambda((v^\alpha_{ij})^*),
U_\beta^*(\delta_k\otimes\delta_l) }}
= \sum_{\gamma,s,t} e^\gamma_{st}
\sqrt\frac{\Lambda_\beta}{\lambda^\beta_l}
\varphi( v^\beta_{kl} v^\gamma_{st} (v^\alpha_{ij})^* ), \]
and also
\[ ( \omega_{ U_\beta^*(\delta_k\otimes\delta_l) ,
\Lambda(1) } \otimes\iota)(W)
= \sum_{\gamma,s,t} \sqrt\frac{\Lambda_\beta}{\lambda^\beta_l}
e^\gamma_{st} \varphi( v^\gamma_{st}(v^\beta_{kl})^* )
= \sqrt\frac{\lambda^\beta_l}{\Lambda_\beta} e^\beta_{kl}. \]
Thus
\[ \hat\Delta( e^\alpha_{ij} )
= \frac{\Lambda_\alpha}{\lambda^\alpha_j}
\sum_{\beta,k,l} \sum_{\gamma,s,t}
\varphi( v^\beta_{kl} v^\gamma_{st} (v^\alpha_{ij})^* ) \,
e^\gamma_{st} \otimes e^\beta_{kl}. \]

Then
\begin{align*}
\hat\varphi\big( (e^{*\gamma}_{st}\otimes\iota)\hat\Delta(e^\alpha_{ij}) \big)
&= \frac{\Lambda_\alpha}{\lambda^\alpha_j}
\sum_{\beta,k,l} \varphi( v^\beta_{kl} v^\gamma_{st} (v^\alpha_{ij})^* )
\hat\varphi(e^\beta_{kl})
= \frac{\Lambda_\alpha}{\lambda^\alpha_j}
\sum_{\beta,k} \varphi( v^\beta_{kk} v^\gamma_{st} (v^\alpha_{ij})^* )
\frac{\Lambda_\beta}{\lambda^\beta_k}
\end{align*}



\subsection{Aspects of the locally compact setting}

Recall the operator $P$ defined by $P^{it}\Lambda(a) = \Lambda(\tau_t(a))$
(the scaling constant is trivial).  Thus
\begin{align*} U_\alpha P^{it} U_\alpha^*(\delta_i\otimes\delta_j) &=
U_\alpha \sqrt{\frac{\Lambda_\alpha}{\lambda^\alpha_j}}
   P^{it} \Lambda((v^\alpha_{ij})^*)
= U_\alpha \sqrt{\frac{\Lambda_\alpha}{\lambda^\alpha_j}}
   \Lambda(\tau_t(v^\alpha_{ij})^*) \\
&= U_\alpha \sqrt{\frac{\Lambda_\alpha}{\lambda^\alpha_j}}
   (\lambda^\alpha_i)^{-it} (\lambda^\alpha_j)^{it} 
   \Lambda(\tau_t(v^\alpha_{ij})^*)
= (\lambda^\alpha_i)^{-it} (\lambda^\alpha_j)^{it} \delta_i\otimes\delta_j.
\end{align*}


\section{Using the right regular representation}

It is more common to use the right regular representation, which we
shall denote by $V$.  This satisfies
\[ V(\Lambda(a) \otimes \Lambda(b)) = (\Lambda\otimes\Lambda)
(\Delta(a)(1\otimes b)), \]
where of course in generality $\Lambda$ is using the right Haar weight;
in the compact case, this agrees with the left Haar weight, of course.
Thus we see that
\[ V(\Lambda(v^\alpha_{ij}) \otimes \xi) = \sum_k \Lambda(v^\alpha_{ik})
\otimes v^\alpha_{kj}(\xi). \]

For each $\alpha\in\mathbb A$, let $H'_\alpha$ be the subspace of $H$
spanned by $\{ \Lambda(v^\alpha_{ij}) : 1\leq i,j\leq n_\alpha \}$.
As $\mc A$ is dense in $H$, it follows that $H$ is isomorphic to the Hilbert
space direct sum of $\{ H'_\alpha : \alpha\in\mathbb A \}$.  We can construct
a unitary $U'_\alpha: H_\alpha \rightarrow \ell^2_{n_\alpha} \otimes
\ell^2_{n_\alpha}$ given by
\[ U'_\alpha : \Lambda(v^\alpha_{ij}) \mapsto
(\Lambda_\alpha \lambda^\alpha_i)^{-1/2} \delta_i \otimes \delta_j. \]
This is clearly a linear bijection, and it is unitary because
\[ \big( U'_\alpha(v^\alpha_{ij}) \big| U'_\alpha(v^\alpha_{kl}) \big)
= \frac{1}{\Lambda_\alpha \sqrt{\lambda^\alpha_i \lambda^\alpha_k}}
\big( \delta_i\otimes\delta_j \big| \delta_k\otimes\delta_l \big)
= \delta_{i,k} \delta_{j,l} \frac{1}{\Lambda_\alpha \lambda^\alpha_i}
= \big( \Lambda(v^\alpha_{ij}) \big| \Lambda(v^\alpha_{kl}) \big). \]

So again $V$ restricts to an operator on $H_\alpha \otimes H$, and
\[ (U'_\alpha\otimes 1) V ({U'_\alpha}^*\otimes 1)
: \delta_i \otimes \delta_j \otimes \xi \mapsto
\sum_k \delta_i \otimes \delta_k \otimes v^\alpha_{kj}(\xi). \]

Setting
\[ \omega^\alpha_{ij} = \Lambda_\alpha \lambda^\alpha_i
\omega_{\Lambda(1), \Lambda(v^\alpha_{ij})}, \]
we see that
\[ \ip{v^\beta_{kl}}{\omega^\alpha_{ij}}
= \Lambda_\alpha \lambda^\alpha_i \varphi( (v^\alpha_{ij})^* v^\beta_{kl} )
= \delta_{\alpha,\beta} \delta_{i,k} \delta_{j,l}. \]
Then
\begin{align*}
\rho(\omega^\alpha_{ij}) \Lambda(v^\beta_{kl}) =
(\iota\otimes\omega^\alpha_{ij})(V) \Lambda(v^\beta_{kl})
&= \Lambda_\alpha \lambda^\alpha_i
(\iota\otimes\omega_{\Lambda(1), \Lambda(v^\alpha_{ij})})(V) 
\Lambda(v^\beta_{kl}) \\
&= \Lambda_\alpha \lambda^\alpha_i \sum_p
\Lambda(v^\beta_{kp}) \big( \Lambda(v^\beta_{pl}) \big|
   \Lambda(v^\alpha_{ij}) \big)
= \delta_{\alpha,\beta} \delta_{j,l} \Lambda(v^\alpha_{ki}).
\end{align*}
Thus $\rho(\omega^\alpha_{ij})$ restricts to the zero map on each $H_\beta$
with $\beta\not=\alpha$, and
\[ U'_\alpha \rho(\omega^\alpha_{ij}) {U'_\alpha}^* : \delta_k\otimes\delta_l
\mapsto \delta_{j,l} \delta_k \otimes \delta_i
\implies U'_\alpha \rho(\omega^\alpha_{ij}) {U'_\alpha}^* = 1\otimes e_{ij}. \]






\appendix
\section{Finding the unitary corepresentations}

\subsection{The left regular representation}\label{sec:leftregcorep}

\begin{definition}
A \emph{(unitary) corepresentation} of $(A,\Delta)$ is a (unitary) element
$U$ of $M(A\otimes\mc B_0(H))$ such that $(\Delta\otimes\iota)U
= U_{13} U_{23}$.
\end{definition}

Let $H$ have an orthonormal basis $(e_n)$, and let $U_{n,m}$ be the matrix
elements of $U$; this means that $U_{n,m} = (\iota\otimes\omega_{e_m,e_n})U
\in M(A)$.  Then $U$ is a corepresentation if and only if
\begin{align*} \Delta(U_{n,m})
&= (\iota\otimes\iota\otimes\omega_{e_m,e_n})(U_{13}U_{23})
= \sum_k (\iota\otimes\iota\otimes\omega_{e_k,e_n})(U_{13})
(\iota\otimes\iota\otimes\omega_{e_m,e_k})(U_{23}) \\
&= \sum_k U_{n,k} \otimes U_{k,m}. \end{align*}

Let $\varphi$ be the Haar state on $A$ and let $L^2(\varphi)$ be the GNS space,
with cyclic vector $\xi_0$.  Let $K$ be some auxiliary Hilbert space upon which
$A$ acts non-degenerately, say with $*$-homomorphism $\pi:A\rightarrow\mc B(K)$.
At this stage, we shall not assume that $\pi$ is injective.

\begin{proposition}
There is a (unique) unitary operator $U$ on $K\otimes L^2(\varphi)$ with
$U^*(\xi\otimes a\xi_0) = (\pi\otimes\iota)\Delta(a)(\xi\otimes\xi_0)$
for $a\in A$ and $\xi\in K$.
\end{proposition}
\begin{proof}
For $(a_i)\subseteq A$ and $(\xi_i)\subseteq K$, we have that
\begin{align*}
\Big\| \sum_i (\pi\otimes\iota)\Delta(a_i)(\xi_i\otimes\xi_0) \Big\|^2 &=
\sum_{i,j} \big( (\pi\otimes\iota)\Delta(a_j^*a_i) \xi_i\otimes\xi_0
   \big| \xi_j\otimes\xi_0 \big) \\
&= \sum_{i,j} \big( \pi((\iota\otimes\varphi)\Delta(a_j^*a_i))
   \xi_i\big|\xi_j\big) \\
&= \sum_{i,j} \varphi(a_j^*a_i) (\xi_i|\xi_j)
= \Big\| \sum_i \xi_i \otimes a_i\xi_0 \Big\|^2.
\end{align*}
This shows that $U^*$ is an isometry; clearly $U^*$ is densely defined,
and so $U^*$ extends to an isometry on all of $K\otimes L^2(\varphi)$.  As
$\Delta(A)(A\otimes 1)$ is linearly dense in $A\otimes A$, we see that
the image of $U^*$ contains the closed linear span of
\[ \big\{ \pi(a)\xi \otimes b\xi_0 : a,b\in A, \xi\in K \big\}. \]
As $A$ acts non-degenerately on $K$, this shows that $U^*$ is a surjection,
so $U$ is unitary as required.
\end{proof}

\begin{proposition}\label{prop:corepgivescomult}
The operator $U$ is a member of $M(\pi(A)\otimes\mc B_0(L^2(\varphi)))$,
and for $a\in A$, we have that $(\pi\otimes\iota)\Delta(a) =
U^*(1\otimes a)U$ in $\mc B(K\otimes L^2(\varphi))$.
\end{proposition}
\begin{proof}
For $a,b\in A$ and $\xi\in K$, we have that
$U^*(1\otimes a)(\xi\otimes b\xi_0)
= (\pi\otimes\iota)\Delta(ab)(\xi\otimes\xi_0)
= (\pi\otimes\iota)\Delta(a) U^*(\xi\otimes b\xi_0)$ and so
$U^*(1\otimes a)U = (\pi\otimes\iota)\Delta(a)$.

Let $a,b\in A, \xi_1,\xi_2\in L^2(\varphi)$ and $\xi\in K$.  For $\epsilon>0$ we
can find $\sum_i a_i\otimes b_i\in A\otimes A$ with $\|\sum_i a_i\otimes b_i
- \Delta(a)(b\otimes 1)\|<\epsilon$.  Then
\[ U^*(\pi(b)\otimes\theta_{a\xi_0,\xi_1})(\xi\otimes \xi_2)
= (\xi_2|\xi_1) U^*(\pi(b)\xi\otimes a\xi_0)
= (\xi_2|\xi_1) (\pi\otimes\iota)(\Delta(a)(b\otimes 1))(\xi\otimes\xi_0). \]
It follows that
\begin{align*} & \Big\| \Big( U^*(\pi(b)\otimes\theta_{a\xi_0,\xi_1})
   - \sum_i \pi(a_i) \otimes \theta_{b_i\xi_0,\xi_1}\Big)
   (\xi \otimes \xi_2) \Big\| \\
&= \Big\| (\xi_2|\xi_1) (\pi\otimes\iota)(\Delta(a)(b\otimes 1))
   (\xi\otimes\xi_0)
   - \sum_i \pi(a_i)\xi \otimes b_i\xi_0 (\xi_2|\xi_1) \Big\| \\
&\leq \epsilon \|\xi_2\| \|\xi_1\| \|\xi\| \|\xi_0\|.
\end{align*}
As $\epsilon>0$ was arbitrary, this shows that $U^*(\pi(b)\otimes 
\theta_{a\xi_0,\xi_1}) \in \pi(A)\otimes\mc B_0(L^2(\varphi))$.  By linearity and
continuity, $U^*(\pi(A)\otimes\mc B_0(L^2(\varphi))) \subseteq
\pi(A)\otimes\mc B_0(L^2(\varphi))$.

Now consider
\[ U(1\otimes\theta_{a\xi_0,\xi_1}) (\xi\otimes\xi_2)
= (\xi_2|\xi_1) U(\xi\otimes a\xi_0). \]
For $\epsilon>0$ we can find $(a_i),(b_i)\subseteq A$ with
$\| \sum_i \Delta(a_i)(b_i\otimes 1) - 1\otimes a\|<\epsilon$.  Then
\begin{align*} &
\Big\| \sum_i (\pi(b_i)\xi\otimes a_i\xi_0) - U(\xi\otimes a\xi_0) \Big\|
= \Big\| \sum_i U^*(\pi(b_i)\xi\otimes a_i\xi_0) - \xi\otimes a\xi_0 \Big\|\\
&= \Big\| \sum_i (\pi\otimes\iota)(\Delta(a_i)(b_i\otimes 1))(\xi\otimes\xi_0)
- \xi\otimes a\xi_0 \Big\|
< \epsilon \| \xi\otimes\xi_0 \|.
\end{align*}
Thus we can approximate $U(1\otimes\theta_{a\xi_0,\xi_1})$ by
$\sum_i \pi(b_i) \otimes \theta_{a_i\xi_0,\xi_1}$.  We conclude that
$U(\pi(A)\otimes\mc B_0(L^2(\varphi))) \subseteq
\pi(A)\otimes\mc B_0(L^2(\varphi))$.
Hence $U\in M(\pi(A)\otimes\mc B_0(L^2(\varphi)))$.
\end{proof}

\begin{lemma}\label{lem:dense}
We have that for $a,b\in A$,
\[ (\iota\otimes\omega_{a\xi_0,b\xi_0})(U)
= \pi (\iota\otimes\varphi)(\Delta(b^*)(1\otimes a)), \quad
(\iota\otimes\omega_{a\xi_0,b\xi_0})(U^*) = 
\pi  (\iota\otimes\varphi)((1\otimes b^*)\Delta(a)). \]
Consequently, the collections
$\{ (\iota\otimes\omega)(U) : \omega\in \mc B(L^2(\varphi))_* \}$
and $\{ (\iota\otimes\omega)(U^*) : \omega\in \mc B(L^2(\varphi))_* \}$
are dense in $\pi(A)$.
\end{lemma}
\begin{proof}
For $a,b \in A$ and $\xi_1,\xi_2\in K$, we have that
\begin{align*} \big( (\iota\otimes\omega_{a\xi_0,b\xi_0})(U)
   \xi_1 \big| \xi_2 \big) &=
\big( \xi_1\otimes a\xi_0 \big| U^*(\xi_2 \otimes b\xi_0) \big) \\
&= \big((\pi\otimes\iota)\Delta(b^*)\xi_1 \otimes a\xi_0 \big|
   \xi_2\otimes\xi_0 \big)
= \big( \pi(\iota\otimes\varphi)(\Delta(b^*)(1\otimes a)) \xi_1 \big| \xi_2 \big),
\end{align*}
which gives the first result.  Similarly,
\begin{align*} \big( (\iota\otimes\omega_{a\xi_0,b\xi_0})(U^*)
   \xi_1 \big| \xi_2 \big) &=
\big( (\pi\otimes\iota)((1\otimes b^*)\Delta(a))(\xi_1\otimes\xi_0) \big|
   \xi_2\otimes \xi_0 \big),
\end{align*}
which gives the second result.
As $\Delta(A)(1\otimes A)$ is linearly dense in $A\otimes A$, the
density result follows.
\end{proof}

Suppose now that $\pi$ is faithful, so we can identify $A$ with $\pi(A)$,
and so $U$ is a member of $M(A\otimes\mc B_0(L^2(\varphi)))$.

\begin{proposition}\label{prop:regcorep}
Suppose there is a $*$-homomorphism $\Phi:\pi(A)\rightarrow\mc B(K\otimes K)$
with $\Phi\pi = (\pi\otimes\pi)\Delta$.  Then $U_{13} U_{23}
= (\Phi\otimes\iota)U$.
In particular, when $\pi$ is faithful, $U$ is a unitary corepresentation.
\end{proposition}
\begin{proof}
We shall instead equivalently show that $(\Phi\otimes\iota)(U^*)
= U^*_{23} U^*_{13}$.  For $a,b\in A$ and $\xi_1,\xi_2\in K$, we have that
\[ U^*_{13}(\pi(a)\xi_1 \otimes \xi_2 \otimes b\xi_0)
= \big( (\pi\otimes\iota)((\Delta(b)(a\otimes 1)) \big)_{13}
(\xi_1\otimes\xi_2\otimes\xi_0). \]
Similarly,
\[ U^*_{23}(\pi(a_1)\xi_1 \otimes \xi_2 \otimes a_2\xi_0)
= \pi(a_1)\xi_1 (\pi\otimes\iota)\Delta(a_2)(\xi_2\otimes\xi_0)
= (\pi\otimes\pi\otimes\iota)( (\iota\otimes\Delta)(a_1\otimes a_2) )
(\xi_1\otimes\xi_2\otimes\xi_0). \]
As $\Delta(b)(a\otimes 1)\in A\otimes A$, it follows by continuity that
\begin{align*} U^*_{23}U^*_{13}(\pi(a)\xi_1 \otimes \xi_2 \otimes b\xi_0)
&= (\pi\otimes\pi\otimes\iota)( (\iota\otimes\Delta)(\Delta(b)(a\otimes 1) )
(\xi_1\otimes\xi_2\otimes\xi_0) \\
&= (\pi\otimes\pi\otimes\iota)( \Delta^2(b) )
(\pi(a)\xi_1\otimes\xi_2\otimes\xi_0).
\end{align*}
By hypothesis, this is equal to
\[ (\Phi\pi\otimes\iota)\Delta(b)
(\pi(a)\xi_1\otimes\xi_2\otimes\xi_0). \]
It hence follows that for $a,b\in A$,
\[ (\iota\otimes\iota\otimes\omega_{a\xi_0,b\xi_0})(U^*_{23}U^*_{13})
= \Phi\pi\big( (\iota\otimes\varphi)(1\otimes b^*)\Delta(a) \big). \]
By the previous lemma, this is equal to
\[ \Phi( (\iota\otimes\omega_{a\xi_0,b\xi_0})(U^*) ), \]
and the result follows.
\end{proof}


\subsection{Irreducible representations}

\begin{definition}
Let $U\in M(A\otimes\mc B_0(H))$ be a corepresentation of $(A,\Delta)$.
A closed subspace $H_1$ of $H$ is \emph{invariant} for $U$ if $(1\otimes e)
U (1\otimes e) = U (1\otimes e)$ where $e$ is the orthogonal projection onto
$H_1$.

$U$ is said to be \emph{irreducible} if the only invariant subspaces are
$\{0\}$ and $H$.
\end{definition}

\begin{lemma}
Let $H_1$ be an invariant subspace for a corepresentation $U$.  Let $e$ be the
orthogonal projection onto $H_1$, and let $U_e = (1\otimes e)U(1\otimes e)$.
Then $U_e$ is a corepresentation on $H_1$, unitary if $U$ is.
\end{lemma}
\begin{proof}
We have that
\[ (\Delta\otimes\iota)(U_e)
= (1\otimes 1\otimes e)U_{13} U_{23}(1\otimes 1\otimes e)
= (1\otimes 1\otimes e)U_{13} (1\otimes 1\otimes e) U_{23}(1\otimes 1\otimes e)
= (U_e)_{13} (U_e)_{23}. \]
Thus $U_e$ is a corepresentation.  If $U$ is unitary then
\[ U_e^* U_e = (1\otimes e)U^*(1\otimes e)U (1\otimes e)
= (1\otimes e)U^*U (1\otimes e) = 1\otimes e. \]
So $U_e$ is unitary, as a member of $M(A\otimes\mc B_0(H_1))$.
\end{proof}

\begin{definition}
A corepresentation of the form $U_e$ is a \emph{sub-corepresentation} of $U$.
\end{definition}

\begin{proposition}\label{prop:cstar_corep}
Let $U$ be a unitary corepresentation of $(A,\Delta)$.  Let $B$ be the norm closure
of $\{ (\varphi\otimes\iota)(U(a\otimes 1)) : a\in A \}$.  Then $B$ is
a non-degenerate C$^*$-subalgebra of $\mc B(H)$, and $U\in M(A\otimes B)$.
\end{proposition}
\begin{proof}
Let $a\in A$ and set $x = (\varphi\otimes\iota)(U(a\otimes 1)) \in
\mc B(H)$.  Then
\begin{align*}
U(\iota\otimes\varphi\otimes\iota)\big(U_{23}(\Delta(a)\otimes 1)\big)
&= (\iota\otimes\varphi\otimes\iota)
   \big(U_{13}U_{23}(\Delta(a)\otimes 1)\big) \\
&= (\iota\otimes\varphi\otimes\iota)\big((\Delta\otimes\iota)(U(a\otimes 1))
   \big) \\
&= 1 \otimes (\varphi\otimes\iota)(U(a\otimes 1)) = 1\otimes x.
\end{align*}
Thus $U^*(1\otimes x) = (\iota\otimes\varphi\otimes\iota)
(U_{23}(\Delta(a)\otimes 1))$.

So if also $y=(\varphi\otimes\iota)(U(b\otimes 1))$ for some $b\in A$, then
\begin{align*}
y^*x = (\varphi\otimes\iota)\big( (b^*\otimes 1)U^*(1\otimes x) \big)
&= (\varphi\otimes\iota)\big( (b^*\otimes 1)U^*(1\otimes x) \big) \\
&= (\varphi\otimes\varphi\otimes\iota)
((b^*\otimes U)(\Delta(a)\otimes 1)) \\
&= (\varphi\otimes\iota)(U(c\otimes 1)),
\end{align*}
where $c = (\varphi\otimes\iota)( (b^*\otimes 1)\Delta(a) ) \in A$.
So we have shown that $B^* B \subseteq B$.  As $(A\otimes 1)\Delta(A)$ is
dense in $A\otimes A$, as $a$ and $b$ carry, $c$ varies over a dense subset
of $A$.  Thus $B^*B$ is dense in $B$.  In particular, $B$ is self-adjoint.
Thus also $BB\subseteq B$, and we conclude that $B$ is a C$^*$-algebra.

Now let $\theta\in\mc B_0(H)$ and $a\in A$, so that
$(\varphi\otimes\iota)(U(a\otimes \theta)) \in B \mc B_0(H)$.  As $U$ is
a unitary multiplier of $M(A\otimes\mc B_0(H))$, the set
$\{ U(a\otimes \theta) : a\in A, \theta\in\mc B_0(H) \}$ is linearly dense in
$A\otimes \mc B_0(H)$.  It follows that $B \mc B_0(H)$ is linearly dense in
$\mc B_0(H)$, which is enough to show that $B$ acts non-degenerately on $H$.

Finally, we show that $U\in M(A\otimes B)$.  For $b\in A$ and $x$ as above,
\[ U^*(b\otimes x) = (\iota\otimes\varphi\otimes\iota)
(U_{23}(\Delta(a)(b\otimes 1)\otimes 1)). \]
As $\Delta(a)(b\otimes 1)\in A\otimes A$, we see immediately that
$U^*(b\otimes x) \in A\otimes B$.  Moreover, as $\Delta(A)(A\otimes 1)$
is dense in $A\otimes A$, we set $\{ U^*(b\otimes x) : b\in A, x\in B \}$
is linearly dense in $A\otimes B$.  So also $U(A\otimes B) \subseteq
A\otimes B$, and $U\in M(A\otimes B)$ as required.
\end{proof}

\begin{proposition}
Let $U$ be a unitary corepresentation of $(A,\Delta)$, and let $H_1$ be an
invariant subspace of $H$ for $U$.  Then $H_1^\perp$ is also invariant.
\end{proposition}
\begin{proof}
Let $e$ be the orthogonal projection of $H$ onto $H_1$.  Let
$x = (\varphi\otimes\iota)(U(a\otimes 1)) \in B$, so as $U(1\otimes e)
= (1\otimes e)U(1\otimes e)$, it follows that
\[ xe = (\varphi\otimes\iota)(U(a\otimes e))
= (\varphi\otimes\iota)((1\otimes e)U(a\otimes e))
= exe. \]
As $B=B^*$, also $ex = (x^*e)^* = (ex^*e)^* = exe$, and so $ex=xe$.  Thus
$H_1$ is an invariant subspace for $B$, and as $B$ acts non-degenerately on
$H$, it follows that $ex=xe$ for all $x\in M(B)$.\footnote{Indeed, let
$x\in M(B)$ so for $y\in B,\xi\in H$, we have that $xe(y\xi) = (xy)e\xi
= e(xy)\xi = ex(y\xi)$.  By non-degeneracy, it follows that $xe=ex$.}
As $U\in M(A\otimes B)$, it follows that $(1\otimes e)U = U(1\otimes e)$,
and then a short calculation shows that
\[ (1\otimes e^\perp)U(1\otimes e^\perp) = U(1\otimes e^\perp), \]
where $e^\perp = 1-e$, as required.
\end{proof}

\begin{definition}
Let $U_1$ and $U_2$ be unitary corepresentations of $(A,\Delta)$ on 
$H_1$ and $H_2$ respectively.  The \emph{direct sum} of $U_1$ and $U_2$
is $U_1\oplus U_2 \in M(A\otimes\mc B_0(H_1\oplus H_2))$ is
\[ U_1 \oplus U_2 = \begin{pmatrix} U_1 & 0 \\ 0 & U_2 \end{pmatrix}, \]
where here we make the identification
\[ \mc B_0(H_1\oplus H_2) = \begin{pmatrix} \mc B_0(H_1) & \mc B_0(H_2,H_1) \\
\mc B_0(H_1,H_2) & \mc B_0(H_2) \end{pmatrix}. \]

The \emph{tensor product} of $U_1$ and $U_2$ is $U_1 \cotp U_2
= (U_1)_{12} (U_2)_{13} \in M(A \otimes \mc B_0(H_1\otimes H_2))
\cong M(A \otimes \mc B_0(H_1)\otimes\mc B_0(H_2))$.

An \emph{intertwiner} between $U_1$ and $U_2$ is a bounded operator
$T:H_1 \rightarrow H_2$ with $(1\otimes T)U_1 = U_2(1\otimes T)$.
We denote the collection of intertwiners by $\mor(U_1,U_2)$.  Two
corepresentations are \emph{equivalent} if there is an invertible intertwiner,
and unitarily equivalent if there is a unitary intertwiner. 
\end{definition}

\begin{lemma}\label{lem:one}
Let $U$ and $V$ be corepresentations of $(A,\Delta)$ on $H_1$ and $H_2$
respectively.  Let $x\in\mc B(H_1,H_2)$, and set
\[ y = (\varphi\otimes\iota)(V^*(1\otimes x)U). \]
Then $y\in\mc B(H_1,H_2)$, and $V^*(1\otimes y)U = 1\otimes y$.
\end{lemma}
\begin{proof}
We identify $\mc B(H_1,H_2)$ with a ``corner'' of $\mc B(H_1\oplus H_2)$
in the obvious way.  Then $U$ and $V$ are both (on diagonal) corners of $M(A\otimes\mc B_0(H_1\oplus H_2))$; thus $V^*(1\otimes x)U
\in M(A\otimes\mc B_0(H_1\oplus H_2))$ and so $y$ makes sense as a member
of $M(\mc B_0(H_1\oplus H_2)) = \mc B(H_1\oplus H_2)$.  A simple calculation
shows that $y$ only has non-zero component in the $\mc B(H_1,H_2)$ corner;
thus $y$ is well-defined.

Notice that
\[ (\Delta\otimes\iota)(V^*(1\otimes x)U) = V_{23}^* V_{13}^*
(1\otimes 1\otimes x) U_{13} U_{23}. \]
Then observe that
\[ (\varphi\otimes\iota\otimes\iota)(\Delta\otimes\iota)(V^*(1\otimes x)U)
= 1\otimes(\varphi\otimes\iota)(V^*(1\otimes x)U) = 1\otimes y, \]
while
\[ (\varphi\otimes\iota\otimes\iota)V_{23}^* V_{13}^*
(1\otimes 1\otimes x) U_{13} U_{23}
= V^* (1\otimes y) U, \]
and the result follows.
\end{proof}

The obvious use of this lemma is that if $V$ is unitary, then
$(1\otimes y)U = V(1\otimes y)$, and so $y\in\mor(U,V)$.  Notice that an
obvious modification of the proof shows that if $x$ is compact, then
also $y$ will be compact.

\begin{proposition}\label{prop:invsimuni}
Let $U$ be an invertible\footnote{This simply means that there is
some operator $U^{-1}\in M(A\otimes\mc B_0(H))$ with $U^{-1}U=UU^{-1}=1$.}
corepresentation of $(A,\Delta)$.  Then $U$ is equivalent to a unitary
corepresentation.
\end{proposition}
\begin{proof}
Let $U$ act on $H$, and set
\[ y = (\varphi\otimes\iota)(U^*U). \]
By the previous lemma, $U^*(1\otimes y)U = 1\otimes y$.  Clearly $y\geq0$
and as $U$ is invertible, $U^*U\geq\epsilon 1$ for some $\epsilon>0$;
thus also $y\geq \epsilon 1$, so $y$ is invertible.  Now set
\[ V = (1\otimes y^{1/2})U (1\otimes y^{-1/2}). \]
Then $(\Delta\otimes\iota)V = (1\otimes 1\otimes y^{1/2})
U_{13} U_{23} (1\otimes 1\otimes y^{-1/2}) = V_{13} V_{23}$ and so
$V$ is a corepresentation.  Then
\[ V^*V = (1\otimes y^{-1/2})U^*(1\otimes y) U (1\otimes y^{-1/2})
= (1\otimes y^{-1/2}) (1\otimes y)  (1\otimes y^{-1/2}) = 1, \]
and as $V$ is clearly invertible, it follows that $V$ is unitary.
By definition, $y^{1/2}$ intertwines $U$ and $V$, and so $U$ is equivalent
to a unitary corepresentation, as required.
\end{proof}

\begin{theorem}\label{thm:corepdec}
Let $U$ be a unitary corepresentation of $(A,\Delta)$ on a Hilbert space
$H$.  Then there is a family of mutually orthogonal, finite-dimensional
projections $\{ e_\alpha : \alpha\in I \}$ with sum $1$, with
$U(1\otimes e_\alpha) = (1\otimes e_\alpha) U$ for each $\alpha$,
and with $U(1\otimes e_\alpha)$, considered as an element of
$A\otimes \mc B(e_\alpha H)$, being a finite-dimensional unitary
corepresentation.
\end{theorem}
\begin{proof}
Let $B$ be the collection of operators $x\in\mc B(H)$ with $(1\otimes x)U
= U(1\otimes x)$.  Then $B$ is clearly a norm-closed subalgebra, and as
$U$ is unitary, it is easy to see that $B$ is self-adjoint.  So $B$ is a
C$^*$-algebra.  

By Lemma~\ref{lem:one}, if $x\in\mc B_0(H)$ then 
$y=(\varphi\otimes\iota)(U^*(1\otimes x)U)$ will be in $B$, and will be
compact.  Let $(x_i)$ be an increasing net in $\mc B_0(H)$ with supremum $1$.
Then the associated family $(y_i)$ is an increasing net in $B$ with supremum
$1$.  As each $y_i$ is compact, we see that $B$ will contain sufficiently
many finite-rank projections to form the required family $(e_\alpha)$.
\end{proof}

The following is then a quantum Schur's Lemma.

\begin{proposition}\label{prop:schur}
Let $U,V$ be corepresentations of $(A,\Delta)$.  For each $T\in\mor(U,V)$,
the space $\ker T$ is invariant for $U$, and the closure of the image of $T$
is invariant for $V$.  Suppose that one of the following conditions holds:
\begin{enumerate}
\item $U$ and $V$ are irreducible;
\item $U$ or $V$ are finite-dimensional of the same dimension, and one of
$U$ or $V$ is irreducible.
\end{enumerate}
If $U$ and $V$ are not equivalent, then $\mor(U,V)=\{0\}$; otherwise
$\mor(U,V)=\mathbb C x$ for some invertible $x\in\mc B(H_U,H_V)$.  Furthermore,
if $U$ and $V$ are unitary, then $x$ can be chosen to be unitary.
\end{proposition}
\begin{proof}
Let $U$ act on $H_U$, and $V$ act on $H_V$.  Let $T\in\mor(U,V)$.  We first
show that $\ker T$ and $\overline{T(H_U)}$ are invariant for $U$ and $V$
respectively.  Let $e$ be the orthogonal projection onto $\ker T$.  Then
$0 = V(1\otimes Te) = (1\otimes T)U(1\otimes e)$, and it follows that
$(1\otimes e)U(1\otimes e) = U (1\otimes e)$.  Similarly, if $e$ is the
orthogonal projection onto $T(H_U)$, then we wish to show that
$(1\otimes e)V(1\otimes e) = V(1\otimes e)$.  Equivalently, as
$e(H_V)=T(H_U)$, we wish to show that $(1\otimes e)V(1\otimes T) =
V(1\otimes T)$.  However, 
\[ (1\otimes e)V(1\otimes T) = (1\otimes e)(1\otimes T)U
= (1\otimes T)U = V(1\otimes T), \]
as required.

Then, if $U$ and $V$ are both irreducible, we immediately see that any
$T\in\mor(U,V)$ is an isomorphism, or is $0$.
If $U$ is both finite-dimensional and irreducible, then any $T\in\mor(U,V)$
is $0$ or injective, but as $\dim(H_U)=\dim(H_V)<\infty$, then $T$ injective
means that $T$ is an isomorphism.  Similarly, if $V$ is irreducible then $T$ is
either $0$ or surjective (and so an isomorphism).

So in either case, if $U$ and $V$ are not equivalent, then $\mor(U,V)=\{0\}$.
If $T\in\mor(U,V)$ is non-zero, then $U$ and $V$ are equivalent.  If now
$S\in\mor(U,V)$ is also non-zero, then for any
$\lambda\in\mathbb C$, the operator $\lambda T-S$ is in $\mor(U,V)$ and so
is an isomorphism $H_U\rightarrow H_V$, or is $0$.  So choosing $\lambda$
with $\det(\lambda T-S)=0$, we see that actually $\lambda T=S$ as required.

Finally, suppose that $U$ and $V$ are unitary, so as $U = (1\otimes T^{-1}) V
(1\otimes T)$,
\[ 1 = U^*U = (1\otimes T^*)V^*(1\otimes (TT^*)^{-1}) V (1\otimes T), \quad
1 = UU^* = (1\otimes T^{-1}) V^* (1\otimes TT^*) V (1\otimes (T^*)^{-1}). \]
Thus
\[ 1\otimes TT^* = V^*(1\otimes TT^*) V, \]
so as $V$ is unitary, we see that $TT^*\in\mor(V,V)$.  Thus the previous work
shows that $TT^*$ is a (necessarily positive) scalar multiple of the identity.
We may suppose then that $TT^*=I$, so as $T$ is invertible, $T$ is
unitary, as required.
\end{proof}

Now let $\pi:A\rightarrow\mc B(K)$ be a faithful, non-degenerate
$*$-homomorphism and form the regular corepresentation $U$ as in
Proposition~\ref{prop:regcorep}.

\begin{theorem}\label{thm:lrcontains}
Let $U$ be the regular corepresentation, acting on the GNS space $H$.
Let $V$ be an irreducible unitary corepresentation, acting on $H_V$ say.
Then $V$ is equivalent to a sub-corepresentation of (that is,
\emph{contained in}) $U$.
\end{theorem}
\begin{proof}
Let $x\in\mc B_0(H,H_V)$ and set $y=(\varphi\otimes\iota)(V^*(1\otimes x)U)
\in \mc B_0(H,H_V)$ so that $(1\otimes y)U = V(1\otimes y)$, by
Lemma~\ref{lem:one}.  By Proposition~\ref{prop:schur}, if $y$ is non-zero,
then $y$ is surjective.  As $U,V$ are unitary,
\[ V^*(1\otimes y) = (1\otimes y)U^* \implies
(1\otimes y^*) V  =  U (1\otimes y^*) \]
so $y^*:H_V\rightarrow H$ is an intertwiner, and hence $y^*$ is injective,
and the image of $y^*$ is invariant for $U$.  So, if $y$ is non-zero, $y^*$
implements the required equivalence between $V$ and a sub-corepresentation
of $U$.

Alternatively, $y=0$ for all choices of $x$.
So for any $\xi\in H$ and $\eta\in H_V$,
if $x(\gamma) = (\gamma|\xi)\eta$, then
\[ 0 = \big( (\varphi\otimes\iota)(V^*(1\otimes x)U) \xi_1 \big| \eta_1 \big)
= \ip{\varphi}{(\iota\otimes\omega_{\eta,\eta_1})(V^*)
(\iota\otimes\omega_{\xi_1,\xi})(U)}
\qquad (\xi_1\in H, \eta_1\in H_V). \]
By Lemma~\ref{lem:dense}, this means that
\[ \ip{\varphi}{(\iota\otimes\omega)(V^*)a} = 0
\qquad (a\in A, \omega\in\mc B(H_V)_*). \]
This implies that $(\varphi\otimes\iota)(V^*(a\otimes 1))=0$ for all $a\in A$,
and so also $(\varphi\otimes\iota)(V^*(a\otimes x))=0$ for all $a\in A$ and
$x\in\mc B(H_V)$.  As $V$ is irreducible, $H_V$ is finite-dimensional,
and so $V\in A\otimes\mc B(H_V)$.  Thus $(\varphi\otimes\iota)(V^*V)=0$,
which contradicts that $V$ is unitary.
\end{proof}



\subsection{Contragradient representations}

Let $U$ be a finite-dimensional corepresentation of $(A,\Delta)$ on $H$.
Given an orthonormal basis $(e_i)_{i=1}^n$ for $H$ we can let $(e_{ij})$ be
the matrix units of $\mathbb M_n \cong\mc B(H)$.  Then we can write
\[ U = \sum_{i,j=1}^n u_{ij} \otimes e_{ij} \]
for some $u_{ij}\in A$.  Recall from before that $U$ being a corepresentation
is equivalent to $\Delta(u_{ij}) = \sum_k u_{ik} \otimes u_{kj}$.

Let $K$ be another finite-dimensional Hilbert space with orthonormal basis
$(f_j)_{j=1}^m$.  Then $S\in\mc B(H,K)$ can be represented by a matrix in
$\mathbb M_{m,n}$, say $(s_{ij})$.  Then
\[ (1\otimes S)U = \sum_{i,j,p,q} u_{ij} \otimes s_{pq} e_{pq} e_{ij}
= \sum_{i,j,p} s_{pi}u_{ij} \otimes e_{pj}. \]
Similarly, if $V=\sum_{i,j=1}^m v_{ij}\otimes e_{ij}$ is a corepresentation
on $K$, then
\[ V(1\otimes S) = \sum_{i,j,p,q} v_{ij} \otimes e_{ij} s_{pq} e_{pq}
= \sum_{i,j,q} v_{ij} s_{jq} \otimes e_{iq}. \]
Thus $S\in\mor(U,V)$ if and only if, using matrix multiplication,
$(s_{ij})(u_{pq}) = (v_{pq})(s_{ij})$.

\begin{definition}
Given $U$ and $(e_n)$ as above, the \emph{contragradient} corepresentation
is $\overline U = \sum_{i,j} u_{ij}^* \otimes e_{ij}$.
\end{definition}

The definition of $\overline U$ does depend upon $(e_n)$.  Indeed, picking
a new orthonormal basis for $H$ is equivalent to finding a unitary matrix
$S$ and setting $V = (1\otimes S^*)U(1\otimes S)$.  So $V$ is a
corepresentation (unitarily) equivalent to $U$.  Then $\overline V
= (1\otimes \overline{S}^*) \overline U (1\otimes \overline{S})$, and so
$\overline V$ is equivalent to $\overline U$, but the equivalence is given by
the matrix $\overline{S}$, which in general is not equal to $S$.

\begin{lemma}\label{lem:contra_irrep}
Let $U$ be a corepresentation.  Then $\overline{U}$ is also a corepresentation.
If $U$ is irreducible, then so is $\overline{U}$.
\end{lemma}
\begin{proof}
We see that as $\Delta$ is a $*$-homomorphism,
\[ \Delta(u_{ij}^*) = \Delta(u_{ij})^*
= \Big( \sum_k u_{ik}\otimes u_{kj} \Big)^*
= \sum_k u_{ik}^* \otimes u_{kj}^*. \]
So $\overline{U}$ is a corepresentation.

Let $\gamma:\mathbb M_n\rightarrow\mathbb M_n$ be the transpose map, which is
an anti-homomorphism.  Notice that $\overline{U} = (\iota\otimes\gamma)(U^*)$.
Suppose that $e$ is an orthogonal projection on $H$ with
$\overline{U}(1\otimes e) = (1\otimes e) \overline{U} (1\otimes e)$.  Then
applying $\gamma$ gives that
\[ (1\otimes\gamma(e))U^* = (1\otimes\gamma(e))U^*(1\otimes\gamma(e))
\implies U(1\otimes e') = (1\otimes e')U(1\otimes e'), \]
where $e'=\gamma(e)^*$ is still an orthogonal projection.  As $U$ is
irreducible, $e'=0$ or $1$, and hence also $e=0$ or $1$, showing that
$\overline{U}$ is irreducible.
\end{proof}

Notice that $\iota\otimes\gamma$ is not an anti-homomorphism on all of
$\mathbb M_n(A)$, unless $A$ is commutative.  Thus we have to work hard(er)
to prove the next result.

\begin{proposition}\label{prop:conjunitary}
Let $V$ be a finite-dimensional irreducible unitary corepresentation.
Then $\overline{V}$ is equivalent to a unitary corepresentation.
\end{proposition}
\begin{proof}
We again use Lemma~\ref{lem:one}, with $U$ being the left regular
representation, acting on the GNS space $H$.  Let $V$ act on the 
finite-dimensional Hilbert space $H_V$.  Pick $x\in\mc B_0(H,H_V)$ and set
\[ y = (\varphi\otimes\iota)(\overline{V}^*(1\otimes x)U). \]
So $y\in\mc B_0(H,H_V)$ with $\overline{V}^*(1\otimes y)U=1\otimes y$.
Then $U^*(1\otimes y^*)\overline{V} = 1\otimes y^*$ and thus
$(1\otimes y^*)\overline{V} = U(1\otimes y^*)$.  So $y^* \in \mor(\overline{V},
U)$.  By Proposition~\ref{prop:schur}, as $\overline{V}$ is irreducible,
$y^*$ has zero kernel, or $y^*=0$.  As in the proof of
Theorem~\ref{thm:lrcontains}, the image of $y^*$ is an invariant subspace of
$U$, and so either $y=0$, or $y^*$ implements an isomorphism between
$\overline{V}$ and a sub-co-representation of $U$.

Thus, towards a contradiction, suppose that $y=0$ for any choice of $x$.
Again, this implies that
\[ \ip{\varphi}{(\iota\otimes\omega)(\overline{V}^*)a} = 0
\qquad (a\in A, \omega\in\mc B(H_V)_*). \]
Let $\omega\in\mc B(H_V)_*$ be the functional which sends $e_{ij}$ to $1$,
and $e_{pq}$ to $0$ for all other $(p,q)$.  Thus 
$(\iota\otimes\omega)(\overline{V}^*) = v_{ji}$.  We hence see that
\[ \ip{\varphi}{(\iota\otimes\omega)(V)a} = 0
\qquad (a\in A, \omega\in\mc B(H_V)_*). \]
Thus $(\varphi\otimes\iota)(V(a\otimes x))=0$ for $a\in A, x\in\mc B(H_V)$.
This again implies that $(\varphi\otimes\iota)(VV^*)=0$, contradicting
$V$ being unitary.
\end{proof}

In particular, if $V$ is merely an invertible corepresentation, then
$V$ is equivalent to the direct sum of finite-dimensional unitary
corepresentations; the same is then true of $\overline V$, and thus in
particular $\overline V$ is invertible.


\subsection{The Hopf $*$-algebra of matrix elements}

Let $A_0$ be the linear span of the matrix elements\footnote{That is,
elements of the form $(\iota\otimes\omega)(V)$ where $V$ is a unitary
corepresentation, and $\omega\in\mc B(H_V)_*$.} of unitary irreducible
corepresentations.  By the previous work, $A_0$ is also the linear
span of the matrix elements of finite-dimensional invertible
corepresentations.

\begin{proposition}
The space $A_0$ is a dense unital $*$-subalgebra of $A$.
\end{proposition}
\begin{proof}
Let $U$ and $V$ be corepresentations, and let $\omega_U\in\mc B(H_U)_*$
and $\omega_V\in\mc B(H_V)_*$.  Then
\[ (\iota\otimes\omega_U\otimes\omega_V)(U \cotp V)
= (\iota\otimes\omega_U\otimes\omega_V)(U_{12} V_{13})
= (\iota\otimes\omega_U)(U) (\iota\otimes\omega_V)(V). \]
If $U$ and $V$ are finite-dimensional and unitary, then $U\cotp V$ is
also finite-dimensional and unitary.  We conclude that $A_0$ is an algebra.
Notice that $1\in A = M(A\otimes\mathbb C)$ is a unitary corepresentation;
thus $1\in A_0$.

Similarly, as $\overline{U}$ is equivalent to a unitary corepresentation
whenever $U$ is finite-dimensional and unitary, it follows easily that
$A_0$ is closed under the $*$ operation.

It remains to show that $A_0$ is dense in $A$.  Choose a faithful,
non-degenerate $*$-representation $\pi:A\rightarrow\mc B(K)$ and form the
left regular representation $U$ as in Proposition~\ref{prop:regcorep}.
By Theorem~\ref{thm:corepdec}, $U$ decomposes as a direct sum of
finite-dimensional, irreducible unitary corepresentations.  By
Theorem~\ref{thm:lrcontains} every finite-dimensional, irreducible unitary 
corepresentation is equivalent to a sub-corepresentation of $U$.  Hence $A_0$
is the span of the matrix elements of finite-dimensional sub-corepresentations
of $U$.

By Lemma~\ref{lem:dense}, the space $\{ (\iota\otimes\omega)(U) :
\omega\in\mc B(H)_* \}$ is dense in $A$.  Given $\xi,\eta\in H$, we claim
that we can approximate $(\iota\otimes\omega_{\xi,\eta})(U)$ by elements of
$A_0$; this will show that $A_0$ is dense in $A$.  Let $(e_\alpha)$ be a
family of mutually orthogonal projections with sum $1$, as given by
Theorem~\ref{thm:corepdec} when applied to $U$.  Let $U_\alpha
= U(1\otimes e_\alpha) = (1\otimes e_\alpha) U$, a finite-dimensional
unitary corepresentation.  Then
\begin{align*} (\iota\otimes\omega_{\xi,\eta})(U)
&= \sum_\alpha (\iota\otimes\omega_{e_\alpha(\xi),\eta})(U)
= \sum_\alpha (\iota\otimes\omega_{e_\alpha(\xi),\eta})(U(1\otimes e_\alpha))
\\
&= \sum_\alpha (\iota\otimes\omega_{e_\alpha(\xi),e_\alpha(\eta)})
   ((1\otimes e_\alpha)U(1\otimes e_\alpha))
= \sum_\alpha (\iota\otimes\omega_{e_\alpha(\xi),e_\alpha(\eta)})(U_\alpha).
\end{align*}
Thus $(\iota\otimes\omega_{\xi,\eta})(U)$ is in the closure of $A_0$,
as required.
\end{proof}

Let $\{u_\alpha : \alpha\in I\}$ be a maximal family of non-equivalent
unitary corepresentations.  For each $\alpha$, let $u_\alpha \in A
\otimes \mathbb M_{n_\alpha}$ with $u_\alpha = \sum_{i,j=1}^{n_\alpha}
u^\alpha_{ij} \otimes e_{ij}$.  We shall prove that $\{ u^\alpha_{ij}
: \alpha\in I, 1\leq i,j\leq n_\alpha \}$ is a (linear) basis for $A_0$.

We first take a small diversion.  Let $\sigma:A\otimes A\rightarrow A\otimes A$
be the swap map, which is a $*$-homomorphism.  It is easy to see that
$\sigma \Delta$ is co-associative if and only if $\Delta$ is, and so
$(A,\sigma\Delta)$ is a C$^*$-bialgebra (called the ``opposite'' or, less
commonly but more accurately, the ``co-opposite'' quantum group).  We see that
$(A,\Delta)$ satisfies the density conditions to be a compact quantum group
if and only if $(A,\sigma\Delta)$ does.  In this case, $\varphi$ remains the
Haar measure for $(A,\sigma\Delta)$.  Notice however that $U$ is a
corepresentation for $(A,\Delta)$ if and only if $U^*$ is a 
corepresentation for $(A,\sigma\Delta)$.

\begin{proposition}\label{prop:fmatrices}
For each $\alpha\in I$, there is a positive invertible matrix $F^\alpha$
such that
\[ \ip{\varphi}{(u^\beta_{ip})^* u^\alpha_{jq}}
= \delta_{\alpha,\beta} \delta_{p,q} F^\alpha_{j,i}
\qquad (\beta\in I, 1\leq i,p\leq n_\beta, 1\leq j,q\leq n_\alpha). \]
The trace of each matrix $F^\alpha$ is $1$.
\end{proposition}
\begin{proof}
Consider the operator $\theta_{e_i,e_j}\in\mc B_0(\ell^2_{n_\alpha},
\ell^2_{n_\beta})$.  Then by Lemma~\ref{lem:one},
\[ y = (\varphi\otimes\iota)(u_\beta^*(1\otimes x)u_\alpha)
= \sum_{p,b,c,q} \ip{\varphi}{(u^\beta_{bp})^* u^\alpha_{cq}} e_{pb}xe_{cq}
= \sum_{p,q} \ip{\varphi}{(u^\beta_{ip})^* u^\alpha_{jq}} e_{pq} \]
is an operator in $\mc B_0(\ell^2_{n_\alpha},\ell^2_{n_\beta})$ with
$(1\otimes y)u_\alpha = u_\beta(1\otimes y)$.  As $u_\alpha$ and $u_\beta$ are
irreducible, by Proposition~\ref{prop:schur},
we see that $y=0$ if $\alpha\not=\beta$.

When $\alpha=\beta$, by Proposition~\ref{prop:schur}, we see that $y$ must be
a scalar multiple of the identity.  Thus we obtain numbers $F^\alpha_{j,i}$
with $\ip{\varphi}{(u^\beta_{ip})^* u^\alpha_{jq}}
= \delta_{\alpha,\beta} \delta_{p,q} F^\alpha_{j,i}$.  That $u^\alpha$ is
unitary means that
\[ \sum_k (u^\alpha_{k,i})^* u^\alpha_{k,j} = \delta_{i,j}1
\implies \delta_{i,j} = \sum_k \ip{\varphi}{(u^\alpha_{k,i})^* u^\alpha_{k,j}}
= \delta_{i,j} \sum_k F^\alpha_{k,k}. \]

Now consider $y = (\varphi\otimes\iota)(\overline{u^\alpha} 
(\overline{u^\alpha})^* )$.  By Lemma~\ref{lem:one}, applied to
$(A,\sigma\Delta)$, we have that $1\otimes y = \overline{u^\alpha}(1\otimes y)
(\overline{u^\alpha})^*$.  Now,
\[ y = \sum_{i,j,k} \ip{\varphi}{ (\overline{u^\alpha})_{ik}
((\overline{u^\alpha})^*)_{kj}} e_{ij}
= \sum_{i,j,k} \ip{\varphi}{(u^\alpha_{ik})^* u^\alpha_{jk}} e_{ij}
= n_\alpha \sum_{i,j} F^\alpha_{j,i} e_{ij}. \]
Thus $y = n_\alpha (F^\alpha)^t$.  However, clearly $y$ is a positive matrix,
and so $F^\alpha$ is positive.  Now, as $\overline{u^\alpha}$ is equivalent
to a unitary corepresentation, and is hence invertible, we see that $y$
intertwines the corepresentations $(\overline{u^\alpha})^*$ and
$(\overline{u^\alpha})^{-1}$, again working with $(A,\sigma\Delta)$.
Taking adjoints shows that $\overline{u^\alpha}(1\otimes y^*) =
(1\otimes y^*)(\overline{u^\alpha})^{*-1}$.  As $\overline{u^\alpha}$ is
irreducible and has the same dimension as $(\overline{u^\alpha})^{*-1}$,
Proposition~\ref{prop:schur} shows that $y^*=0$ or $y^*$ is an isomorphism.
As the trace of $y$ is $n_\alpha$, we conclude that $y$, and hence also
$F^\alpha$ are invertible.
\end{proof}

\begin{proposition}\label{prop:basis_of_hopf}
The collection $\{ u^\alpha_{i,j} \}$ is linearly independent, and hence forms
a basis for $A_0$.
\end{proposition}
\begin{proof}
Suppose that the finite linear combination $\sum_{\alpha,i,j}
\lambda^\alpha_{i,j} u^\alpha_{i,j}$ is zero.  Then, for any $\beta,p,q$
\[ 0 = \sum_{\alpha,i,j} \lambda^\alpha_{i,j} 
\ip{\varphi}{(u^\beta_{p,q})^* u^\alpha_{i,j}}
= \sum_i F^\beta_{i,p} \lambda^\beta_{i,q}. \]
As $F^\beta$ is invertible, $\lambda^\beta=0$, for any $\beta$, as required.
\end{proof}

Let $m:A_0 \odot A_0\rightarrow A_0$ be the multiplication map on the
algebraic tensor product $A_0\odot A_0$.  We define linear maps $\kappa:
A_0\rightarrow A_0$ and $\epsilon:A_0\rightarrow\mathbb C$ by
\[ \epsilon(u^\alpha_{i,j}) = \delta_{i,j}, \qquad
\kappa(u^\alpha_{i,j}) = (u^\alpha_{j,i})^*
\qquad (\alpha\in I, 1\leq i,j\leq n_\alpha). \]
Notice that then, for any finite-dimensional unitary corepresentation
$U$, we have that
\[ (\kappa\otimes\iota)(U) = U^*, \qquad
(\epsilon\otimes\iota)(U) = 1. \]
In particular, $\kappa$ and $\epsilon$ are well-defined, independent of
our choice of maximal family $\{ u^\alpha \}$.
If $a_i=(\iota\otimes\omega_i)(U_i)$ for $i=1,2$ then
\[ \epsilon(a_1a_2) = (\epsilon\otimes\omega_1\otimes\omega_2)(U_1 \cotp U_2)
= (\omega_1\otimes\omega_2)(1) = \ip{1}{\omega_1} \ip{1}{\omega_2}
= \epsilon(a_1) \epsilon(a_2), \]
and so $\epsilon$ is a character.

\begin{theorem}\label{thm:ishopf}
The maps $\kappa$ and $\epsilon$ turn $(A_0,\Delta)$ into a Hopf $*$-algebra.
To be more precise,
\[ (\epsilon\otimes\iota)\Delta(a) = (\iota\otimes\epsilon)\Delta(a) = a,
\qquad m(\kappa\otimes\iota)\Delta(a) = m(\iota\otimes\kappa)\Delta(a)
= \epsilon(a) 1
\qquad (a\in A_0). \]
Automatically, $\kappa$ is an anti-homomorphism, and $\Delta\kappa
= \sigma(\kappa\otimes\kappa)\Delta$.  Furthermore, $\kappa * \kappa * = \iota$.
\end{theorem}
\begin{proof}
As $\Delta(u^\alpha_{i,j}) = \sum_k u^\alpha_{i,k} \otimes u^\alpha_{k,j}$,
it follows that $\Delta$ restricts to a map $A_0\rightarrow A_0\odot A_0$.
Then
\[ (\epsilon\otimes\iota)\Delta(u^\alpha_{i,j})
= \sum_k \epsilon(u^\alpha_{i,k}) u^\alpha_{k,j}
= u^\alpha_{i,j}, \]
showing that $(\epsilon\otimes\iota)\Delta=\iota$ on $A_0$;
similarly $(\iota\otimes\epsilon)\Delta=\iota$.

Also
\[ m(\kappa\otimes\iota)\Delta(u^\alpha_{i,j})
= \sum_k m((u^\alpha_{k,i})^* \otimes u^\alpha_{k,j})
= \sum_k (u^\alpha_{k,i})^*u^\alpha_{k,j} = \delta_{i,j} 1
= \epsilon(u^\alpha_{i,j}) 1, \]
using that $u^\alpha$ is unitary.  Similarly
$m(\iota\otimes\kappa)\Delta=\epsilon(\cdot) 1$.

That $\kappa$ is an anti-homomorphism and an anti-co-homomorphism follows
from the theory of Hopf algebras, see \cite[Section~1.3.3]{timm} for
example.  That $* \kappa * \kappa = \iota$ follows from our definition of
$\kappa$.
\end{proof}

\begin{proposition}\label{prop:when_corep_units_in_hopf}
Let $\sum_{i,j} a_{ij} \otimes e_{ij}$ be a finite-dimensional corepresentation
of $(A,\Delta)$ with $a_{ij}\in A_0$ for all $i,j$.
Then the following are equivalent:
\begin{enumerate}
\item\label{prop:when_corep_units_in_hopf:one}
The matrix $(a_{ij})$ is invertible;
\item\label{prop:when_corep_units_in_hopf:two}
If $(\xi_j)_{j=1}^n \subseteq \mathbb C$ satisfies that
   $\sum_j a_{ij} \xi_j = 0$ for all $i$, then $\xi=0$.
\item\label{prop:when_corep_units_in_hopf:twoa}
If $(\xi_j)_{j=1}^n \subseteq \mathbb C$ satisfies that
   $\sum_j a_{ij} \xi_i = 0$ for all $j$, then $\xi=0$.
\item\label{prop:when_corep_units_in_hopf:three}
$\epsilon(a_{ij})=\delta_{i,j}$ for all $i,j$;
\item\label{prop:when_corep_units_in_hopf:four}
The matrix $(a_{ij})$ is invertible with inverse $(\kappa(a_{ij}))$.
\end{enumerate}
\end{proposition}
\begin{proof}
Clearly (\ref{prop:when_corep_units_in_hopf:one})$\implies$%
(\ref{prop:when_corep_units_in_hopf:two}) and
(\ref{prop:when_corep_units_in_hopf:one})$\implies$%
(\ref{prop:when_corep_units_in_hopf:twoa}), and
(\ref{prop:when_corep_units_in_hopf:four})$\implies$%
(\ref{prop:when_corep_units_in_hopf:one}).
If (\ref{prop:when_corep_units_in_hopf:two}) then consider the map
$\pi:A_0' \rightarrow \mathbb M_n; \mu\mapsto (\ip{\mu}{a_{ij}})$; here we
write $A_0'$ for the vector space of linear (not necessarily bounded) functionals
$A_0\rightarrow\mathbb C$.  This is
a homomorphism, and so $\pi(\epsilon)$ is a (not necessarily orthogonal)
projection.  If $\pi(\epsilon)\xi=0$ then $\pi(\mu)\xi = \pi(\mu\epsilon)\xi=0$
for all $\mu$.  So $\sum_j \ip{\mu}{a_{ij}} \xi_j = 0$ for all $i$ and $\mu$,
that is, $\sum_j a_{ij}\xi_j=0$.  As (\ref{prop:when_corep_units_in_hopf:two})
holds, $\xi=0$ and so $\pi(\epsilon)=I$ which shows
(\ref{prop:when_corep_units_in_hopf:three}).

Similar, if (\ref{prop:when_corep_units_in_hopf:twoa}) then, if for all
$\mu\in A_0'$ and $(\eta_j)_{j=1}^n\subseteq\mathbb C^n$ we have that
$\sum_{i,j} \eta_j \xi_i \ip{\mu}{a_{ij}} = 0$, then $\xi=0$.  Hence
the linear span of
\[ \big\{ \sum_j \ip{\mu}{a_{ij}} \eta_j : \mu\in A_0', \eta\in\mathbb C^n
\big\} \]
is all of $\mathbb C^n$.  Again, this implies that $\pi(\epsilon)=I$, showing
(\ref{prop:when_corep_units_in_hopf:three}).

By the previous theorem,
if (\ref{prop:when_corep_units_in_hopf:three}) holds then
\[ \sum_k \kappa(a_{ik}) a_{kj} = m(\kappa\otimes\iota)\Delta(a_{ij})
= \epsilon(a_{ij}) 1 = \delta_{i,j}1. \]
Similarly, $\sum_k a_{ik} \kappa(a_{kj}) = \delta_{i,j}1$ and so
(\ref{prop:when_corep_units_in_hopf:four}) holds.
\end{proof}

Notice that the proof shows that condition
(\ref{prop:when_corep_units_in_hopf:twoa}) is equivalent to the homomorphism
$A_0' \rightarrow \mathbb M_n$ being non-degenerate.  Equivalent conditions
are that the induced homomorphisms $A^*\rightarrow\mathbb M_n$ or
$L^1(A)\rightarrow\mathbb M_n$ are non-degenerate.
Theorem~\ref{thm:when_in_poly} below shows that if the Haar state is
faithful on $A$, then any non-degenerate homomorphism $L^1(A)\rightarrow
\mathbb M_n$ arises from an invertible $U$ in this way (that is, the
hypothesis that each $a_{ij}\in A_0$ can be removed).





\subsection{Automorphisms}

We now study the ``$F$-matrices'' $F^\alpha$ more closely.

\begin{proposition}\label{prop:haarotherway}
For $\alpha,\beta\in I$, we have that
\[ \ip{\varphi}{u^\alpha_{ip} (u^\beta_{jq})^*}
= \delta_{\alpha,\beta} \delta_{i,j} \frac{(F^\alpha)^{-1}_{q,p}}
{\tr((F^\alpha)^{-1})}. \]
\end{proposition}
\begin{proof}
Consider the compact quantum group $(A,\sigma\Delta)$.  Then
$\{ (u^\alpha)^* : \alpha\in I\}$ forms a complete set of unitary 
corepresentations for $(A,\sigma\Delta)$.  Thus we can apply
Proposition~\ref{prop:fmatrices} to find positive, invertible, trace $1$
matrices $G^\alpha$ with
\[ \ip{\varphi}{((u^\alpha)^*_{pi})^* (u^\beta)^*_{qj}}
= \ip{\varphi}{u^\alpha_{ip} (u^\beta_{jq})^*}
= \delta_{\alpha,\beta} \delta_{i,j} G^\alpha_{q,p}. \]
The proof of Proposition~\ref{prop:fmatrices} shows that
$1\otimes (F^\alpha)^t = \overline{u^\alpha}(1\otimes (F^\alpha)^t)
(\overline{u^\alpha})^*$ and thus also that $1\otimes (G^\alpha)^t
= (\overline{u^\alpha})^* (1\otimes (G^\alpha)^t) \overline{u^\alpha}$.
Thus both $(F^\alpha)^t$ and $((G^\alpha)^{-1})^t$ intertwine
$\overline{u^\alpha}$ (which is irreducible) and
$((\overline{u^\alpha})^*)^{-1}$ (which is of the same dimension).
Thus Proposition~\ref{prop:schur} shows that
$G^\alpha = \lambda (F^\alpha)^{-1}$ for some $\lambda\in\mathbb C$,
which may be determined by the condition that $G^\alpha$ has trace $1$.
\end{proof}

\begin{lemma}\label{lem:fmatrixkappa}
Let $T\in\mathbb M_n$ be such that $(1\otimes T^{-1})\overline{u^\alpha}
(1\otimes T)$ is unitary.  Then $F^\alpha$ is a scalar multiple of
$\overline{T}T^t$, and $(F^\alpha)^{-1}$ intertwines $u^\alpha$ and the
corepresentation $(\kappa^2(u^\alpha_{ij}))$.
\end{lemma}
\begin{proof}
By Proposition~\ref{prop:conjunitary} there is an invertible $T\in\mathbb M_n$
with $v = (1\otimes T^{-1}) \overline{u^\alpha} (1\otimes T)$ unitary.
Thus
\[ 1 = vv^* = (1\otimes T^{-1}) \overline{u^\alpha} (1\otimes T)
(1\otimes T^*) \overline{u^\alpha}^* (1\otimes (T^{-1})^*), \]
and so $(1\otimes TT^*) = \overline{u^\alpha} (1\otimes TT^*)
\overline{u^\alpha}^*$.  Hence by the proof of Proposition~\ref{prop:fmatrices},
$(F^\alpha)^t$ is a scalar multiple of $TT^*$, or equivalently,
$F^\alpha$ is a scalar multiple of $\overline{T}\overline{T}^*
= \overline{T} T^t$.

Now, as $v$ is unitary, $v^* = \kappa(v)$, where $\kappa(v)$ is the
matrix $(\kappa(v_{ij}))_{i,j=1}^n$.  So
\[ \kappa(v)^t = \overline{v}
= (1\otimes\overline{T}^{-1})u^\alpha(1\otimes\overline{T}). \]
However, also
\[ \kappa(v)^t = \big( (1\otimes T^{-1}) \kappa(\overline{u^\alpha})
(1\otimes T) \big)^t
= (1\otimes T^t) \kappa^2(u^\alpha) (1\otimes (T^{-1})^t), \]
here using that $(\kappa(\overline{u^\alpha})^t)_{i,j} =
\kappa((u^\alpha_{ji})^*) = \kappa^2(u^\alpha_{ij})$.  Thus
\[ (1\otimes (T^{-1})^t \overline{T}^{-1}) u^\alpha
= \kappa^2(u^\alpha) (1\otimes (T^{-1})^t \overline{T}^{-1}). \]
So conclude that $(F^\alpha)^{-1}$ intertwines $u^\alpha$ and
$\kappa^2(u^\alpha)$.
\end{proof}

Notice that a corollary of this result is that $T$ is determined up to a unitary
matrix, and a scalar.  Indeed, by rescaling, we may assume that
$T T^* = \overline F^\alpha$.  As $\overline F^\alpha$ is positive and invertible,
there is a unique unitary%
\footnote{For any vector $x$ we have that $\|T^*x\|^2 = (TT^*x|x)
= (\overline F^\alpha x|x) = \| (\overline F^\alpha)^{1/2} x \|^2$.
So there is a well-defined isometry $U$ with $UT^* = 
(\overline F^\alpha)^{1/2}$.  As $(\overline F^\alpha)^{1/2}$ is invertible,
$U$ is everywhere defined and invertible, so a unitary.  Then $TU^* =
(\overline F^\alpha)^{1/2}$ so $T = (\overline F^\alpha)^{1/2}U$ as required.}
matrix $U$ with $T = (\overline F^\alpha)^{1/2} U$.

\begin{corollary}\label{cor:fmatrixkappa}
The matrix $\overline{ (F^\alpha)^{-1} }$ intertwines the corepresentations
$\overline{u^\alpha}$ and $((u^\alpha)^t)^{-1}$, where of course $(u^\alpha)^t$
has matrix $(u^\alpha_{j,i})$.
\end{corollary}
\begin{proof}
Using the properties of $\kappa$ established in Theorem~\ref{thm:ishopf} we
see that as $u^\alpha$ is unitary, for any $i,j$
\begin{align*} & \sum_k u^\alpha_{i,k} (u^\alpha_{j,k})^*
= \delta_{i,j} 1 = \sum_k (u^\alpha_{k,i})^* u^\alpha_{k,j}
\implies \sum_k u^\alpha_{i,k} \kappa(u^\alpha_{k,j})
= \delta_{i,j} 1 = \sum_k \kappa(u^\alpha_{i,k}) u^\alpha_{k,j} \\
&\implies \sum_k u^\alpha_{k,j} \kappa^{-1}(u^\alpha_{i,k})
= \delta_{i,j} 1 = \sum_k \kappa^{-1}(u^\alpha_{k,j}) u^\alpha_{i,k}.
\end{align*}
This implies that $((u^\alpha)^t)^{-1}$ is the matrix
$(\kappa^{-1}(u^\alpha_{j,i})) = (\kappa((u^\alpha_{j,i})^*)^*)
= (\kappa^2( u^\alpha_{i,j} )^*)$.  By the previous result, for all $i,j$,
\begin{align*}
\sum_k (F^\alpha)^{-1}_{i,k} u^\alpha_{k,j}
= \sum_k \kappa^2(u^\alpha_{i,k}) (F^\alpha)^{-1}_{k,j}
\implies \sum_k \overline{(F^\alpha)^{-1}_{i,k}} (u^\alpha_{k,j})^*
= \sum_k \kappa^2(u^\alpha_{i,k})^* \overline{(F^\alpha)^{-1}_{k,j}},
\end{align*}
that is, $\overline{(F^\alpha)^{-1}}$ intertwines $\overline{u^\alpha}$ and
$((u^\alpha)^t)^{-1}$ as required.
\end{proof}

In particular, this result shows that
\[ (u^\alpha)^t \overline{ (F^\alpha)^{-1} } \overline{u^\alpha}
= \overline{ (F^\alpha)^{-1} }, \qquad 
\overline{u^\alpha} \overline{F^\alpha} (u^\alpha)^t = \overline{ F^\alpha }. \]

Let us think about how the ``$F$-matrices'' are effected by unitary equivalence.
Let $v$ be a unitary corepresentation equivalent to $u^\alpha$, so by
Proposition~\ref{prop:schur}, there is a unitary intertwiner, $X$ say.  Thus
$v = (1\otimes X^*)u^\alpha(1\otimes X)$.  Then
\begin{align*} \ip{\varphi}{v^*_{ip} v_{jq}}
&= \sum \ip{\varphi}{\big( \overline{X_{ai}} u^\alpha_{ab} X_{bp} \big)^*
\overline{X_{cj}} u^\alpha_{cd} X_{dq}}
= \sum X_{ai} \overline{X_{bp}} \overline{X_{cj}} X_{dq}
   \ip{\varphi}{(u^\alpha_{ab})^* u^\alpha_{cd}} \\
&= \sum X_{ai} \overline{X_{bp}} X_{bq} \overline{X_{cj}} F^\alpha_{c,a}
= \delta_{p,q} \big( X^* F^\alpha X \big)_{j,i}.
\end{align*}
Thus the ``$F$-matrix'' associated with $v$ is $X^* F^\alpha X$.

For each $\alpha$, set $t_\alpha = \tr((F^\alpha)^{-1})$.  As $F^\alpha$ is
a positive invertible matrix, $t_\alpha>0$.
For each $z\in\mathbb C$, define a linear map by
\[ f_z:A_0\rightarrow\mathbb C; \qquad
u^\alpha_{i,j} \mapsto ((F^\alpha)^{-z})_{i,j} t_\alpha^{-z/2}. \]
Here we use the functional calculus to define $F^z = \exp(z\log F)$ for
a positive matrix $F$.

Because $(T^*FT)^z = T^* F^z T$ for any positive invertible $F$, unitary $T$
and $z\in\mathbb C$, we see that $f_z$ is well-defined, independent of the
choice of irreducibles $\{ u^\alpha \}$ (of course, $t_\alpha$ is
well-defined).

As is standard, we turn $A^*$ into a Banach algebra, with the product denoted
by $*$, by
\[ \ip{\mu*\lambda}{a} = \ip{\mu\otimes\lambda}{\Delta(a)}
\qquad (a\in A, \mu,\lambda\in A^*). \]
Notice that $*$ is also well-defined on the algebraic dual of $A_0$,
because of Theorem~\ref{thm:ishopf}.  Define $\sigma:A_0\rightarrow A_0$ by
\[ \sigma(a) = f_1* a * f_1 = (f_1 \otimes \iota \otimes f_1)\Delta^2(a)
\qquad (a\in A). \]

\begin{proposition}\label{prop:firstpropsfz}
The maps $f_z$ have the following properties:
\begin{enumerate}
\item For $a\in A_0$, the map $\mathbb C\rightarrow\mathbb C;
z\mapsto f_z(a)$ is entire and of exponential growth in the right half-plane
(meaning that there are $C>0$ and $d\in\mathbb R$ with $|f_z(a)| \leq
C e^{d\re(z)}$ when $\re(z)>0$);
\item $f_0=\epsilon$ the counit, and $f_z * f_w = f_{z+w}$ for all
$z,w\in\mathbb C$;
\item for $a,b\in A_0$, we have that $\ip{\varphi}{ab}
= \ip{\varphi}{b\sigma(a)}$.
\end{enumerate}
\end{proposition}
\begin{proof}
(1) follows almost immediately.  To see this easily, suppose that $F^\alpha$
is diagonal (as we may, as $F^\alpha$ is positive, so diagonalisable).
Then, if $t>0$, the function $z\mapsto t^{-z} = e^{-z\log t}$ is of
exponential growth in the right half-plane, as $|e^{-zs}| = e^{-s\re(z)}$ for
$s\in\mathbb R$.  As any $a\in A_0$ is a finite linear combination of
elements of the form $u^\alpha_{i,j}$ the result follows.

For (ii), first notice that $F^0 = \exp(0) = I$ for any positive matrix $F$,
and so $f_0(u^\alpha_{ij}) = \delta_{i,j}$ as required to show that
$f_0 = \epsilon$.  Now notice that
\begin{align*} \ip{f_z*f_w}{u^\alpha_{ij}}
&= \sum_k \ip{f_z}{u^\alpha_{ik}} \ip{f_w}{u^\alpha_{kj}}
= \sum_k (F^\alpha)^{-z}_{ik}(F^\alpha)^{-w}_{kj}
   t_\alpha^{-z/2} t_\alpha^{-w/2} \\
&= \big( (F^\alpha)^{-z} (F^\alpha)^{-w} \big)_{ij} t_\alpha^{-(z+w)/2}
= (t_\alpha^{1/2} F^\alpha)^{-(z+w)}_{ij}
= \ip{f_{z+w}}{u^\alpha_{ij}}. \end{align*}

For (iii), notice that
\[ \sigma(u^\alpha_{ij}) = \sum_{k,l} \ip{f_1}{u^\alpha_{il}}
u^\alpha_{lk} \ip{f_1}{u^\alpha_{kj}}
= t_\alpha^{-1} \sum_{k,l} (F^\alpha)^{-1}_{il} (F^\alpha)^{-1}_{kj}
u^\alpha_{lk}. \]
Thus, if $a=u^\alpha_{ip}$ and $b=(u^\beta_{jq})^*$, then
\begin{align*} \ip{\varphi}{b \sigma(a)}
&= t_\alpha^{-1} \sum_{k,l} (F^\alpha)^{-1}_{il} (F^\alpha)^{-1}_{kp}
   \ip{\varphi}{(u^\beta_{jq})^*u^\alpha_{lk}}
= t_\alpha^{-1} \delta_{\alpha,\beta} \sum_{l} (F^\alpha)^{-1}_{il}
   (F^\alpha)^{-1}_{qp} F^\alpha_{lj} \\
&= t_\alpha^{-1} \delta_{\alpha,\beta} \delta_{i,j} (F^\alpha)^{-1}_{qp}
= \ip{\varphi}{ab},
\end{align*}
where the final equality uses Proposition~\ref{prop:haarotherway}.
Then (iii) follows by linearity.
\end{proof}

\begin{theorem}
Each $f_z$ is a character on $A_0$.  Furthermore:
\begin{enumerate}
\item $f_z(1)=1$, $f_z(\kappa(a)) = f_{-z}(a)$ and $f_z(a^*)
  = \overline{ f_{-\overline z}(a) }$ for all $a\in A,z\in\mathbb C$;
\item $\kappa^2(a) = (f_1\otimes\iota\otimes f_{-1})\Delta^2(a)$
  for each $a\in A$.
\end{enumerate}
The characters $f_z$ are uniquely determined by the properties shown in
the previous proposition.
\end{theorem}
\begin{proof}
We first claim that $\sigma$ is a character.  For $a,b,c\in A_0$,
\[ \ip{\varphi}{abc} = \ip{\varphi}{c\sigma(ab)}
= \ip{\varphi}{bc\sigma(a)} = \ip{\varphi}{c\sigma(a) \sigma(b)}. \]
As this holds for all $c$, we conclude that $\sigma(ab) = \sigma(a)\sigma(b)$
as required.  Then, for $a\in A_0$,
\[ \ip{f_2}{a} = \ip{f_1 * f_0 * f_1}{a} = \ip{\epsilon}{\sigma(a)}, \]
and so $f_2 = \epsilon\circ\sigma$ is a character.  Then
$f_4 = f_2*f_2 = (f_2\otimes f_2)\circ\Delta$ is a character, as $\Delta$ is
a homomorphism.  Similarly, $f_{2k}$ is a character for all $k\in\mathbb N$.
Thus, for $a,b\in A_0$, the functions
\[ z\mapsto f_z(ab), \quad\text{and}\quad z\mapsto f_z(a)f_z(b) \]
are both entire and of exponential growth in the right-half plane, and are
equal on $\{ 2k : k\in\mathbb N\}$.  Thus they agree everywhere
(see \cite{woro1}{Page~228}).  So $f_z$ is a character for all $z$.

In this argument, we have only used the properties of the family $(f_z)$
established by the previous proposition.  Then $\sigma$ is uniquely
determined by condition (3) (of the previous proposition), and so
$f_2 = \epsilon\circ\sigma$ is uniquely determined.  Thus also $f_{2k}$
is uniquely determined, given condition (2).  But then $(f_z)$ is
uniquely determined by the same complex analysis argument.

Clearly $f_z(1)=1$ for all $z$.  Then
\[ f_z\kappa = (f_z\kappa\otimes\epsilon)\Delta
= (f_z\kappa\otimes f_0)\Delta
= (f_z\kappa\otimes f_z \otimes f_{-z})\Delta^2
= (f_z\otimes f_z\otimes f_{-z})\big( (\kappa\otimes\iota)\Delta
\otimes\iota\big)\Delta. \]
That $f_z$ is a character means that $f_zm = f_z\otimes f_z$, and so
\[ f_z\kappa = 
(f_z\otimes f_{-z})\big( m(\kappa\otimes\iota)\Delta \otimes \iota\big) \Delta
= f_z(1) (\epsilon\otimes f_{-z})\Delta
= f_{-z}, \]
as required.  Notice now that if $t>0$ then $\overline{t^{\overline z}}
= \overline{\exp(\overline{z}\log t)} = \exp(z\log t) = t^z$.  Being careful,
this shows that $(F^{\overline z})^* = F^z$ for a positive invertible
matrix $F$.  Thus
\[ f_z( (u^\alpha_{ij})^* ) = f_z( \kappa(u^\alpha_{ji}))
= f_{-z}(u^\alpha_{ji})
= (F^\alpha)^z_{j,i} t_\alpha^{z/2}
= \overline{ (F^\alpha)^{\overline z}_{i,j} t_\alpha^{\overline z/2} }
= \overline{ f_{\overline z}( u^\alpha_{ij} ) }, \]
which completes showing (1).

By Lemma~\ref{lem:fmatrixkappa}, $(1\otimes (F^\alpha)^{-1})u^\alpha
(1\otimes F^\alpha) = \kappa^2(u^\alpha)$, and so
\[ \kappa^2(u^\alpha_{ij})
= \sum_{k,l} (F^\alpha)^{-1}_{i,k} u^\alpha_{k,l}
F^\alpha_{l,j} t_\alpha^{-1/2} t_\alpha^{1/2}
= (f_1 \otimes \iota \otimes f_{-1})\Delta^2(u^\alpha_{ij}), \]
which shows (2).
\end{proof}

\begin{proposition}
For $z,z'\in\mathbb C$, define a map $\rho_{z,z'}:A_0\rightarrow A_0$ by
\[ \rho_{z,z'} = (f_{z'}\otimes\iota\otimes f_z)\Delta^2. \]
Then $\rho_{z,z'}$ is an algebra homomorphism, and for any $w,w'\in\mathbb C$,
\begin{align*}
\rho_{0,0}&=\iota, & \rho_{z,z'}\circ\rho_{w,w'}&=\rho_{z+w,z'+w'}, \\
\varphi\circ\rho_{z,z'} &= \varphi,  &  \rho_{z,z'}\circ *
  &= *\circ\rho_{-\overline z,-\overline z'} \\
\rho_{z,z'}\circ \kappa &= \kappa\circ\rho_{-z',-z},  &
  \Delta\circ\rho_{z,z'} &= (\rho_{w,z'} \otimes \rho_{z,-w})\circ\Delta, \\
\kappa^{-1} &= \rho_{1,-1}\circ\kappa.
\end{align*}
\end{proposition}
\begin{proof}
These are all immediate from the previous proposition.
\end{proof}

In particular, define two one-parameter families of $*$-homomorphisms of
$A_0$ by
\[ \sigma_t = \rho_{it,it}, \qquad \tau_t = \rho_{-it,it}
\qquad (t\in\mathbb R). \]
These have analytic extensions to all of $\mathbb C$, and we see that
$\sigma = \rho_{1,1} = \sigma_{-i}$ while $\kappa^2 = \rho_{-1,1}
= \tau_{-i}$.  Also $\Delta\tau_t = (\tau_t\otimes\tau_t)\Delta$ and
$\Delta\sigma_t = (\tau_t\otimes\sigma_t)\Delta$.  It follows that
$(\sigma_t)$ is the modular automorphism group of $\varphi$, while
$(\tau_t)$ is the scaling group of $(A,\Delta)$.  Notice that
$\rho_{z,z'} = \sigma_{-i(z+z')/2} \tau_{-i(z'-z)/2}$.




\subsection{Slicing against coreps}

We take a slight diversion, and follow \cite[Section~4]{woro2}.

\begin{proposition}\label{prop:slices}
Let $U\in M(A\otimes\mc B_0(H))$ be a unitary corepresentation, and let
$\omega\in\mc B_0(H)^*$.  Then:
\begin{enumerate}
\item\label{slice:one} Set $a=(\iota\otimes\omega)(U)\in A$.
If $\varphi(aa^*)=0$ then $a=0$.
\item\label{slice:two} $(\iota\otimes\omega)(U)=0$ if and only if
$(\iota\otimes\omega)(U^*)=0$.
\end{enumerate}
For any $a,b\in A$ fixed, we have that
$(\iota\otimes\varphi)(\Delta(b^*)(1\otimes a))=0$ if and only if
$(\iota\otimes\varphi)((1\otimes b^*)\Delta(a))=0$.
\end{proposition}
\begin{proof}
By Proposition~\ref{prop:cstar_corep}, if $B$ is the norm closure of
$\{ (c \varphi\otimes\iota)(U) : c\in A \}$, then $B$ is a non-degenerate
C$^*$-algebra acting on $H$, and $U\in M(A\otimes B)$.  In particular,
we can find $b_0\in B,\omega_0\in\mc B_0(H)^*$ with $\omega=b_0\omega_0$.

For (\ref{slice:one}), for any $c\in A$, we have by Cauchy-Schwarz that
$|\varphi(ac)|^2 \leq \varphi(aa^*) \varphi(c^*b)=0$, and so
$\ip{(c \varphi\otimes\iota)(U)}{\omega} = \ip{c\varphi}{a} = 0$.
Thus $\ip{b}{\omega}=0$ for all $b\in B$.  As $U\in M(A\otimes B)$ we
can find a bounded net $(u_i)$ in $A\otimes B$ with $u_i\rightarrow U$
strictly.  Then
\[ a = (\iota\otimes\omega)(U) = (\iota\otimes\omega_0)(U(1\otimes b_0))
= \lim_i (\iota\otimes\omega_0)(u_i(1\otimes b_0))
= \lim_i (\iota\otimes\omega)(u_i) = 0, \]
as $u_i\in A\otimes B$.

For (\ref{slice:two}), suppose that $(\iota\otimes\omega)(U)=0$.  As just
argued, this certainly implies that $(\iota\otimes\omega)(V)=0$ for any
$V\in M(A\otimes B)$.  In particular, $(\iota\otimes\omega)(U^*)=0$.
Conversely, if $(\iota\otimes\omega)(U^*)=0$ then $(\iota\otimes\omega)(U^*)^*
= (\iota\otimes\omega^*)(U) = 0$, and so $0=(\iota\otimes\omega^*)(U^*)
= (\iota\otimes\omega)(U)^*$ as required.

Finally, follow Section~\ref{sec:leftregcorep},
as applied to some faithful representation of $A$, to form the
left regular corepresentation $U$.  Then Lemma~\ref{lem:dense} combined
with (\ref{slice:two}) gives immediately the final claim.
\end{proof}

\begin{theorem}\label{thm:when_in_poly}
Suppose that $\varphi$ is faithful.  If $a\in A$ with $\Delta(a)$ in
the algebraic tensor product of $A$ with itself, then $a\in A_0$.
\end{theorem}
\begin{proof}
Let $\Delta(a) = \sum_{i=1}^n a_i \otimes b_i$.  For $b\in A$, notice that
\[ (\iota\otimes\varphi)((1\otimes b^*)\Delta(a))
= \sum_{i=1}^n \varphi(b^*b_i) a_i
= \sum_{i=1}^n \ip{b_i\varphi}{b^*} a_i. \]
Thus $(\iota\otimes\varphi)((1\otimes b^*)\Delta(a))=0$ if and only if
$b^* \in \ker(b_1\varphi)\cap\cdots\cap\ker(b_n\varphi)$.  By the previous
proposition, this is equivalent to
$(\iota\otimes\varphi)(\Delta(b^*)(1\otimes a))=0$.  In particular,
we conclude that $\{ (\iota\otimes\varphi)(\Delta(b^*)(1\otimes a)) :
b\in A \}$ is a finite-dimensional subspace of $A$.

Now let $b=u^\alpha_{i,j}$ to see that
\[ \Big\{ \sum_k (u^\alpha_{i,k})^* \varphi((u^\alpha_{k,j})^* a) : \alpha\in I,
1\leq i,j\leq n_\alpha \Big\} \]
is also a finite-dimensional subspace of $A$ (actually, of $A_0$).
As the set $\{ u^\alpha_{i,j} \}$ is a basis for $A_0$, it follows that
there is a finite subset $F\subseteq I$ such that
\[ \varphi((u^\alpha_{k,j})^* a) = 0 \qquad
(\alpha\not\in F, 1\leq j,k\leq n_\alpha). \]

Using Proposition~\ref{prop:fmatrices}, if we set $H_\alpha = \lin\{
u^\alpha_{i,j} \xi_0 : 1\leq i,j\leq n_\alpha \} \subseteq L^2(\varphi)$,
then $L^2(\varphi)$ is the orthogonal direct sum of the finite-dimensional
subspaces $\{ H_\alpha : \alpha\in I \}$.  We have just shown that
$a\xi_0 \in \lin \{ H_\alpha : \alpha\in F \}$.  As $\varphi$ is faithful,
the GNS map $A\rightarrow L^2(\varphi); b\mapsto b\xi_0$ is injective, and
so $a \in \lin \{ u^\alpha_{i,j} : \alpha\in F \} \subseteq A_0$ as required.
\end{proof}

An example given in \cite{ks} shows that this result may fail if $\varphi$
is not faithful.





\subsection{Faithfulness of the Haar state}\label{sec:faith_haar}

\begin{proposition}\label{prop:haarfaithhopf}
The restriction of $\varphi$ to $A_0$ is a faithful state.
\end{proposition}
\begin{proof}
Let $a\in A_0$ with $\ip{\varphi}{a^*a}=0$.  By Cauchy-Schwarz,
$\ip{\varphi}{a^*b}=0$ for all $b\in A_0$.  Thus, if
$a = \sum_{\alpha,i,j} \lambda^\alpha_{i,j} u^\alpha_{i,j}$ a finite
linear combination, then taking $b=u^\beta_{p,q}$ shows that
\[ 0 = \sum_{i,j} \overline{\lambda^\beta_{i,j}}
\delta_{j,q} F^\beta_{p,i} = \sum_i \overline{\lambda^\beta_{i,q}}
F^\beta_{p,i}. \]
Again, as $F^\beta$ is invertible, this shows that $\lambda^\beta=0$ for
all $\beta$, as required.
\end{proof}

\begin{proposition}
For any $a\in A$, we have that $\ip{\varphi}{a^*a}=0$ if and only if
$\ip{\varphi}{aa^*}=0$.  In particular:
\begin{enumerate}
\item $N_\varphi = \{a\in A : \ip{\varphi}{a^*a}=0 \}$ is a two-sided closed
  ideal of $A$;
\item Let $(L^2(\varphi),\pi,\Lambda)$ be the GNS construction for $\varphi$.
  Then $\ker\Lambda = \ker\pi = N_\varphi$.
\end{enumerate}
\end{proposition}
\begin{proof}
Suppose $\ip{\varphi}{a^*a}=0$.  By Cauchy-Schwarz, $\ip{\varphi}{a^*b}=0$
for all $b\in A$, in particular, for all $b\in A_0$.  As $A_0$ is dense
in $A$, we can find a sequence $(a_n)$ in $A_0$ with $a_n^*\rightarrow a^*$
in norm.  So
\begin{align*}
0 &= \ip{\varphi}{a^*\sigma(b)} = \lim_n \ip{\varphi}{a_n^*\sigma(b)}
= \lim_n \ip{\varphi}{ba_n^*} = \ip{\varphi}{ba^*}
\end{align*}
where here we use Proposition~\ref{prop:firstpropsfz}.  As this holds for
all $b\in A_0$, again by density, we conclude that $\ip{\varphi}{aa^*}=0$,
as required.

That $N_\varphi$ is a left ideal follows from the inequality
$a^* x^*x a \leq \|x\|^2 a^*a$; clearly $N_\varphi$ is closed.  However,
we have just shown that $N_\varphi$ is self-adjoint, and hence is a
right ideal as well, showing (1).

By definition,
$\ker\Lambda = N_\varphi$.  Suppose that $\pi(a)=0$, so $0 = \pi(a)\Lambda(1)
= \Lambda(a)$, so $a\in N_\varphi$.  Conversely, if $a\in N_\varphi$ then
for $b\in A$, as $abb^*a^* \leq \|b\|^2 aa^*$ and $\ip{\varphi}{aa^*}=0$,
also $\ip{\varphi}{abb^*a^*}=0$, so also $\ip{\varphi}{b^*a^*ab}=0$, showing
that $\pi(a)\Lambda(b)=0$.  As $b$ was arbitrary, $\pi(a)\xi=0$ for all
$\xi\in H$, showing that $\pi(a)=0$.  Thus (2) holds.
\end{proof}

So we can form the quotient algebra $A_r = A/N_\varphi$, and let
$\varphi_r$ be the functional induced by $\varphi$ on $A_r$; it follows that
$\varphi_r$ is a faithful state on $A_r$.  Let $(L^2(\varphi),\pi,\Lambda)$ be
the GNS construction for $\varphi$ on $A$, and let $(H_r,\pi_r,\Lambda_r)$ be
the GNS construction for $\varphi_r$ on $A_r$.  Let $q:A\rightarrow A_r$
be the quotient map.  By Proposition~\ref{prop:haarfaithhopf}, we see that
$q$ restricts to an injection on $A_0$, and hence we can identify $A_0$
as a dense subalgebra of $A_r$.

\begin{theorem}
The map $\Lambda(a) \mapsto \Lambda_r(q(a))$ extends to an isometric
isomorphism $\theta$ from $L^2(\varphi)$ to $H_r$.  Then $\pi_r(q(a))\theta
= \theta \pi(a)$ for all $a\in A$, and so $\pi(A), \pi(A_r)$ and $A_r$
are all isometrically isomorphic.

There is a unital $*$-homomorphism $\Delta_r:A_r\rightarrow A_r
\otimes A_r$ with $(q\otimes q)\Delta = \Delta_r q$, and such that
$(A_r,\Delta_r)$ becomes a compact quantum group.  $\Delta_r$ restricts
to $\Delta$ on $A_0$.  The corepresentation theory of $(A_r,\Delta_r)$
agrees with that of $(A,\Delta)$.
\end{theorem}
\begin{proof}
As $\ker q = N_\varphi = \ker\Lambda$, the map $\theta$ is well-defined
on $\Lambda(A)$.  Then $\|\theta\Lambda(a)\|^2 = \ip{\varphi_r}{q(a^*a)}
= \ip{\varphi}{a^*a} = \|\Lambda(a)\|^2$, and so $\theta$ is an isometry
with dense range, and hence extends to an isometric isomorphism.
Clearly $\theta$ intertwines $\pi_rq$ and $\pi$, and so we can identify
$\pi(A)$ with $\pi_r(A_r) \cong A_r$.

We now use Proposition~\ref{prop:corepgivescomult}.  Use
$\pi:A\rightarrow\mc B(L^2(\varphi))$ to form $U$, a unitary in
$M(\pi(A)\otimes\mc B_0(L^2(\varphi))) \subseteq
\mc B(L^2(\varphi)\otimes L^2(\varphi))$ with
$(\pi\otimes\pi)\Delta(a) = U^*(1\otimes\pi(a))U$
for $a\in A$.  Using the isomorphism with $H_r$, we obtain a unitary
$W\in\mc M(A_r\otimes\mc B_0(H_r))$ with $W^*(1\otimes q(a))W =
(\pi_rq\otimes\pi_rq)\Delta(a)$ for $a\in A$.  Thus, if $a\in\ker q$, then
$(q\otimes q)\Delta(a)=0$ (as $\pi_r\otimes\pi_r$ injects on $A_r\otimes A_r$).
Then we can set $\Delta_r(a) = W^*(1\otimes a)W$ for $a\in A_r$, and we
see that $\Delta_r(q(a)) = (q\otimes q)\Delta(a)$, as required.

It is clear that $\Delta_r$ agrees with $\Delta$ on $A_0$.  The statement
about corepresentations follows as we can phrase everything in terms of
$A_0$.[\footnote{Should probably be more precise here-- a target would be
to prove: For $V\in M(A\otimes\mc B_0(H))$ a unitary corepresentation of $A$,
clearly $(q\otimes\iota)V$ is a unitary corepresentation of $A_r$; we claim that
this establishes a bijection between unitary corepresentations of $A$ and
of $A_r$.}]
\end{proof}

As an aside, from LCQG theorem, we define
\[ W^*(\Lambda(a)\otimes\Lambda(b))
= (\Lambda\otimes\Lambda)(\Delta(b)(a\otimes 1)). \]
This is the same definition as given by Proposition~\ref{prop:corepgivescomult}.

We can now also construct the von Neumann algebraic version of $A_r$,
as $M = A_r''$ in $\mc B(L^2(\varphi))$.  It is easy to see that we can
extend $\Delta$ to a $M$ by defining $\Delta(x)=W^*(1\otimes x)W$ for $x\in M$
($\sigma$-weak continuity shows that $\Delta$ does map into $M\vnten M$,
and that $\Delta$ is coassociative).  We extend $\varphi$ to $M$ by identifying
it with normal state $\omega_{\Lambda(1)}$.

\begin{lemma}
The extension of $\varphi$ to $M$ is a faithful normal state on $M$.
\end{lemma}
\begin{proof}
We argue above.  If $x\in M$ with $\varphi(x^*x)=0$, then $x\Lambda(1)=0$.
We can find a net $(a_i)$ in $A_0$ which converges strongly on $x$ (by
Kaplansky Density).  Then, for $b,c\in A_0$,
\begin{align*} \big( x\Lambda(\sigma(b)) \big| \Lambda(c) \big)
&= \lim_n \varphi( c^* a_n \sigma(b) )
= \lim_n \varphi( bc^* a_n )
= \lim_n \big( a_n \Lambda(1) \Big| \Lambda(cb^*) \big) \\
&= \big( x \Lambda(1) \Big| \Lambda(cb^*) \big) = 0. \end{align*}
By density, $(x\xi|\eta)=0$ for all $\xi,\eta\in L^2(\varphi)$, so $x=0$.
\end{proof}

\begin{theorem}
Let $x\in M$ with $\Delta(x)$ in the algebraic tensor product of
$M$ with itself.  Then $x\in A_0$.
\end{theorem}
\begin{proof}
We copy the proof of Theorem~\ref{thm:when_in_poly}.
To do so, we need to use a version of Proposition~\ref{prop:slices},
where $a\in M$ in the final claim.  In turn, this follows from a version
of Lemma~\ref{lem:dense}, which in turn follows from the construction of
$W\in\mc B(L^2(\varphi)\otimes L^2(\varphi))$ as $W^*(\xi\otimes \Lambda(a))
= \Delta(a)(\xi\otimes\Lambda(1))$ for $a\in A, \xi\in L^2(\varphi)$.
For $x\in M$, if $(a_n)$ is a net in $A$ converging strongly to $x$,
then $\Delta(x)$ will be the strong limit of $\Delta(a_n)$, and
$\Delta(x) = x\Delta(1)=\lim_n a_n\Delta(1)=\lim_n \Delta(a_n)$ in norm.
Thus $W^*(\xi\otimes \Lambda(x)) = \Delta(x)(\xi\otimes\Lambda(1))$ for all
$x\in M$, and the proof is complete.
\end{proof}




\section{Character theory}

Much of this theory comes from \cite[Section~5]{woro3}.

\begin{definition}
Let $U = (U_{ij})\in A\otimes M_n$ be a (finite-dimensional, unitary)
corepresentation.  Then the \emph{character} of $U$ is the element
$\chi(U) = \chi_U = \sum_{i=1}^n U_{ii} \in A$.
\end{definition}

If $\tr$ denotes the (non-normalised) trace, then
$\chi_U = (\iota\otimes\tr)U$, showing $\chi_U$ to be coordinate independent.

\begin{lemma}
Let $U,V$ be corepresentations of $A$.  Then $\chi(U\oplus V) = \chi(U) +
\chi(V), \chi(U\cotp V) = \chi(U)\chi(V)$, $\chi(\overline{U}) = \chi(U)^* =
\kappa(\chi(U))$.
If $U$ and $V$ are equivalent of dimension $n$, then $\chi(U)=\chi(V)$
and $\epsilon(\chi(U)) = n$.
\end{lemma}
\begin{proof}
We only prove the non-obvious claims.  We may suppose that $U$ is unitary, so
then $\chi(\overline U) = \sum_i U_{ii}^* = \sum_i \kappa(U_{ii})$ and
so $\chi(\overline U) = \chi(U)^* = \kappa(\chi(U))$.  Similarly,
$\epsilon(\chi(U)) = \sum_i \epsilon(U_{ii}) = n$.
\end{proof}

\begin{proposition}\label{prop:char_are_on}
If $U,V$ are irreducible (unitary) corepresentations, then
$\varphi(\chi_U^*\chi_V) = \varphi(\chi_U\chi_V^*) = 1$ if $U$ is equivalent
to $V$, and equals $0$ otherwise.
\end{proposition}
\begin{proof}
This follows immediately from Proposition~\ref{prop:fmatrices}
and Proposition~\ref{prop:haarotherway}.
\end{proof}

Then, as for classical compact groups, knowing $\chi_U$ allows us to find
how $U$ is decomposed as irreducibles.  To be precise, if we set
$n_\alpha = \varphi(\chi_{u_\alpha}^* \chi_U)$, then
\[ U \cong \bigoplus_\alpha (u^\alpha)^{\oplus n_\alpha}, \quad
\chi_U = \sum_\alpha n_\alpha \chi(u^\alpha). \]
Furthermore, the space of intertwiners between $U$ and itself has
dimension $\sum_\alpha n_\alpha^2 = \varphi(\chi_U^*\chi_U)$.

\begin{lemma}
Assume diagonalised F-matrices.\footnote{Maybe we don't need to do this--
but then we need to define the ``quantum-dimension'' somewhere!}
Then $f_1(\chi_U) = f_{-1}(\chi_U) = \Lambda_\alpha$.
\end{lemma}
\begin{proof}
Simply note that $f_z(\chi_U) = \sum_i (\lambda^\alpha_i)^z$ and so
$f_1(\chi_U) = f_{-1}(\chi_U) = \Lambda_\alpha$.
\end{proof}

Notice that
\[ \Delta(\chi_U) = \sum_i \Delta(U_{ii})
= \sum_{i,j} U_{ij} \otimes U_{ji}, \]
and so $\Delta(\chi_U) = \sigma \Delta(\chi_U)$.  Woronowicz says that
this corresponds to the classical situation where characters are always
invariant under inner-automorphisms.\footnote{Can we expand?}

\subsection{Woronowicz's question}

\newcommand{\cen}{\operatorname{cen}}

Let $A_{\cen} = \{ a\in A : \Delta(a)=\sigma\Delta(a) \}$ and
$A^0_{\cen} = A_0 \cap A_{\cen}$.

\begin{lemma}
Let $a\in A^0_{\cen}$.  Then $a$ is a finite linear combination of characters.
\end{lemma}
\begin{proof}
As $a\in A_0$, we can write $a=\sum a_{\alpha,i,j} u^\alpha_{ij}$.
Then
\[ \Delta(a) = \sum a_{\alpha,i,j} u^\alpha_{ik} \otimes u^\alpha_{kj}
= \sigma\Delta(a)
= \sum a_{\alpha,i,j} u^\alpha_{kj} \otimes u^\alpha_{ik}. \]
Then for all $\beta,p,q$,
\[ \sum_i a_{\beta,i,q} u^\beta_{ip}
= \sum_{j} a_{\beta,p,j} u^\beta_{qj}. \]
But then looking at the $u^\gamma_{r,s}$ component shows that
for all $\gamma,p,q,r,s$ we have that
\[ a_{\gamma,r,q} \delta_{s,p} = a_{\gamma,p,s} \delta_{r,q}. \]
So if $s\not=p$ then $a_{\gamma,p,s}=0$, while taking $r=q$ and $s=p$ shows
that $a_{\gamma,r,r} = a_{\gamma,s,s}$ for all $r,s$.  So there
are scalars $b_\alpha$ such that $a_{\alpha,i,j} = \delta_{i,j} b_\alpha$.
Hence
\[ a = \sum_\alpha b_\alpha \sum_i u^\alpha_{ii}
= \sum_\alpha b_\alpha \chi(u^\alpha), \]
as required.
\end{proof}

Woronowicz asked:
\begin{itemize}
\item Is $A^0_{\cen}$ dense in $A_{\cen}$?
\item Equivalently, is the span of characters dense in $A_{\cen}$.
\end{itemize}

Again, if we believe that when $A=C(G)$ then $A_{\cen}$ is the space
of functions invariant under inner-automorphisms (i.e. the space of
``class functions'') then this is true in the classical group case.





\section{Diagonalisation}\label{sec:diag}

Recall (from Proposition~\ref{prop:fmatrices}) that the F-matrices satisfy
\[ \ip{\varphi}{(u^\beta_{ip})^* u^\alpha_{jq}}
= \delta_{\alpha,\beta} \delta_{p,q} F^\alpha_{j,i}
\qquad (\alpha,\beta\in I, 1\leq i,p\leq n_\beta, 1\leq j,q\leq n_\alpha), \]
where $F^\alpha$ is a positive invertible matrix with trace $1$.

Then we can find a unitary matrix $X^\alpha$ such that $(X^\alpha)^* F^\alpha
X^\alpha$ is diagonal, say with diagonal entries $(\mu^{(\alpha)}_i)
\subseteq (0,1]$, with $\sum_i \mu^{(\alpha)}_i = 1$.

Set $v^\alpha = (X^\alpha)^* u^\alpha X^\alpha$, a unitary corepresentation
(unitarily) equivalent to $u^\alpha$.  Then
\begin{align*} \ip{\varphi}{(v^\beta_{i,p})^* v^\alpha_{j,q}}
&= \sum_{a,b,c,d} \ip{\varphi}{( (X^\beta)^*_{i,a} u^\beta_{a,b} X^\beta_{b,p} )^*
(X^\alpha_{j,c})^*  u^\alpha_{c,d} X^\alpha_{d,q}} \\
&= \delta_{\alpha,\beta} \sum_{a,b,c,d} X^\alpha_{a,i} \overline{ X^\alpha_{b,p} }
\overline{ X^\alpha_{c,j} } X^\alpha_{d,q} \ip{\varphi}{(u^\alpha_{a,b})^*
u^\alpha_{c,d}}
= \delta_{\alpha,\beta} \sum_{a,b,c,d} X^\alpha_{a,i} \overline{ X^\alpha_{b,p} }
\overline{ X^\alpha_{c,j} } X^\alpha_{d,q} \delta_{b,d} F^\alpha_{c,a} \\
&= \delta_{\alpha,\beta} \sum_{a,c} X^\alpha_{a,i}
\overline{ X^\alpha_{c,j} } ((X^\alpha)^* X^\alpha)_{p,q} F^\alpha_{c,a}
= \delta_{\alpha,\beta} \delta_{p,q} ((X^\alpha)^* F^\alpha X^\alpha)_{j,i} \\
&= \delta_{\alpha,\beta} \delta_{p,q} \delta_{i,j} \mu^{(\alpha)}_i.
\end{align*}

We now use Proposition~\ref{prop:haarotherway}.  First note that
$(X^\alpha)^*(F^\alpha)^{-1}X^\alpha$ is diagonal with entries $(\mu_i^{-1})$.
As before, set $t_\alpha = \tr((F^\alpha)^{-1}) = \sum_i \mu_i^{-1}$.
So we see that
\begin{align*} \ip{\varphi}{v^\beta_{i,p} (v^\alpha_{j,q})^*}
&= \sum_{a,b,c,d} \ip{\varphi}{ (X^\beta)^*_{i,a} u^\beta_{a,b} X^\beta_{b,p} 
((X^\alpha_{j,c})^*  u^\alpha_{c,d} X^\alpha_{d,q})^*} \\
&= \delta_{\alpha,\beta} \sum_{a,b,c,d} (X^\alpha)^*_{i,a} X^\alpha_{b,p}
X^\alpha_{c,j} \overline{ X^\alpha_{d,q} } \delta_{a,c}
\frac{ (F^\alpha)^{-1}_{d,b} }{t_\alpha} \\
&= \delta_{\alpha,\beta} \delta_{i,j} \sum_{b,d}  X^\alpha_{b,p}
(X^\alpha)^*_{q,d} \frac{ (F^\alpha)^{-1}_{d,b} }{t_\alpha} \\
&= \delta_{\alpha,\beta} \delta_{i,j}
\frac{ ((X^\alpha)^*(F^\alpha)^{-1}X^\alpha)_{q,p}}{t_\alpha}
= \delta_{\alpha,\beta} \delta_{i,j} \delta_{p,q} (\mu_p^\alpha)^{-1}
t_\alpha^{-1}.  \end{align*}

Let $\lambda^\alpha_i = (\mu^\alpha_i)^{-1} t_\alpha^{-1/2}$, so that
\[ \sum_i (\lambda^\alpha_i)^{-1} = (t^\alpha)^{1/2}, \quad
\sum_i \lambda^\alpha_i = (t^\alpha)^{-1/2} t_\alpha = (t^\alpha)^{1/2}. \]
So with $\Lambda_\alpha = (t^\alpha)^{1/2}$, we see that
\[ \ip{\varphi}{(v^\beta_{i,p})^* v^\alpha_{j,q}}
= \delta_{\alpha,\beta} \delta_{p,q} \delta_{i,j}
\frac{1}{\lambda^{\alpha}_i \Lambda_\alpha}, \qquad
\ip{\varphi}{v^\beta_{i,p} (v^\alpha_{j,q})^*}
= \delta_{\alpha,\beta} \delta_{i,j} \delta_{p,q}
\frac{\lambda^\alpha_p}{\Lambda_\alpha}. \]

Thus, to recap, for the new family of unitary corepresentations $(v^\alpha)$,
the associated ``$F$-matrices'' are diagonal, with entries $(\mu^{(\alpha)}_i)$
or equivalently, with entries $( (\lambda^{\alpha}_i)^{-1} \Lambda_\alpha^{-1} )$.

Thus this does agree with my PAMS paper.

Notice that Lemma~\ref{lem:fmatrixkappa} shows that
\[ \frac{\delta_{i,j}}{\lambda^\alpha_i \Lambda_\alpha}
= \sum_{k,l} (\overline{v^\alpha})_{i,k}
\frac{\delta_{k,l}}{\lambda^\alpha_k \Lambda_\alpha}
(\overline{v^\alpha})^*_{l,j}
= \sum_k (v^\alpha_{i,k})^* v^\alpha_{j,k}
\frac{1}{\lambda^\alpha_k \Lambda_\alpha}. \]


\subsection{Decomposing the left-regular corepresentation}

[\footnote{This is just a variant of the construction at the start,
but where now we don't work with the \emph{reduced} version of $A$.}]

Form the left-regular corepresentation $U$ as in
Proposition~\ref{prop:regcorep}, so that
$U\in M(A\otimes\mc B_0(L^2(\varphi))$.  Recall that $L^2(\varphi)$ is the
GNS space for $\varphi$, with cyclic vector $\xi_0$.  As at the start,
$L^2(\varphi)$ decomposes as the orthogonal direct sum $L^2(\varphi) =
\bigoplus_\alpha H_\alpha$ where $H_\alpha$ is the span of the vectors
$(v^\alpha_{ij})^*\xi_0$.  There is then a unitary
\[ U_\alpha : H_\alpha \rightarrow \ell^2_{n_\alpha} \otimes\ell^2_{n_\alpha};
\quad (v^\alpha_{ij})^*\xi_0 \mapsto \sqrt\frac{\lambda^\alpha_j}{\Lambda_\alpha}
\delta_i \otimes \delta_j. \]
Let $X=\bigoplus_\alpha U_\alpha: L^2(\varphi) \rightarrow \bigoplus_\alpha
\ell^2_{n_\alpha} \otimes\ell^2_{n_\alpha}$.  Then as before,
\begin{align*}
& (1\otimes X)U^*(1\otimes X^*)\big( \xi\otimes\delta^\alpha_i\otimes
\delta^\alpha_j \big) =
\sqrt\frac{\Lambda_\alpha}{\lambda^\alpha_j}
  (1\otimes X)U^*(\xi\otimes (v^\alpha_{ij})^*\xi_0) \\
&= \sqrt\frac{\Lambda_\alpha}{\lambda^\alpha_j}
  \sum_k (1\otimes X) \big( (v^\alpha_{ik})^*\xi \otimes
  (v^\alpha_{kj})^*\xi_0 \big)
= \sum_k (v^\alpha_{ik})^*\xi \otimes \delta^\alpha_k \otimes
  \delta^\alpha_j.
\end{align*}
It follows that
\[ (1\otimes X)U^*(1\otimes X^*)
= \sum_{\alpha,i,k} (v^\alpha_{ik})^* \otimes e_{ki}^\alpha \otimes 1, \]
and so
\[ (1\otimes X)U(1\otimes X^*)
= \sum_{\alpha,i,k} v^\alpha_{ik} \otimes e_{ik}^\alpha \otimes 1. \]
Hence $(1\otimes X)U(1\otimes X^*)$ decomposes as $(v^\alpha)$ where
each $v^\alpha \in M_n(A) = A \otimes M_n$ acts on the first component
of $\ell^2_{n_\alpha} \otimes \ell^2_{n_\alpha}$.


\subsection{The right regular representation}

Again, let $(A,\sigma\Delta)$ be the opposite quantum group.  Then $\varphi$
remains the Haar weight for $(A,\sigma\Delta)$, and so we can form the regular 
representation $U^\op$ for $(A,\sigma\Delta)$, acting on $L^2(\varphi)$.
It is easy to see that $Y$ is a (unitary) corepresentation of $(A,\Delta)$ if
and only if $Y^*$ is a
(unitary) corepresentation of $(A,\sigma\Delta)$.  Set $V=(U^\op)^*$, the
\emph{right regular representation} of $(A,\Delta)$.  By definition,
\[ V(\xi\otimes a\xi_0) = \sigma\Delta(a)(\xi\otimes\xi_0). \]

Thus we find that
\begin{align*}
(1\otimes X)V(1\otimes X^*) & (\xi\otimes\delta^\alpha_i\otimes\delta^\alpha_j)
= \sqrt{\frac{\Lambda_\alpha}{\lambda^\alpha_j}} (1\otimes X)V(\xi
   \otimes (v^\alpha_{ij})^*\xi_0) \\
&= \sqrt{\frac{\Lambda_\alpha}{\lambda^\alpha_j}} (1\otimes X) \sum_k
   (v^\alpha_{kj})^*\xi \otimes (v^\alpha_{ik})^*\xi_0
= \sum_k (v^\alpha_{kj})^*\xi \otimes \delta^\alpha_i \otimes
   \sqrt\frac{\lambda^\alpha_k}{\lambda^\alpha_j} \delta^\alpha_k.
\end{align*}
Hence we see that
\begin{align*} (1\otimes X)V(1\otimes X^*)
&= \sum_{\alpha,j,k} (v^\alpha_{kj})^* \otimes 1 \otimes 
\sqrt\frac{\lambda^\alpha_k}{\lambda^\alpha_j} e_{kj}^\alpha
= \sum_{\alpha,j,k} (\tau_{-i/2}(v^\alpha_{kj}))^* \otimes 1
   \otimes e_{kj}^\alpha \\
&= \sum_{\alpha,j,k} R(v^\alpha_{jk}) \otimes 1 \otimes e_{kj}^\alpha.
\end{align*}


\subsection{Products of compact quantum groups}

Let $(A,\Delta_A)$ and $(B,\Delta_B)$ be compact quantum groups, with Haar
states $\varphi_A$ and $\varphi_B$.  We form a coproduct $\Delta$ on
$A\otimes B$ by $\Delta = (1\otimes\sigma\otimes 1)(\Delta_A\otimes\Delta_B)$.
This is clearly a map $A\otimes B \rightarrow (A\otimes B)\otimes(A\otimes B)$.
A tedious but easy calculation shows that this is cocommutative.  We call
$(A\otimes B,\Delta)$ the \emph{product} of $A$ and $B$.

Let $U$ be a corepresentation of $A$, and $V$ be a corepresentation of $B$,
both acting on the same space $H$.  We shall say that $U$ and $V$
\emph{commute} if $U_{13} V_{23} = V_{23} U_{13}$.  Under this assumption, if
we set $X=U_{13} V_{23} = U \times V \in M(A\otimes B\otimes \mc B_0(H))$, then
\begin{align*} (\Delta\otimes\iota)X
&= (\iota\otimes\sigma\otimes\iota\otimes\iota)
\big( (\Delta_A\otimes\iota)(U)_{125} (\Delta_B\otimes\iota)(V)_{345} \big)
= (\iota\otimes\sigma\otimes\iota\otimes\iota)
   \big( U_{15} U_{25} V_{35} V_{45} \big) \\
&= U_{15} U_{35} V_{25} V_{45}
= U_{15} V_{25} U_{35} V_{45}
= X_{13} X_{23}. \end{align*}
Hence $X$ is a corepresentation of $A\otimes B$.

In particular, set $B=A$ and let $U,V$ be the left (respectively, right) regular
representations.  Thanks to the previous calculations, we see that $U$ and $V$
commute.  Furthermore, by taking suitable $\mu\in A^*\odot A^* \subseteq
(A\otimes A)^*$, we have
\[ (\mu\otimes\iota)(U_{13} V_{23}) = e^\alpha_{ik}\otimes e^\alpha_{jl} \]
for any $\alpha,i,j,k,l$.  Hence $U_{13} V_{23}$ is irreducible.  This is in
some sense the analogue of the classical Peter-Weyl theorem.

\begin{itemize}
\item Can we show that every irrep of $A\times A$ occurs in this way?
\end{itemize}


\subsection{``Central'' elements}

In a similar manner, we can show that $UV$ (or $VU$) is a unitary
corepresentation of $(A,\Delta)$; indeed
\[ (\Delta\otimes\iota)(UV) = U_{13} U_{23} V_{13} V_{23}
= U_{13} V_{13} U_{23} V_{23} = (UV)_{13} (UV)_{23}. \]
We shall say that $\eta\in L^2(\varphi)$ is \emph{central} or \emph{invariant}
if $(UV)(\xi\otimes\eta)=\xi\otimes\eta$ for all $\xi$.  It is easy to
see that this is equivalent to
\[ (\mu\otimes\iota)(UV) \eta = \mu(1) \eta
\qquad (\mu\in A^*), \]
which also shows that the original definition is independent of the chosen
faithful representation of $A$.

\begin{lemma}
The operator $p=(\varphi\otimes\iota)(UV)$ is a projection, and $\eta\in
L^2(\varphi)$ is central if and only if $p\eta=\eta$.
\end{lemma}
\begin{proof}
Let $X$ be any (unitary) corepresentation of $A$, and for now, let
$p=(\varphi\otimes\iota)X$.
Applying $\varphi\otimes\iota\otimes\iota$ to the relation
$(\Delta\otimes\iota)(X) = X_{13} X_{23}$ shows that $(1\otimes p)X =
1\otimes p$.  Similarly, applying $\iota\otimes\varphi\otimes\iota$ yields
that $X(1\otimes p) = 1\otimes p$.  Then, applying $\varphi\otimes\iota$
gives that $p^2=p$.  Finally, as $\varphi$ is a state and $\|X\|\leq 1$,
it follows that $\|p\|\leq 1$, and so $p$ must be an orthogonal projection.

Now say that $\eta$ is invariant for $X$ if $(\mu\otimes\iota)(X) \eta
= \mu(1) \eta$ for all $\mu\in A^*$.  It follows immediately that if
$\eta$ is invariant, then $p\eta=\eta$.  Conversely, if $p\eta=\eta$ then
\[ \xi\otimes \eta = (1\otimes p)(\xi\otimes\eta) = X
(1\otimes p)(\xi\otimes\eta) = X(\xi\otimes\eta), \]
and so $\eta$ is invariant.

The lemma now follows from the special case $X=UV$.
\end{proof}

\begin{itemize}
\item What happens if we instead use $VU$?
\end{itemize}

Dropping now the isomorphism $X$, we see that
\begin{align*} p &= (\varphi\otimes\iota)(UV) = \sum \varphi(v^\alpha_{ik}
(v^\alpha_{lj})^*) e^\alpha_{ik} \otimes e^\alpha_{lj}
\sqrt\frac{\lambda^\alpha_l}{\lambda^\alpha_j}
= \sum_{\alpha,i,j}
\frac{\sqrt{\lambda^\alpha_i \lambda^\alpha_j}}{\Lambda_\alpha}
e^\alpha_{ij} \otimes e^\alpha_{ij} \\
&= \sum_\alpha \sum_{i,j} \sqrt\frac{\lambda^\alpha_i}{\Lambda_\alpha}
\sqrt\frac{\lambda^\alpha_j}{\Lambda_\alpha}
\theta_{\delta^\alpha_i\otimes\delta^\alpha_i,
\delta^\alpha_j\otimes\delta^\alpha_j}
= \sum_\alpha \theta_{e_\alpha,e_\alpha},
\end{align*}
say, where $e_\alpha = \sum_i \sqrt\frac{\lambda^\alpha_i}{\Lambda_\alpha}
\delta^\alpha_i\otimes\delta^\alpha_i$.  Here we use the obvious isomorphism
$\mathbb M_{n_\alpha} \otimes \mathbb M_{n_\alpha}
\cong \mathbb M_{n_\alpha\times n_\alpha}$.  Notice that actually
$e_\alpha = X(\chi_\alpha^*\xi_0)$ where $\chi_\alpha$ is the character
of $v^\alpha$.  It immediately follows that $p(e_\alpha) = e_\alpha$ for
each $\alpha$.  Less obvious in this picture is that $X(\chi_\alpha\xi_0)$ is
also invariant.  We can prove this by observing that
\begin{align*} V(\xi\otimes\chi_\alpha\xi_0)
= \sum_i \sigma\Delta(v^\alpha_{ii}) (\xi\otimes\xi_0)
= \sum_{ij} v^\alpha_{ji}\xi \otimes v^\alpha_{ij}\xi_0
= \sum_j \Delta(v^\alpha_{jj})(\xi \otimes \xi_0)
= U^*(\xi\otimes\chi_\alpha\xi_0).
\end{align*}
Hence $UV(\xi\otimes\xi_0) = \xi\otimes\xi_0$, which is true for any $\xi$,
showing that $\xi_0$ is invariant.

\begin{corollary}
The family $(e_\alpha)$ is an orthonormal basis for the subspace of
central vectors in $L^2(\varphi)$.
\end{corollary}
\begin{proof}
From Proposition~\ref{prop:char_are_on} we know that $\varphi(\chi_\alpha
\chi_\beta^*) = \delta_{\alpha,\beta}$, showing that $(\chi_\alpha^*\xi_0)
= (e_\alpha)$ is an orthonormal set.  The result now follows given the
form of $p$ established above.
\end{proof}



\subsubsection{Actions}

In the commutative case, we can consider the action of $G$ on itself
given by $s\cdot t = sts^{-1}$.  This gives a coaction $\alpha:
C(G) \rightarrow C(G\times G)$ given by $\alpha(f)(s,t) = f(sts^{-1})$.
This is a left coaction, as
\[ (\iota\otimes\alpha)\alpha(f)(s,t,r) = \alpha(f)(s,trt^{-1})
= f(strt^{-1}s^{-1}) = (\Delta\otimes\iota)\alpha(f)(s,t,r). \]
First observe that $V\xi(s,t) = \xi(s,ts)$ for $\xi\in L^2(G\times G)$.
Hence
\begin{align*} V^*U^*(1\otimes f)UV\xi(s,t) = V^*\Delta(f)V\xi(s,t)
= \Delta(f)V\xi(s,ts^{-1}) = f(sts^{-1}) V\xi(s,ts^{-1})
= \alpha(f)(s,t) \xi(s,t), \end{align*}
and so $V^*U^*(1\otimes f)UV=\alpha(f)$.

However, in the compact quantum group case, this doesn't work, because
in general $V^*\Delta(v^\alpha_{ij})V \in M(A\otimes\mc B_0(L^2(\varphi)))$
is not in $A\otimes A$.
\textbf{How to show this?  Is it true in the Kac case?}


\subsection{Convolution product}

We identify a dense subspace of $L^1(A)$ with a (dense) subspace of $A$ by
saying that $\omega\in L^1(A)$ corresponds to $a\in A$ when
$\hat\Lambda(\lambda(\omega)) = a\xi_0$ in $L^2(\varphi)$.  This is
equivalent to
\[ \big( a\xi_0 \big| b\xi_0 \big) = \varphi(b^*a)
= \ip{\varphi}{b^*a} = \ip{b^*}{a\varphi}
= \big( \hat\Lambda(\lambda(\omega)) \big| b\xi_0 \big)
= \ip{b^*}{\omega} \qquad (b\in A). \]
That is, if and only if $a\varphi = \omega$.
Then, given $a,b\in A$ we define the \emph{convolution product} $a*b$ to be
(if it exists) the element $c$ of $A$ which corresponds to
$(a\varphi)*(b\varphi)\in L^1(A)$, that is, $c\varphi = (a\varphi)*(b\varphi)$.

Let $a=v^\alpha_{ij}$ and $b=v^\beta_{kl}$.  Then to find $c$, it is
enough that
\[ \ip{(v^\gamma_{st})^*}{c\varphi} = 
\ip{(v^\gamma_{st})^*}{(a\varphi)*(b\varphi)} \]
for all $\gamma,s,t$.  However,
\begin{align*}
\ip{(v^\gamma_{st})^*}{(a\varphi)*(b\varphi)}
&= \sum_r \varphi((v^\gamma_{sr})^* v^\alpha_{ij})
\varphi((v^\gamma_{rt})^* v^\beta_{kl})
= \delta_{\alpha,\beta} \delta_{\alpha,\gamma}
\delta_{s,i} \delta_{t,l} \delta_{j,k}
\frac{1}{\Lambda_\alpha^2 \lambda^\alpha_i \lambda^\alpha_j} \\
&= \delta_{\alpha,\beta} \delta_{j,k}
\frac{1}{\Lambda_\alpha \lambda^\alpha_j}
\varphi( (v^\gamma_{st})^* v^\alpha_{il} ),
\end{align*}
from which it follows that
\[ v^\alpha_{ij} * v^\beta_{kl} = \delta_{\alpha,\beta} \delta_{j,k}
\frac{1}{\Lambda_\alpha \lambda^\alpha_j} v^\alpha_{il}. \]
In particular,
\[ \chi_\alpha * \chi_\beta = \delta_{\alpha,\beta} \sum_i
\frac{1}{\Lambda_\alpha \lambda^\alpha_i} v^\alpha_{i,i}. \]

We could instead consider ``twisted'' convolution:
\[ a \star \omega = \hat\lambda\big( \hat\omega[ a\xi_0,
\hat\Lambda(\lambda(\omega)^*) ] ). \]
Note quite sure where this goes-- to copy the Dixmier idea, we'd need
to find a ``central bai'' of such $\omega$, and it's not clear when we
can do this-- at the very best, we'd need $\G$ coamenable!

(So, maybe, spend some time thinking about what happens when for $A(G)$
with G discrete??)


\subsection{Todo}

\begin{itemize}
\item We do know that $WV$ (and/or $VW$) is a corep of $\G$, and so can talk
about ``central'' $L^2(\G)$ vectors.  However, should show that this does not
(in non-Kac case?) give a coaction of $A$ (unfortunately).
\item Then think about Dixmier's proof:
\begin{itemize}
\item Does ``convolution'' of central elements of $L^2(\G)$ make sense?
\item I think want something like
central $\eta$ such that there is a bounded operator $T$ with
$\Lambda(\hat\lambda(\hat\omega_{\xi,\eta})) = (\xi*\eta^*) = T(\xi)$?
Then want these to give a bai\ldots
\end{itemize}
\end{itemize}


\section{Commutative case}

Suppose now that $(A,\Delta)$ is a compact quantum group with $A$ commutative.
We shall show that $A=C(G)$ for some compact group $G$, and that $\Delta$ is the
canonical comultiplication.

As $A$ is commutative, $A=C(G)$ for some compact Hausdorff space $G$.  Then
$\Delta:C(G)\rightarrow C(G\times G)$ is a $*$-homomorphism, and so corresponds
to some map $G\times G\rightarrow G$.  That $\Delta$ is coassociative means that
$G$ becomes a compact semigroup.  At this stage, we remark that it is possible
to use some compact semigroup theory to show directly that the cancellation
conditions imply that $G$ must be a compact group.  Instead, we shall use some
general theory.

Let $U\in M_n(C(G))$ be a finite-dimensional corepresentation, and let
$\pi:G\rightarrow M_n$ be the associated continuous map, given by the
isomorphism $M_n(C(G)) = C(G;M_n)$.  Then $U$ being a corepresentation
corresponds to $\pi$ being a homomorphism.
We now adapt an argument from \cite{woro3}.  For any finite-dimensional unitary
representation $\pi:G\rightarrow U(n)$ (where $U(n)$ is the $n$-dimensional
unitary group) we note that $\pi(G)$ is a compact sub-semigroup of $U(n)$.
If $A\in \pi(G)$ then by compactness, we can find a sequence $n(i)$ of naturals
with $n(i+1)>n(i)+1$, and with $A^{n(i)} \rightarrow B$ as $i\rightarrow\infty$.
Notice that $B\in\pi(G)$.  Then set $m(i) = n(i+1)-(n(i)+1)>0$, so that
\[ A^{m(i)} = A^{n(i+1)} (A^{-1})^{n(i)+1} \rightarrow B B^{-1} A^{-1}. \]
Hence $A^{-1}\in\pi(G)$, and so $\pi(G)$ is a compact subgroup of $U(n)$.

By following the general theory, we find a dense Hopf $*$-algebra $P(G)$ inside
$C(G)$; we see that $P(G)$ is precisely the collection of coefficients of
finite-dimensional unitary representations of $G$.

\begin{proposition}
We have that $G$ is a compact group, and the counit $\epsilon$ and antipode
$\kappa$ extend to $C(G)$ with the usual definitions coming from the group
structure of $G$.
\end{proposition}
\begin{proof}
That $P(G)$ is dense in $C(G)$ means that $P(G)$ separates the points of $G$;
that is, for $s,t\in G$ distinct, there is a unitary representation
$\pi:G\rightarrow U(n)$ with $\pi(s)\not=\pi(t)$.  

Consider the collection $\mc N$ of all subsets $N_\pi=\{ s\in G:
\pi(s)=1 \}$ where $\pi$ is a finite-dimensional unitary representation.
Then each $N_\pi$ is a non-empty compact set, as $\pi(G)$ is a compact group.
Then $\mc N$ has the finite-intersection property, and $N_{\pi_1}
\cap\cdots\cap N_{\pi_n} = N_\pi$ where $\pi=\pi_1\oplus\cdots\oplus\pi_n$.
So $\bigcap\mc N$ is non-empty, and thus there is some $e_G\in G$ with
$e_G\in N_\pi$ for all $\pi$.  As such $\pi$ separate points, $e_G$ is unique.
Then, for any $\pi$ and $t\in G$, we find that $\pi(te_G) = \pi(t) = \pi(e_Gt)$,
so by the separation of points property, $e_G$ is the identity of $G$.

Now fix $t_0\in G$.  For each $\pi$ there is at least one $t\in G$ with
$\pi(t)=\pi(t_0)^{-1}$ so that $\pi(tt_0) = \pi(t_0t) = \pi(e_G)$.  Again by
a finite-intersection property argument, we can show that there is at least one
such $t$ that works for all $\pi$.  Then separation of points shows that $t$ is
unique, and that $t=t_0^{-1}$.  So $G$ is a group.

The defining properties of $\epsilon$ and $\kappa$ now easily show that, for
$f\in P(G)$, we have $\epsilon(f) = f(e_G)$, and $\kappa(f)(s) = f(s^{-1})$
for $s\in G$.  These maps obviously extend by continuity to $C(G)$.
\end{proof}

The Haar state $\varphi$ corresponds to a Borel probability measure, $ds$, on
$G$.  That $\varphi$ is left and right invariant means that
\[ \int_G f(st) \ ds = \int_G f(ts) \ ds = \int_G f(s) \ ds
\qquad (t\in G, f\in C(G)). \]
Then by uniqueness, $ds$ must be the Haar measure on $G$.  We quickly remind
the reader why $ds$ has full support (equivalently, why $\varphi$ is faithful).
Towards a contradiction, suppose that $\varphi(f)=0$ for some non-zero positive
$f\in C(G)$.  Then there is a non-empty open set $U$ with $|U|=0$.  Then all
(left) translates of $U$ have zero measure; but as $G$ is a group, these cover
$G$, so by compactness, there is a finite subcover, and hence $|G|=0$,
contradiction.  So $\varphi$ is faithful.  Hence $A$ is already reduced,
and we can identify $L^2(G)$ with the GNS space for $\varphi$.

Let $U$ be a (unitary) corepresentation, and consider the contragradient
corepresentation $\overline{U}$, corresponding to $\overline{\pi}$.  Then
\[ \overline{\pi}(s) = \sum_{i,j=1}^n u_{ij}^*(s) e_{ij}
= \sum_{i,j=1}^n \overline{u_{ij}(s)} e_{ij}
= \overline{\pi(s)}, \]
where for $x\in M_n$, we again denote by $\overline{x} = (x^*)^t = (x^t)^*$ the
matrix obtained by pointwise conjugation of complex numbers.  As $A$ is
commutative, it is clear that $U$ unitary (respectively, invertible) implies
also that $\overline{U}$ is unitary (respectively, invertible), and so
Proposition~\ref{prop:conjunitary} becomes a triviality in this case.

From Lemma~\ref{lem:fmatrixkappa} we see that each ``$F$-matrix'' is a scalar
multiple of the identity, and so in particular diagonal.  Taking the normalisation
that $\tr(F^\alpha) = \tr((F^\alpha)^{-1})$, we must have that
$F^\alpha=I_{n_\alpha}$, and so $\Lambda_\alpha=n_\alpha$, for all $\alpha$.
Then each character $f_z$ is equal to the counit, and the scaling group (and of
course the modular group) is trivial.  Hence $\kappa=R$ the unitary antipode.

\subsection{Some formulae}

The GNS construction for $\varphi$ has the concrete form that $H=L^2(G)$,
the map $\Lambda:C(G)\rightarrow L^2(G)$ is formal identification of functions,
and $\pi:C(G)\rightarrow\mc B(L^2(G))$ is such that $\pi(f)$ is the operator
given by multiplication by $f$.  Then $Jf(s) = \overline{f(s)}$ for $s\in G,
f\in L^2(G)$ and $\hat J(f)(s) = \overline{f(s^{-1})}$.  Also
\[ W\in\mc B(L^2(G\times G)); \quad W\xi(s,t) = \xi(s,s^{-1}t)
\qquad(\xi\in L^2(G\times G), s,t\in G). \]

Let $(v^\alpha)$ be a complete family of pairwise non-equivalent irreducible
unitary corepresentations, with associated unitary representations
$(\pi_\alpha)$.  Then we identify $\ell^2_{n_\alpha} \otimes \ell^2_{n_\alpha}$
with a subspace of $L^2(G)$ via
\[ \delta_i^\alpha\otimes\delta_j^\alpha \mapsto \sqrt{n_\alpha}
\overline{ v^\alpha_{ij} }. \]
Then identifying $L^2(G)$ with $\bigoplus \ell^2_{n_\alpha} \otimes
\ell^2_{n_\alpha}$ we again find that
\[ W = \big( w_\alpha \big) = \Big( \sum_{i,j} v^\alpha_{ij} \otimes
e_{ij} \otimes 1 \Big) \in \mc B\Big( L^2(G) \otimes \bigoplus \ell^2_{n_\alpha}
\otimes \ell^2_{n_\alpha} \Big). \]

The left-regular representation is $\lambda:L^1(G)\rightarrow\mc B(L^2(G))$
given by
\[ \lambda(\omega) = (\omega\otimes\iota)(W); \quad
\lambda(\omega)(f) = \omega*f \qquad (\omega\in L^1(G), f\in L^2(G)), \]
that is, $\lambda(\omega)$ is the operator of left convolution by $\omega$.
In the above picture,
\[ \lambda(\omega) = \big( (\omega\otimes\iota)w_\alpha \big) \in \bigoplus_\alpha
\mathbb M_{n_\alpha}\otimes \mathbb M_{n_\alpha}, \]
where for each $\alpha$,
\[ (\omega\otimes\iota)w_\alpha = \sum_{ij} \ip{v^\alpha_{ij}}{\omega} e_{ij}
\otimes 1
= \int_G \omega(s) \pi_\alpha(s) ds \otimes 1. \]
So as usual, as a C$^*$-algebra, $C^*_r(G)$ is isomorphic to
$\bigoplus_n \mathbb M_{n_\alpha}$, but when concretely acting on $L^2(G)$,
we have to remember that each factor $\mathbb M_{n_\alpha}$ acts with
multiplicity $n_\alpha$; here I have chosen to write this as $e_{ij}\otimes 1$,
whereas classical sources usually add an ``$n_\alpha$'' term to indicate
multiplicity.

Let's just check this:
\begin{align*}
\lambda(\omega)(\delta^\alpha_i\otimes\delta^\alpha_j)
\leftrightarrow &\ n_\alpha^{1/2} \lambda(\omega)(\overline{v^\alpha_{ij}})
= n_\alpha^{1/2} \int_G \omega(s) \overline{v^\alpha_{ij}}(s^{-1}t) \ ds \\
&= n_\alpha^{1/2} \int_G \omega(s) \overline{
  \sum_k v^\alpha_{ik}(s^{-1}) v^\alpha_{kj}(t) } \ ds
= n_\alpha^{1/2} \int_G \omega(s) \sum_k v^\alpha_{ki}(s)
  \overline{ v^\alpha_{kj}(t) } \ ds \\
\leftrightarrow &\ \sum_k \int_G \omega(s) v^\alpha_{ki}(s) \ ds \
  \delta^\alpha_k \otimes \delta^\alpha_j
= \int_G \omega(s) \pi_\alpha(s)\delta^\alpha_i \otimes \delta^\alpha_j.
\end{align*}

From general LCQG theory, it's
easy\footnote{We have $(\hat\Lambda(\lambda(\omega))|\Lambda(a))
= \ip{a^*}{\omega} = \int_G \omega(s) \overline{a(s)} \ ds$ and as
$\Lambda(a)=a$ under formal identification of functions $C(G)\subseteq
L^2(G)$ the result follows.} to see
that $\hat\Lambda(\lambda(\omega)) = \omega$ for $\omega \in L^1(G)\cap L^2(G)$.
From above, we find that the weight on $C^*_r(G) \cong \bigoplus_\alpha
\mathbb M_{n_\alpha}$ is
\[ \hat\varphi\big( (x_\alpha) \big) = \sum_\alpha n_\alpha \tr(x_\alpha), \]
where $\tr:\mathbb M_{n_\alpha}\rightarrow\mathbb C$ is the usual trace
$\tr(x) =\sum_{i=1}^{n_\alpha} x_{ii}$.  Then
\begin{align*} \big( \hat\Lambda((x_\alpha)) \big| \hat\Lambda((y_\alpha)) \big)
&= \sum_\alpha n_\alpha \tr(y_\alpha^*x_\alpha)
= \sum_\alpha n_\alpha \sum_{ij} \overline{y^\alpha_{ij}} x^\alpha_{ij} \\
&= \sum_\alpha \Big(
\sum_{ij} \sqrt{n_\alpha} x^\alpha_{ij} \delta^\alpha_i\otimes\delta^\alpha_j
\Big|
\sum_{kl} \sqrt{n_\alpha} x^\alpha_{kl} \delta^\alpha_k\otimes\delta^\alpha_l
\Big).
\end{align*}
Hence there is an isomorphism $H_{\hat\varphi} \rightarrow
\bigoplus_\alpha \ell^2_{n_\alpha} \otimes \ell^2_{n_\alpha}$,
\[ \hat\Lambda((x_\alpha)) \mapsto \Big( 
\sum_{ij} \sqrt{n_\alpha} x^\alpha_{ij} \delta^\alpha_i\otimes\delta^\alpha_j
\Big). \]
Under this, for $\omega\in L^1(G)\cap L^2(G)$,
\[ \omega = \hat\Lambda(\lambda(\omega)) \mapsto
\Big( \sum_{ij} \sqrt{n_\alpha} \ip{v^\alpha_{ij}}{\omega}
\delta^\alpha_i\otimes\delta^\alpha_j \Big). \]
If we identify $\ell^2_{n_\alpha} \otimes \ell^2_{n_\alpha}$ with the space of
Hilbert-Schmidt operators on $\ell^2_{n_\alpha}$, then the $\alpha$-component
of $\hat\Lambda(\lambda(\omega))$ is precisely
$\sqrt{n_\alpha}\int_G \omega(s)\pi_\alpha(s) \ ds$.
We need to be a little careful: here
\[ \ell^2_{n_\alpha} \otimes \ell^2_{n_\alpha} \ni \delta_i \otimes
\delta_j \mapsto e_{ij} \in \mc{HS}(\ell^2_{n_\alpha}), \]
where $e_{ij}:\delta_k \mapsto \delta_{j,k} \delta_i$ and so
$e_{ij} = \theta_{\delta_i,\delta_j}$.



\subsection{The Fourier algebra}

As usual, $L^1(\hat A)$ is the Fourier algebra $A(G)$.  Let $\xi,\eta\in
L^2(G)$ and let $\hat\omega_{\xi,\eta} \in A(G)$ be the functional
\[ VN(G) \mapsto \mathbb C; \quad x\mapsto (x\xi|\eta). \]
Now, as $VN(G) \cong \prod \mathbb M_{n_\alpha}$ it follows that
$A(G) \cong \ell^1-\bigoplus \mathbb T_{n_\alpha}$, an $\ell^1$-direct sum
of trace-class spaces.

Let us introduce some notation.  For a Hilbert space $H$, let
$\omega_{\xi,\eta} \in \mc B(H)_*$ be $x\mapsto (x\xi|\eta)$.  Then let
$\theta_{\xi,\eta}$ be the (rank-one) operator $\gamma \mapsto (\gamma|\eta)\xi$.
Then the map $\omega_{\xi,\eta} \mapsto \theta_{\xi,\eta}$ extends to the
identification of $\mc B(H)_*$ with the trace-class operators $\mc T(H)$.  For
$x\in\mc B(H)$ we have $\tr(x\theta_{\xi,\eta}) = \tr(\theta_{\xi,\eta} x)
= (x\xi|\eta) = \ip{x}{\omega_{\xi,\eta}}$.

When $H=\ell^2_n$, as usual we have $e_{ij} = \theta_{\delta_i,\delta_j}$
and so $\omega_{ij} = \omega_{\delta_i,\delta_j}=e_{ij}$ as a trace-class
operator.  Then
\[ \ip{e_{ij}}{\omega_{kl}} = \tr(e_{ij} e_{kl}) = \delta_{jk} \delta_{il}. \]
So $\omega_{kl}$ sends $e_{lk}$ to $1$, and all the other matrix units to $0$.
(This is sometimes called ``trace-duality'' to distinguish it from
``parallel-duality'').  

Suppose $\xi,\eta\in L^1(G)\cap L^2(G)$ so that
$\xi = \hat\Lambda(\lambda(\xi))$ and the same for $\eta$.  Then
\begin{align*} \ip{e^\alpha_{ij}}{\hat\omega_{\xi,\eta}}
&= \big( e^\alpha_{ij} \hat\Lambda(\lambda(\xi)) \big|
   \hat\Lambda(\lambda(\eta)) \big)
= \hat\varphi\big( \lambda(\eta)^* e^\alpha_{ij} \lambda(\xi) \big) \\
&= n_\alpha \tr\big( \pi_\alpha(\eta)^* e^\alpha_{ij} \pi_\alpha(\xi) \big)
= n_\alpha \tr\big( e^\alpha_{ij} \pi_\alpha(\xi)\pi_\alpha(\eta)^* \big).
\end{align*}
Hence, using that the integrated form of $\pi_\alpha$ is a $*$-homomorphism
$L^1(G)\rightarrow\mathbb M_{n_\alpha}$,
\[ \hat\omega_{\xi,\eta} = (\omega_\alpha) \in \ell^1-\bigoplus_\alpha
\mathbb T_{n_\alpha} \qquad \omega_\alpha = 
n_\alpha \int_G (\xi*\eta^*)(s) \pi_\alpha(s)\ ds. \]
Here $\eta^*(s) = \overline{\eta(s^{-1})}$ and $\xi*\eta^*$ is the convolution
product (again, this reflects the use of the Takesaki--Tatsumma, aka quantum-group,
embedding of $A(G)$ into $C_0(G)$, not the Eymard embedding).



\subsection{Contragradient representations}

For each $\alpha$ consider the contragradient $\overline{v_\alpha}$.  We have
that
\[ (\overline{v_\alpha})_{ij}(s) = \overline{ v^\alpha_{ij}(s) }
= \overline{ \pi_\alpha(s)_{ij} } = \pi_\alpha(s^{-1})_{ji}. \]
Let $\overline{\pi}_\alpha$ be the induced representation, which in this
(commutative) situation is unitary.  We can have two situations: either
$\overline{\pi}_\alpha$ is equivalent to $\pi_\alpha$, or it is not.

\begin{example}
If $G=SU(2)$ then it's well-known that for each $n$ there is exactly one
equivalence class of irreducible representations of dimension $n$.  Hence
here $\overline{\pi}_\alpha$ is always equivalent to $\pi_\alpha$.
\end{example}

\begin{example}
If $G$ is abelian, then every irreducible representation is one-dimensional,
and so is a continuous character $\alpha:G\rightarrow\mathbb T$.  Then
$\overline\alpha$ is just $\overline\alpha(s) = \overline{\alpha(s)}$.
Then observe equivalence of one-dimensional representations corresponds
exactly to genuine equality of functions $G\rightarrow\mathbb T$.  Then
$\alpha = \overline\alpha$ if and only if $\alpha(s)\in\{1,-1\}$ for all $s$.
\end{example}


\subsection{Todo}

Maybe try to write-down the coproduct (and/or product on $A(G)$) using
the ``Fusion-rules''??  Try to write down the antipode on $VN(G)$??




\section{Completions of the Hopf algebra}

It somewhat folklore that the Hopf $*$-algebra $\mc A$ can be completed to
give back $C(\G)$ or $C^u(\G)$.  We justified this (at the reduced level)
in Section~\ref{sec:faith_haar}.

However, there are some subtle points here, going back to Woronowicz and
especially highlighted by Dijkhuizen and Koornwinder.  The issues is that
in general a (unital) $*$-algebra $\mc A$ need not have any interesting
C$^*$-algebra completion.  Let $\mc A^+$ be the positive cone generated
by elements of the form $\{ a^*a : a\in\mc A \}$.  Then a linear map
$\phi:\mc A\rightarrow\mathbb C$ is positive if $\phi(\mc A^+)\subseteq
[0,\infty)$ and is a state if additionally $\phi(1)=1$.
If $\phi$ is a state on $\mc A$ then we can form
the pre-GNS space $(H,\xi_0)$.  Indeed, the Cauchy-Schwarz inequality is
enough to show that $N_\phi=\{a\in\mc A:\phi(a^*a)=0\}$ is a left ideal
in $\mc A$ (compare \cite[Chapter~I, Lemma~9.6]{tak1} for example), and so
we define $H = \mc A/N_\phi$, let $\xi_0$ be the equivalence class of $1$,
so that we can identify the equivalence class of $a$ with $a\xi_0$, and then
equip $H$ with the inner-product $(a\xi_0|b\xi_0) = \phi(b^*a)$.  Note that
we have not completed $H$ and so $H$ is only a pre-Hilbert space.

Then for $a\in\mc A$ define $\pi(a):H\rightarrow H$ by $\pi(a)(b\xi_0)
=(ab)\xi_0$.  That $N_\phi$ is a left ideal shows that $\pi(a)$ is
well-defined; clearly $\pi(a)$ is linear and adjointable, in the sense
that
\[ \big( \pi(a) b\xi_0 \big| c\xi_0) \big)
= \big( b\xi_0 \big| \pi(a^*) c\xi_0) \big)
\qquad (a,b,c\in\mc A). \]
So the only missing piece of the usual GNS construction is whether $\pi(a)$
is bounded, and hence extends to the completion of $H$.  For a C$^*$-algebra
this is a subtle point going back to the early days of the axiomatisation of
the subject.

The following can be found in Dijkhuizen and Koornwinder.

\begin{proposition}
Let $\mc A$ be the Hopf $*$-algebra associated to a CQG $(A,\Delta)$.
Then if $\pi:\mc A\rightarrow\mc L(H_0)$ is a $*$-map into the adjointable
linear maps on an inner-product space $H_0$, then $\pi$ is bounded, and
so extends to a $*$-homomorphism $A\rightarrow\mc B(H)$ where $H$ is the
completion of $H_0$.
\end{proposition}
\begin{proof}
Let $(u_{ij})$ be a finite-dimensional unitary corepresentation of $A$,
so each $u_{ij}\in\mc A$.  As $\sum_k u_{ki}^* u_{kj} = \delta_{ij}1$,
for $\xi\in H_0$, and any $i,j$,
\begin{align*} \|\xi\|^2 = (\xi|\xi) = \sum_{k} (\pi(u_{ki}^*u_{ki})\xi|\xi)
= \sum_k (\pi(u_{ki})\xi|\pi(u_{ki})\xi)
\geq \| \pi(u_{ji})\xi \|^2. \end{align*}
It follows that $\|\pi(u_{ij})\|\leq 1$ for all $i,j$.  As $\mc A$ is spanned
by such elements, we have shown that $\pi(a)$ is bounded for all $a\in\mc A$.
\end{proof}

As such, for any state $\phi$ on $\mc A$ we can find a Hilbert space $H$,
a $*$-homomorphism $\pi:\mc A\rightarrow\mc B(H)$ and $\xi\in H$ such that
$\phi(a) = (\pi(a)\xi|\xi)$ for all $a\in\mc A$.  So states on $\mc A$ biject
with states on the universal C$^*$-algebra completion of $\mc A$, namely
$C^u(\G)$.




\section{Do we need to be so careful?}

In the section on von Neumann algebras, we seemingly used the Hopf $*$-algebra
quite a bit-- this is equivalent to using that the Haar state is KMS.  Here
we present some examples to show that \emph{some sort of condition} is needed.

\subsection{Counter-example}

We find a C$^*$-algebra $A$ which admits a faithful state, but such that in
the GNS representation, the state is not faithful on $A''$.

The following was suggested to us by Narutaka Ozawa%
\footnote{See \texttt{http://mathoverflow.net/questions/%
93295/separating-vectors-for-c-algebras/93383$\sharp$93383}}

Let $A=C([0,1],\mathbb M_2)$.  Let $C\subseteq [0,1]$ be a closed set with empty
interior but positive (Lebesgue) measure.  For example, let $(\epsilon_n)$
be a sequence in $(0,1)$ with $\sum_n \epsilon_n<1/2$, let $(q_n)$ be an
enumeration of the rationals in $[0,1]$, and let
$C=[0,1]\setminus\bigcup_n (q_n-\epsilon_n,q_n+\epsilon_n)$.

Define a state $\phi$ on $A$ by
\[ \phi(a) = \int_C a(x)_{11} \ dx +
\frac12\int_{[0,1]\setminus C} a(x)_{11}+a(x)_{22} \ dx. \]
Here $a$ is a continuous function $[0,1]\rightarrow\mathbb M_2$, and
$a(x)_{ij}$ is the $(i,j)$th entry of the matrix $a(x)$.

Now, $a\geq 0$ if and only if $a(x)\geq 0$ for all $x$, which implies that
$a(x)_{11}, a(x)_{22}\geq 0$.  So $\phi$ is positive, and faithful because
$[0,1]\setminus C$ is dense and open.
Clearly $\phi(1)=1$, so $\phi$ is a state.  

For $a,b\in A$ the pre-inner-product induced by $\phi$ is
\[ (a|b) = \phi(b^*a) = \int_C a(x)_{11}\overline{b(x)_{11}} +
a(x)_{21} \overline{ b(x)_{21} } \ dx +
\frac12\int_{[0,1]\setminus C} \sum_{i,j} a(x)_{ij}\overline{b(x)_{ij}} \ dx. \]
Let $\mu_1$ be the measure of $[0,1]$ given by
\[ \int f \ d\mu_1 = \int_C f + \frac12 \int_{[0,1]\setminus C} f. \]
Let $\mu_2$ be $1/2$ of Lebesgue measure, restricted to $[0,1]\setminus C$.
As Lebesgue measure dominates both $\mu_1$ and $\mu_2$, it's easy to see that
$\mu_1,\mu_2$ are regular measures.
Then the GNS space for $\phi$ can thus be identified with
\[ \mathbb M_{2,1}(L^2(\mu_1)) \oplus \mathbb M_{2,1}(L^2(\mu_2)), \]
thought of as column vectors, with $A$ acting by matrix multiplication,
and then $C([0,1])$ acting by pointwise multiplication, in the obvious way.
The cyclic vector is $\begin{pmatrix} 1 \\ 0 \end{pmatrix} \oplus
\begin{pmatrix} 0 \\ 1 \end{pmatrix}$.  To ease notation, let the GNS space
be $H_1\oplus H_2$, and let $\pi_i:A\rightarrow\mc B(H_i)$ be the resulting
representations.

\begin{lemma}
Let $A$ be a C$^*$-algebra, $\pi_1:A\rightarrow\mc B(H_1)$ a non-degenerate
representation, let $H_2\subseteq H_1$ be an invariant subspace, and let
$\pi_2:A\rightarrow\mc B(H_2)$ be the restriction of $\pi_1$.  Let $\pi:
A\rightarrow\mc B(H_1\oplus H_2)$ be the direct sum of $\pi_1$ with $\pi_2$.
Then $\pi(A)'' = \{ (T,S) : T\in\pi_1(A)'', S=T|_{H_2} \}$ acting diagonally
on $H_1\oplus H_2$, a von Neumann algebra which is isomorphic to $\pi_1(A)''$.
\end{lemma}
\begin{proof}
As $\pi_1$ is non-degenerate, so is $\pi_2$, and hence so is $\pi$.  So we
need to compute the $\sigma$-weak closure of $\pi(A)$.  On bounded sets this
agrees with the strong closure, and from this is it obvious that $\pi(A)''$
has the stated form.
\end{proof}

Notice that in our case $L^2(\mu_2)$ is a subspace of $L^2(\mu_1)$ if we
identify $\xi\in L^2(\mu_2)$ with $\xi\chi_{[0,1]\setminus C}\in L^2(\mu_1)$.

Let $\mf A$ be the commutant of $C([0,1])$ in $\mc B(L^2(\mu_1))$.  Then
$\pi_1(A)'$ consists of matrices $\begin{pmatrix} T & 0 \\ 0 & T \end{pmatrix}$
with $T\in\mf A$.  Thus $\pi_1(A)'' = \mathbb M_2(\mf A')$.  So we need to
compute the bicommutant of $C([0,1])$ in $\mc B(L^2(\mu_1))$.  By duality
arguments, and (for example) Lusin's theorem, this is $L^\infty(\mu_1) \cong
L^\infty([0,1])$.

Thus $\pi(A)'' \cong L^\infty([0,1])$.  However, the cyclic vector for the
GNS construction yields the state
\[ \tilde\phi(a) = \int a_11 \ d\mu_1 + \int a_22 \ d\mu_2, \]
which is not faithful (there are measurable, non-continuous functions
supported on $C$ which are not zero almost everywhere).




\begin{thebibliography}{aa}

\bibitem{bmt} Bedos, Murphy, Tuset, ``Co-amenability of compact quantum groups''.

\bibitem{ks} Kyed, So{\l}tan, ``Property (T) and exotic quantum group norms''.

\bibitem{tak1} Takesaki Volume 1.

\bibitem{timm} T. Timmermann, ``An Invitation to Quantum Groups and Duality''.

\bibitem{woro1} S.\,L. Woronowicz, ``On the purification of factor states''

\bibitem{woro2} S.\,L. Woronowicz, ``Compact quantum groups''

\bibitem{woro3} S.\,L. Woronowicz, ``Compact matrix pseudogroups''

\end{thebibliography}


\end{document}
