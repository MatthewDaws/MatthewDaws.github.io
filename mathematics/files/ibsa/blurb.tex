\documentclass[twoside,12pt,a4paper]{article}

\usepackage[margin=2cm]{geometry}
%\usepackage{times}
\usepackage{amsmath,amssymb,amsthm,latexsym}
%\usepackage{amsfonts}


\begin{document}

\noindent{\large\textbf{Reading course: Banach spaces and algebras}
(MATH5002)}

\medskip

I propose to run a reading course in the 2nd semester, under the MATH5002 remit.
MMATH, MSc, and PhD students are welcome to attend; for MMATH and MSc students this
will be a 20 credit M-Level module.  There will be around 15 hours of contact time,
and so this should count for 15 ``credits'' for PhD student training (but check
with your supervisor!)

\medskip
\noindent\textbf{Syllabus}
\medskip

I will use the book ``Introduction to Banach Spaces and Algebras'' by
Graham Allan (Oxford Graduate Texts in Mathematics, 2010).  The syllabus will
be rather similar to the old ``Linear Analysis II'' module.  In brief outline:
\begin{itemize}
\item Revision of normed spaces; dual spaces; Hahn-Banach.
\item Weak and weak$^*$-topologies; second duals; geometric forms of
Hahn-Banach; Krein-Milman.
\item Baire category, Open Mapping, Closed Graph, Uniform boundedness
theorems.
\item Basics of Banach algebras; constructions; group of units.
\item Spectrum; Characters; Gelfand Theory.
\item Commutative Banach algebras; holomorphic functional calculus.
\item C$^*$-algebras; continuous functional calculus.
\item Representation theory; modules; radicals; uniqueness of norm.
\item Applications and examples to group algebras.
\end{itemize}
I will hold a weekly meeting, of an hour to an hour and half, where the weeks
reading will be discussed, and problem sets discussed (problems will be set
from the book).  I hope that participants will prepare answers and present
these to the group; these can then be written up for assessment purposes,
if required.

The plan is to follow the usual Semester 2 teaching schedule, so we will meet
11 times (maybe briefly in the 1st week, and then also in the final week
after teaching, so perhaps actually 12 meetings).

The library contains 4 copies of this book, but 3 are limited to one week's
loan at a time.  Obviously, if more than a handfull of people attend, then
you will need to buy the book; a paperback version costs \pounds 35.

\medskip
\noindent\textbf{Notes to MMATH/MSc students}
\medskip

I cannot stress enough that \textbf{this is not an easy option}.  The quantity
of material will be comparable to a lectured 20 credit course, but with no
lectures!  In my opinion, reading mathematics is a somewhat different, and
harder, task than attending lectures.  I am likely to be very busy in
Semester 2, and will have no time outside of timetabled hours to help students.
\begin{itemize}
\item I will set the special prerequisite that you must have taken
MATH5015, ``Linear Analysis I'', and obtained 60\% or better in the exam.
\item Assessment, in line with the MATH5002 module catalogue entry, will
be 20\% coursework and 80\% unseen exam.
\item The coursework will take the form of 2 or 3 carefully written answers
to questions which have first been presented to the group.
\item The exam will be a usual 3 hour exam in the summer examination
period.
\end{itemize}

\end{document}