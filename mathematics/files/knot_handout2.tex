\documentclass[a4paper]{article}

\usepackage[margin=2cm,nohead]{geometry}
\usepackage{pstricks}
\usepackage{graphics}
\pagestyle{empty}

\begin{document}

{\Huge\textbf{The writhe of an oriented knot or link}}

\large

\bigskip\bigskip

\psset{linewidth=4mm}
No orientation:
\scalebox{0.5}{\begin{pspicture}(6,6)
  \psccurve(3,0.5)(5.5,3)(4,5.5)(2,4)(3,3.5)(4,4.5)(2,5.5)(0.5,3)
  \psccurve[linewidth=2mm,linecolor=white](3,0.5)(5.5,3)(4,5.5)(2,4)(3,3.5)(4,4.5)(2,5.5)(0.5,3)
  \psecurve(5.5,3)(4,5.5)(2,4)(3,3.5)
  \psecurve[linewidth=2mm,linecolor=white](5.5,3)(4,5.5)(2,4)(3,3.5)
\end{pspicture}}
\qquad
With an orientation:
\scalebox{0.5}{\begin{pspicture}(6,6)
  %\psgrid[subgriddiv=1, griddots=6, gridlabels=7pt](0,0)(6,6)
  \psccurve(3,0.5)(5.5,3)(4,5.5)(2,4)(3,3.5)(4,4.5)(2,5.5)(0.5,3)
  \psccurve[linewidth=2mm,linecolor=white](3,0.5)(5.5,3)(4,5.5)(2,4)(3,3.5)(4,4.5)(2,5.5)(0.5,3)
  \psecurve(5.5,3)(4,5.5)(2,4)(3,3.5)
  \psecurve[linewidth=2mm,linecolor=white](5.5,3)(4,5.5)(2,4)(3,3.5)
  \pstriangle*(5.5,3)(1,0.5)
  \pstriangle*(0.5,3.5)(1,-0.5)
  \pstriangle*[gangle=-90](2.8,3.5)(1,0.5)
\end{pspicture}}

\bigskip

The \textbf{writhe} is calculated by looking at all the \textbf{crossings}: every time we
have a \textbf{positive crossing} we add \textbf{one} to our count; every time
we have a \textbf{negative crossing} we add \textbf{minus one} to our count.

\medskip

Positive crossings:
\scalebox{0.5}{\begin{pspicture}(4,4)
  %\psgrid[subgriddiv=1, griddots=6, gridlabels=7pt](0,0)(4,4)
  \psline{->}(4,0)(0,4)
  \psline[linewidth=8mm,linecolor=white](0,0)(4,4)
  \psline{->}(0,0)(4,4)
\end{pspicture}}
\hspace{1ex}
\scalebox{0.5}{\begin{pspicture}(4,4)
  \psline{->}(2,4)(2,0)
  \psline[linewidth=8mm,linecolor=white](4,2)(0,2)
  \psline{->}(4,2)(0,2)
\end{pspicture}}
\quad
Negative crossings:
\scalebox{0.5}{\begin{pspicture}(4,4)
  %\psgrid[subgriddiv=1, griddots=6, gridlabels=7pt](0,0)(4,4)
  \psline{->}(0,0)(4,4)
  \psline[linewidth=8mm,linecolor=white](4,0)(0,4)
  \psline{->}(4,0)(0,4)
\end{pspicture}}
\hspace{1ex}
\scalebox{0.5}{\begin{pspicture}(4,4)
  \psline{->}(4,2)(0,2)
  \psline[linewidth=8mm,linecolor=white](2,4)(2,0)
  \psline{->}(2,4)(2,0)
\end{pspicture}}

\medskip

So we calculate the writhe of the oriented knot above:

\scalebox{0.5}{\begin{pspicture}(6,6)
  \psccurve(3,0.5)(5.5,3)(4,5.5)(2,4)(3,3.5)(4,4.5)(2,5.5)(0.5,3)
  \psccurve[linewidth=2mm,linecolor=white](3,0.5)(5.5,3)(4,5.5)(2,4)(3,3.5)(4,4.5)(2,5.5)(0.5,3)
  \psecurve(5.5,3)(4,5.5)(2,4)(3,3.5)
  \psecurve[linewidth=2mm,linecolor=white](5.5,3)(4,5.5)(2,4)(3,3.5)
  \pstriangle*(5.5,3)(1,0.5)
  \pstriangle*(0.5,3.5)(1,-0.5)
  \pstriangle*[gangle=-90](2.8,3.5)(1,0.5)
\end{pspicture}}
\quad\raisebox{5ex}[0ex][0ex]{\parbox{25ex}{Pick out the only crossing to consider:}}\quad
\scalebox{0.5}{\begin{pspicture}(4,4)
  \psline{->}(4,1)(0,3)
  \psline[linewidth=8mm,linecolor=white](4,3)(0,1)
  \psline{->}(4,3)(0,1)
\end{pspicture}}
\quad\raisebox{5ex}[0ex][0ex]{\parbox{30ex}{This is a negative crossing, so the writhe is $-1$.}}

\bigskip

Can you find the writhe of the following knots and links?
You will have to choose an orientation first!

\scalebox{0.7}{
\begin{pspicture}(8,8)
  %\psgrid[subgriddiv=1, griddots=6, gridlabels=7pt](0,0)(8,8)
  \psccurve(0.5,4)(3,0.5)(5,2)(3,4.5)(2.5,6)(4,7.5)(5.5,6)(5,4.5)(3,2)(5,0.5)(7.5,4)(4,6)
  \psccurve[linewidth=2mm,linecolor=white](0.5,4)(3,0.5)(5,2)(3,4.5)(2.5,6)(4,7.5)(5.5,6)(5,4.5)(3,2)(5,0.5)(7.5,4)(4,6)
  \psecurve(0.5,4)(3,0.5)(5,2)(2.5,6)
  \psecurve[linewidth=2mm,linecolor=white](0.5,4)(3,0.5)(5,2)(2.5,6)
  \psecurve(5.5,6)(5,4.5)(3,2)(5,0.5)
  \psecurve[linewidth=2mm,linecolor=white](5.5,6)(5,4.5)(3,2)(5,0.5)
  \psecurve(5,2)(3,4.5)(2.5,6)(4,7.5)
  \psecurve[linewidth=2mm,linecolor=white](5,2)(3,4.5)(2.5,6)(4,7.5)
  \psecurve(5,0.5)(7.5,4)(4,6)(0.5,4)
  \psecurve[linewidth=2mm,linecolor=white](5,0.5)(7.5,4)(4,6)(0.5,4)
\end{pspicture}
\begin{pspicture}(8,8)
  \psccurve(0.5,4)(3,0.5)(5,2)(3,4.5)(2.5,6)(4,7.5)(5.5,6)(5,4.5)(3,2)(5,0.5)(7.5,4)(4,6)
  \psccurve[linewidth=2mm,linecolor=white](0.5,4)(3,0.5)(5,2)(3,4.5)(2.5,6)(4,7.5)(5.5,6)(5,4.5)(3,2)(5,0.5)(7.5,4)(4,6)
  \psecurve(5,4.5)(3,2)(5,0.5)(7.5,4)
  \psecurve[linewidth=2mm,linecolor=white](5,4.5)(3,2)(5,0.5)(7.5,4)
  \psecurve(3,0.5)(5,2)(3,4.5)(2.5,6)
  \psecurve[linewidth=2mm,linecolor=white](3,0.5)(5,2)(3,4.5)(2.5,6)
  \psecurve(4,7.5)(5.5,6)(5,4.5)(3,2)
  \psecurve[linewidth=2mm,linecolor=white](4,7.5)(5.5,6)(5,4.5)(3,2)
  \psecurve(7.5,4)(4,6)(0.5,4)(3,0.5)
  \psecurve[linewidth=2mm,linecolor=white](7.5,4)(4,6)(0.5,4)(3,0.5)
\end{pspicture}
\begin{pspicture}(8,8)
  \psccurve(0.5,3)(3,5.5)(5.5,4)(3,0.5)
  \psccurve[linewidth=2mm,linecolor=white](0.5,3)(3,5.5)(5.5,4)(3,0.5)
  \psccurve(2.5,5)(5,7.5)(7.5,6)(5,2.5)
  \psccurve[linewidth=2mm,linecolor=white](2.5,5)(5,7.5)(7.5,6)(5,2.5)
  \psecurve(3,0.5)(0.5,3)(3,5.5)(5.5,4)(3,0.5)
  \psecurve[linewidth=2mm,linecolor=white](3,0.5)(0.5,3)(3,5.5)(5.5,4)(3,0.5)
\end{pspicture}}

\scalebox{0.85}{\begin{pspicture}(8,8)
  \psccurve(0.5,2.5)(2,0.5)(4.5,2.5)(5.5,5.5)(4,7.5)(2.5,5.5)(3.5,2.5)(6,0.5)(7.5,2.5)(4,5)
  \psccurve[linewidth=2mm,linecolor=white](0.5,2.5)(2,0.5)(4.5,2.5)(5.5,5.5)(4,7.5)(2.5,5.5)(3.5,2.5)(6,0.5)(7.5,2.5)(4,5)
  \psecurve(0.5,2.5)(2,0.5)(4.5,2.5)(5.5,5.5)
  \psecurve[linewidth=2mm,linecolor=white](0.5,2.5)(2,0.5)(4.5,2.5)(5.5,5.5)
  \psecurve(4,7.5)(2.5,5.5)(3.5,2.5)(6,0.5)
  \psecurve[linewidth=2mm,linecolor=white](4,7.5)(2.5,5.5)(3.5,2.5)(6,0.5)
  \psecurve(6,0.5)(7.5,2.5)(4,5)(0.5,2.5)
  \psecurve[linewidth=2mm,linecolor=white](6,0.5)(7.5,2.5)(4,5)(0.5,2.5)
\end{pspicture}
\qquad\qquad
\begin{pspicture}(8,8)
  \psccurve(0.5,2.5)(2,0.5)(4.5,2.5)(5.5,5.5)(4,7.5)(2.5,5.5)(3.5,2.5)(6,0.5)(7.5,2.5)(4,5)
  \psccurve[linewidth=2mm,linecolor=white](0.5,2.5)(2,0.5)(4.5,2.5)(5.5,5.5)(4,7.5)(2.5,5.5)(3.5,2.5)(6,0.5)(7.5,2.5)(4,5)
  \psecurve(2,0.5)(4.5,2.5)(5.5,5.5)(4,7.5)
  \psecurve[linewidth=2mm,linecolor=white](2,0.5)(4.5,2.5)(5.5,5.5)(4,7.5)
  \psecurve(7.5,2.5)(4,5)(0.5,2.5)(2,0.5)
  \psecurve[linewidth=2mm,linecolor=white](7.5,2.5)(4,5)(0.5,2.5)(2,0.5)
  \psecurve(2.5,5.5)(3.5,2.5)(6,0.5)(7.5,2.5)
  \psecurve[linewidth=2mm,linecolor=white](2.5,5.5)(3.5,2.5)(6,0.5)(7.5,2.5)
\end{pspicture}}

\end{document}
