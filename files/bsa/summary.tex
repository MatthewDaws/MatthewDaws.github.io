\documentclass[twoside,12pt,a4paper]{article}

\usepackage[margin=2cm]{geometry}
%\usepackage{times}
\usepackage{amsmath,amssymb,amsthm,latexsym}
%\usepackage{amsfonts}


\begin{document}

\noindent{\large\textbf{Reading course: Banach spaces and algebras}
(MATH5002)}\bigskip

This document will list the parts of the book we have read.  When I say
something is ``important'', or just list it without comment,
then it might very well be on an exam.  If something is ``interesting'',
then I think it provides useful context, but an exam wouldn't ask you to
state and prove such a result.

My view is that the book has turned out to be harder to read than I expected,
and so we will in the end not study quite as much as I had originally
expected to.

\section{Revision of normed spaces; dual spaces; Hahn-Banach}

See Sections~2.1, 2.2, 2.3, 2.5, 3.1.  All of this should be revision from
Linear Analysis~I, and hence is not examinable (but will be freely used
in the rest of the course).


\section{Weak and weak$^*$-topologies; second duals; geometric forms of
Hahn-Banach; Krein-Milman}

\begin{itemize}
\item Theorem~3.1, the Hahn-Banach theorem stated for sub-linear functionals,
is probably new (but essentially the same as versions of Hahn-Banach as
seen before).
\item Section~3.3; don't worry too much about proofs.  Read section~3.4.
\item Section~3.5: Theorem~3.18 and Theorem~3.21
(Banach-Alaoglu).
\item Section~3.7 (these are often also called the ``geometric forms of
Hahn-Banach'').  Theorem~3.26 and all its corollaries.
\item Section~3.8: Theorem 3.31.
\end{itemize}


\section{Baire category, Open Mapping, Closed Graph, Uniform boundedness
theorems}

\begin{itemize}
\item Section 3.9: Theorem~3.34 and Theorem~3.36.  You need to read the
relevant part of Chapter~1 as well.
\item Section 3.10: Lemma~3.38
\item Section 3.11: Theorem~3.40 (also remember my ``correction'' to this.)
Try to understand carefully how this fits with Lemma~3.38.  Understand
the corollaries.  Theorem~3.45
\item Section 3.12 is routine, but used later.
\item Sections 3.15 and 3.16 are interesting, but not examinable.
\end{itemize}


\section{Basics of Banach algebras; constructions; group of units}

\begin{itemize}
\item Section~4.1, Section~4.2.  Nothing really to learn here, but
important background.
\item Section~4.3: Standard constructions; lemma~4.8 isn't so important.
\item Section~4.4: Lemma~4.10 is very useful.  It's corollaries are important
too.
\end{itemize}


\section{Spectrum; Characters; Gelfand Theory}

\begin{itemize}
\item Section~4.5: Theorem 4.17, Theorem 4.19, Lemma 4.22.  Don't worry about
the proof of Theorem~4.23, but know the statement.
\item Section~4.6 is interesting, Theorem 4.28 is useful in calculations.
\item Section~4.9: Theorem~4.36 and its corollaries.  Proposition~4.39 is
interesting.
\item Section~4.10: Theorem~4.43.
\item Section~4.11: Theorem~4.46 and the corollary.  Understand examples~4.49
and 4.50 (but 4.52 and 4.53 are off-topic).  Theorem~4.54 is important,
as is Theorem~4.59 (don't worry about 4.55--4.58).  Example~4.62 and
Theorem~4.63 are the classical applications of this theory.
\end{itemize}


\section{Commutative Banach algebras; holomorphic functional calculus}

\begin{itemize}
\item Section~4.14: This is really just background, and is non-examinable.
\item Section~4.15: The statement of Theorem~4.89 is very important, but
don't worry about the proof.  Corollary~4.90 and Lemma~4.91.
Propositions~4.93 and 4.94.  Theorem~4.95.  Proposition~4.98.
Theorem~4.100 is a typical application of this material.
\item Section~4.16: Non-examinable, but useful background.
\end{itemize}


\section{C$^*$-algebras; continuous functional calculus}

\begin{itemize}
\item Sections~2.13 and 2.14.  These should be (mostly) revision from
past courses.  But make sure you know the material.
\item Section~2.15 is useful background material.
\item Section~4.7: all three theorems are important to know, but the proofs
are non-examinable.
\item Section~6.1.  General background; make sure you understand the
examples.  Ignore Lemma~6.3 and Theorem~6.4.
\item Sections~6.2 and 6.3 are a little off-topic.  We shall skip them
on a first reading, but material here will be needed later, so we'll
come back if, and only if, we need to.
\item Section~6.4: Skip Corollary~6.18, but everything else is important.
Our aim here is Theorem~6.24.
\item Section~6.5: Theorem~6.26 and corollaries.  Skip long discussion
on page 274.  Proposition~6.29, Theorem~6.30, Proposition~6.32.
The remaining results are nice background only.
\item Section~6.6: Theorem~6.41 (and have some understanding of how the
previous results combine to prove this).  Theorem~6.47.  I don't like
the way this is proved-- so the proofs here are not examinable; but the
statements should be understood.
\end{itemize}


\section{Representation theory; modules; radicals; uniqueness of norm}

\begin{itemize}
\item Section~5.1: Useful background.
\item Section~5.3: The aim here is Theorem~5.9 and Corollary~5.10.
Ignore the later material.
\item Section ``Automatic continuity'' starting on page~241.
Theorem~5.25 and Corollary~5.26.  I am not sure I like these proofs, so
I might come up with some of my own.  Otherwise we'll also need to go
back are read Section~3.13.
\end{itemize}


\section{Applications and examples to group algebras}

We probably will have run out of time by this point.  Nothing here
will be examinable; I will come up with the reading if needs be.

\end{document}