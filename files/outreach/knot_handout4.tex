\documentclass[a4paper]{article}

\usepackage[margin=2cm,nohead]{geometry}
\usepackage{pstricks}
\usepackage{graphics}
\usepackage{amsmath}
\pagestyle{empty}

\begin{document}

\psset{linewidth=2mm}

{\Huge\textbf{The Jones polynomial}}

\large

\bigskip

Both the writhe and the bracket polynomial are useful, but they have a big
flaw!  It's possible to draw the same knot in different ways, but have the
writhe and bracket polynomial change.

Fortunately, if we are cunning, we can combine them and cancel this problem
out.  This gives the \textbf{Jones polynomial}.

\bigskip

The Jones polynomial is
\begin{center}
\scalebox{1.5}{$V(\text{link}) = (-A^3)^{-w(\text{link})} \langle \text{link} \rangle$.}
\end{center}
Here $w(\text{link})$ is the writhe of the link we are considering.

\bigskip

\raisebox{13ex}[0ex][0ex]{\parbox{50ex}{For example, consider the ``Hopf link''.

\medskip

We now know that the bracket polynomial of this is:

\qquad $-A^4 - A^{-4}$.

The writhe will be $+2$ or $-2$, depending on which
orientation we choose.  Let's suppose we get $+2$.

So we get

\qquad $V = (-A^3)^{-2} \big( -A^4 - A^{-4} \big)$

\qquad\quad\ $= A^{-6} \big( -A^4 - A^{-4} \big) = - A^{-2} - A^{-10}$.
}}
\psset{linewidth=4mm}
\scalebox{0.7}{\begin{pspicture}(8,8)
  \psccurve(0.5,3)(3,5.5)(5.5,4)(3,0.5)
  \psccurve[linewidth=2mm,linecolor=white](0.5,3)(3,5.5)(5.5,4)(3,0.5)
  \psccurve(2.5,5)(5,7.5)(7.5,6)(5,2.5)
  \psccurve[linewidth=2mm,linecolor=white](2.5,5)(5,7.5)(7.5,6)(5,2.5)
  \psecurve(3,0.5)(0.5,3)(3,5.5)(5.5,4)(3,0.5)
  \psecurve[linewidth=2mm,linecolor=white](3,0.5)(0.5,3)(3,5.5)(5.5,4)(3,0.5)
\end{pspicture}}

\bigskip\medskip

Using your previous work, can you find the Jones polynomials of the following knots?

%\bigskip

\scalebox{0.7}{
\begin{pspicture}(8,8)
  %\psgrid[subgriddiv=1, griddots=6, gridlabels=7pt](0,0)(8,8)
  \psccurve(0.5,4)(3,0.5)(5,2)(3,4.5)(2.5,6)(4,7.5)(5.5,6)(5,4.5)(3,2)(5,0.5)(7.5,4)(4,6)
  \psccurve[linewidth=2mm,linecolor=white](0.5,4)(3,0.5)(5,2)(3,4.5)(2.5,6)(4,7.5)(5.5,6)(5,4.5)(3,2)(5,0.5)(7.5,4)(4,6)
  \psecurve(0.5,4)(3,0.5)(5,2)(2.5,6)
  \psecurve[linewidth=2mm,linecolor=white](0.5,4)(3,0.5)(5,2)(2.5,6)
  \psecurve(5.5,6)(5,4.5)(3,2)(5,0.5)
  \psecurve[linewidth=2mm,linecolor=white](5.5,6)(5,4.5)(3,2)(5,0.5)
  \psecurve(5,2)(3,4.5)(2.5,6)(4,7.5)
  \psecurve[linewidth=2mm,linecolor=white](5,2)(3,4.5)(2.5,6)(4,7.5)
  \psecurve(5,0.5)(7.5,4)(4,6)(0.5,4)
  \psecurve[linewidth=2mm,linecolor=white](5,0.5)(7.5,4)(4,6)(0.5,4)
\end{pspicture}
\qquad\qquad
\begin{pspicture}(8,8)
  \psccurve(0.5,4)(3,0.5)(5,2)(3,4.5)(2.5,6)(4,7.5)(5.5,6)(5,4.5)(3,2)(5,0.5)(7.5,4)(4,6)
  \psccurve[linewidth=2mm,linecolor=white](0.5,4)(3,0.5)(5,2)(3,4.5)(2.5,6)(4,7.5)(5.5,6)(5,4.5)(3,2)(5,0.5)(7.5,4)(4,6)
  \psecurve(5,4.5)(3,2)(5,0.5)(7.5,4)
  \psecurve[linewidth=2mm,linecolor=white](5,4.5)(3,2)(5,0.5)(7.5,4)
  \psecurve(3,0.5)(5,2)(3,4.5)(2.5,6)
  \psecurve[linewidth=2mm,linecolor=white](3,0.5)(5,2)(3,4.5)(2.5,6)
  \psecurve(4,7.5)(5.5,6)(5,4.5)(3,2)
  \psecurve[linewidth=2mm,linecolor=white](4,7.5)(5.5,6)(5,4.5)(3,2)
  \psecurve(7.5,4)(4,6)(0.5,4)(3,0.5)
  \psecurve[linewidth=2mm,linecolor=white](7.5,4)(4,6)(0.5,4)(3,0.5)
\end{pspicture}
}

\smallskip

\scalebox{0.75}{\begin{pspicture}(8,8)
  \psccurve(0.5,2.5)(2,0.5)(4.5,2.5)(5.5,5.5)(4,7.5)(2.5,5.5)(3.5,2.5)(6,0.5)(7.5,2.5)(4,5)
  \psccurve[linewidth=2mm,linecolor=white](0.5,2.5)(2,0.5)(4.5,2.5)(5.5,5.5)(4,7.5)(2.5,5.5)(3.5,2.5)(6,0.5)(7.5,2.5)(4,5)
  \psecurve(0.5,2.5)(2,0.5)(4.5,2.5)(5.5,5.5)
  \psecurve[linewidth=2mm,linecolor=white](0.5,2.5)(2,0.5)(4.5,2.5)(5.5,5.5)
  \psecurve(4,7.5)(2.5,5.5)(3.5,2.5)(6,0.5)
  \psecurve[linewidth=2mm,linecolor=white](4,7.5)(2.5,5.5)(3.5,2.5)(6,0.5)
  \psecurve(6,0.5)(7.5,2.5)(4,5)(0.5,2.5)
  \psecurve[linewidth=2mm,linecolor=white](6,0.5)(7.5,2.5)(4,5)(0.5,2.5)
\end{pspicture}
\qquad\qquad
\begin{pspicture}(8,8)
  \psccurve(0.5,2.5)(2,0.5)(4.5,2.5)(5.5,5.5)(4,7.5)(2.5,5.5)(3.5,2.5)(6,0.5)(7.5,2.5)(4,5)
  \psccurve[linewidth=2mm,linecolor=white](0.5,2.5)(2,0.5)(4.5,2.5)(5.5,5.5)(4,7.5)(2.5,5.5)(3.5,2.5)(6,0.5)(7.5,2.5)(4,5)
  \psecurve(2,0.5)(4.5,2.5)(5.5,5.5)(4,7.5)
  \psecurve[linewidth=2mm,linecolor=white](2,0.5)(4.5,2.5)(5.5,5.5)(4,7.5)
  \psecurve(7.5,2.5)(4,5)(0.5,2.5)(2,0.5)
  \psecurve[linewidth=2mm,linecolor=white](7.5,2.5)(4,5)(0.5,2.5)(2,0.5)
  \psecurve(2.5,5.5)(3.5,2.5)(6,0.5)(7.5,2.5)
  \psecurve[linewidth=2mm,linecolor=white](2.5,5.5)(3.5,2.5)(6,0.5)(7.5,2.5)
\end{pspicture}}

\end{document}
